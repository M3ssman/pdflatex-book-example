\documentclass[12pt,a4paper,onecolumn,twoside,ngerman]{book}

\usepackage[a4paper,left=3.5cm,right=2.5cm,bottom=3cm,top=3cm]{geometry}
\usepackage[ngerman,english]{babel}

% farben
\usepackage{color}

% Korrekte Darstellung der Umlaute
\usepackage[utf8]{inputenc}
\usepackage[T1]{fontenc}

% upper case first letter
\usepackage{lettrine}


% personen
% tern und enland
\newcommand{\Tern}{Tern }
\newcommand{\Ternweg}{{\Tern}weg}
\newcommand{\Sena}{Sena}
\newcommand{\Sturmkind}{Sturmkind}
\newcommand{\Daimon}{Daimon}
\newcommand{\Bangiri}{Bangiri}
\newcommand{\Pato}{Pato}
\newcommand{\Papato}{Papato}
\newcommand{\Arwed}{Alfried von \Tern}
\newcommand{\Enland}{Enland}
\newcommand{\Schattenlaufer}{Schattenläufer}

% schattenjager
\newcommand{\Eno}{Eno}
\newcommand{\Bomar}{Bomar}
\newcommand{\Dolo}{Dolo}
% Schwiegersohn von Eno
\newcommand{\Nox}{Nox}
% Frau von \Nox
\newcommand{\Mena}{Mena}
% Kind 1 von \Nox und \Mena
\newcommand{\Umbra}{Umbra}
% Kind 2+3 von \Nox und \Mena
\newcommand{\Enna}{Enna}
\newcommand{\Enno}{Enno}
% ?
\newcommand{\Lobo}{Lobo}
% Freund von \Umbra ?
\newcommand{\Tremor}{Tremor}

% lobarn
\newcommand{\Lobarn}{Lobarn}
\newcommand{\Vester}{Vester}
\newcommand{\Naimo}{Naimo}

% nordmark
\newcommand{\Nordmark}{Nordmark}
\newcommand{\Bergmark}{Bergmark}
\newcommand{\Ipes}{Ipes}
\newcommand{\Bron}{Bron}
\newcommand{\Bornhold}{Bornhold}
\newcommand{\Arn}{Arn}
\newcommand{\Eislaufer}{Eisläufer}
\newcommand{\Eisbestien}{Eisbestien}
\newcommand{\Theodora}{Theodora}

% rhinland
\newcommand{\Rhinland}{Rhinland}
\newcommand{\Rhingell}{Rhingell}
\newcommand{\Mundis}{Mundis}
\newcommand{\Helin}{Helin}
\newcommand{\Golrin}{Golrin}
\newcommand{\Rhinburg}{Rhinburg}
\newcommand{\Rhin}{Rhin}
% personen
\newcommand{\Habino}{Habino}
\newcommand{\Valem}{Valem}
\newcommand{\Palemus}{Palemus}
\newcommand{\Kalemus}{Kalemus}
\newcommand{\Isodoris}{Isodoris}

% freiberge
\newcommand{\Freiberge}{Freiberge}
\newcommand{\Sudern}{Südern}
\newcommand{\Nachtspringe}{Nachtspringe}

% bergmark
\newcommand{\Kogida}{Kogida}
\newcommand{\Denner}{Denner}

% dreifluss
\newcommand{\Dreifluss}{Dreifluß}
\newcommand{\Tars}{Tars}
\newcommand{\Toris}{Toris}
\newcommand{\Planis}{Planis}
\newcommand{\Grunarm}{Grünarm}

% grünland
\newcommand{\Grunland}{Grünland}
\newcommand{\Braucheln}{Braucheln}
\newcommand{\Darmis}{Darmis}
\newcommand{\Darmon}{Darmon}
\newcommand{\Riesenwald}{Riesenwald}

% eisenmacher
\newcommand{\Eisenland}{Eisenland}
\newcommand{\Eisenmeister}{Eisenmeister}
\newcommand{\Dariom}{Dariom}
\newcommand{\Abaton}{Abaton}
\newcommand{\Safir}{Safir}

% oedland
\newcommand{\Staubteufel}{Staubteufel}

\begin{document}

 % Sprache
  \selectlanguage{ngerman}
  
    % arabische Seitezahlen
  \pagenumbering{arabic}
  
  
  % Inhaltsverzeichnis
  \tableofcontents
  
  % Kapitel

% kapitel1.tex

\paragraph{}
\textit{Es war einmal ein Sturm.}

\chapter{Nach dem Sturm}
\section{Nach dem Sturm}
Nach dem Sturm kam die Stadt \Tern nicht zur Ruhe.
Der \Schattenlaufer \Eno rettete das Kind aus den Trümmern des Grauen Turms, bevor der \Bangiri \Pato es erschlagen kann. Er flieht aus der Stadt. Zu einem alten, verlassen Hof, wo ihn \Bomar und \Dolo erwarten.

\section{Das geheime Lager}
Weiter nach Süden. Sie erreichen ein Dorf. Viele Häuser sind verlassen. In einem großen Versteck treffen die \Schattenlaufer auf viele andere \Schattenlaufer und ihre Familien. \Nox hat hier das Sagen. Die Zwillinge \Enna und \Enno , \Nox Kinder, wollen unbedingt bei den Großen mitmachen. Sie sind enttäuscht, dass man sie nicht lässt. Das Mädchen erholt sich. \Nox gibt ihm den Namen \Sena, weil es am 7.Tag der 7.Woche des Jahres gerettet wurde.

\section{\Pato{s} Rache}
Die Übrigen \Bangiri durchstreifen das \Enland . Der Anführer \Pato hat erfahren, dass ein \Schattenlaufer in den Ruinen von \Tern waren. Darum glauben die \Bangiri{,} dass die \Schattenlaufer und das Mädchen etwas mit der Zerstörung des Grauen Turms zu tun haben. Die \Schattenlaufer und die \Bangiri bekriegen sich, weil sie durch die Graue Herrin und ihre Helfer, die \Bangiri , aus ihren alten Wohnungen im \Enland vertrieben wurden. 

\section{Der Aufbruch}
Man beschließt, dass die \Schattenlaufer und das Kind nicht bleiben. Das Lager soll verlassen werden. Man rechnet mit umherziehenden \Bangiri im \Enland . Die Lage ist undurchsichtig. Die Familien sollen in die \Nordmark gehen. Eine kleine Gruppe mit \Eno , \Dolo , \Tremor , \Enna und \Enno soll das Kind nach \Lobarn bringen. Dort sollen sie den Vorsteher \Vester um Rat fragen.

\section{Der Weg nach Norden}
Nachdem sie den Bergkamm erreicht hatten, folgten sie einem alten Pfad, der im Bogen nach Norden führt. Bei der Kreuzung mit dem alten \Ternweg bemerken sie \Bangiri aus Richtung \Tern, die nach Osten Richtung \Lobarn ziehen. Die \Schattenlaufer, die das Mädchen nach \Lobarn bringen, werden kurze Zeit später an einer alten Mauer eingekreist. Die \Schattenlaufer sind schon geschlagen, da prallt der \Bangiri, der das Mädchen packen will, plötzlich zurück und die Angreifer fliehen. Die \Schattenlaufer verstehen nicht, was geschehen ist. \Sena kann es nicht erklären. Sie setzen ihren Weg fort.

\section{Neuigkeiten}
Die Stadt \Lobarn liegt hinter einer Brücke über dem Blauen \Rhin auf einem Hügel.
\Eno verkündete die Nachrichten aus dem \Enland. Dadurch entsteht Unruhe. \Vester lässt sie in ein Dorf außerhalb bringen, wo sie bei \Naimo, der Ortsvorsteherin, warten. \Naimo erzählt die traurige Geschichte der Grauen Herrin und wie das \Enland verloren wurde.

\section{Die \Bangiri kommen}
Eine Rotte von \Bangiri zieht nach \Lobarn. Die \Bangiri verlangten, dass die Leute alle \Schattenlaufer herausgeben, die sich in der Stadt aufhalten. Und das Wesen, was in der silbernen Röhre am Grauen Turm lag.

\section{Das Geschäft mit \Lobarn}
Das Kind wollen sie in \Lobarn nicht haben. \Vester bietet ihnen an, drei Boten nach \Rhingell zu schicken, die das Mädchen zum König bringen. Der König versteht sich nicht gut mit den \Schattenlaufer, die in seinen Augen Räuber und Gesetzlose sind. Im Gegenzug müssen die \Schattenlaufer einen kleinen \Bangiri mitnehmen, der vor einigen Tagen südlich der Stadt gefangen wurde. 
Wenn die \Schattenlaufer zusammen mit dem kleinen \Bangiri und dem \Sturmkind \Sena aus der Stadt verschwinden, dann hoffen sie, keine Probleme mit den verbliebenen \Bangiri oder \Rhingell{s} neuem König zu bekommen. \Eno ist aber ein Todfeind der \Bangiri{,} egal wie alt sie sind. Aber er schlägt ein, wie es die alte \Naimo geraten hat. Sie fliehen in der Nacht weiter nach Osten, nach der Stadt \Mundis am Rand der \Nordmark .

% kapitel2.tex
% \chapter{Kapitel 2: Nach \Rhingell}

\section{Nach \Mundis}
Vor der Stadt \Mundis treffen sie auf Gesandte der \Nordmark mit dem Gesandten \Arn von \Ipes{.} Die \Nordmark wird vom Herrn von \Bornhold im Auftrag der Herren von \Rhingell regiert. Dort ist man in Sorge, weil sich im Nordwesten immer mehr Eiswesen sammeln.

Der Herr von \Bornhold lag bis zuletzt im Streit mit der Herrin des Grauen Turms, weil sie beim Raubzug der Eiswölfe erst in den Kampf  eintrat, da die \Nordmark, die \Bergmark und der Westen \Rhingell{s} verwüstet waren, dass sich die Eiswölfe ins \Enland begaben, wo sie jedoch von der Macht der Grauen Herrin und ihren Dienern, den \Bangiri, völlig ausgelöscht wurden. Die Leute aus der \Nordmark verstehen sich gut mit den \Schattenlaufer{n}. Viele aus dem \Enland sind in die \Nordmark gegangen. Sie verteidigten die neue Heimat beim Einfall der Eiswölfe. 

\Eno verlässt die Gruppe zurück nach \Lobarn . \Arn sendet einen Späher mit \Eno und schickt einen Boten zum Herrn von \Bornhold . \Dolo soll diesen ein Stück begleiten und mit \Enno bei den \Schattenlaufer Familien am Rand der \Nordmark nach dem Rechten sehen. Die Boten aus \Lobarn , \Tremor , \Sena und \Enna ziehen mit \Arn Gefolge weiter nach \Rhingell. 

\section{Die Ungehörigen}

Die Reise von \Mundis nach \Rhin führt nah an \Golrin vorbei. Dort hat sich nach dem Krieg ein Ritter niedergelassen, der sich nicht mehr dem Willen des Königs beugt. Er stammt aus der Stadt und war verbittert über ihre Zerstörung. Die Gesandtschaft folgt dem Fluss \Rhin . Sie sehen die Stadt \Helin am anderen Ufer, die auf einem schrägen Hügel erbaut von hohen Mauern geschützt wird. Im Krieg hatten die Leute dort das Glück, dass die \Eislaufer den Fluss nicht überqueren konnten. So blieben sie vom Schlimmsten verschont. Doch von den höchsten Türmen der Stadt konnte man den Rauch sehen, der tagelang aus der Stadt \Golrin aufstieg. \Arn erzählt vom Krieg mit den Eisleuten . 

\section{Die Stadt der zwei Türme}

Der Gesandte und die \Schattenlaufer kommen mit \Sena nach \Rhin, der Hauptstadt \Rhingell{s}. Die \Schattenlaufer bleiben außerhalb der Stadt, die anderen betreten sie. Sie werden von \Habino , dem Vertreter von \Lobarn , empfangen. Dort müssen sie 3 Tage warten, ehe sie zum Hofmarschall gelassen werden. \Theodora , eine Begleiterin von \Arn , erzählt Geschichten vom alten König und dem Verhältnis zwischen dem \Enland , der Grauen Herrin, \Rhinland und \Arwed von \Tern. Der weise \Valem erzählt vom neuen König \Palemus dem 13. und seinem Bruder \Kalemus , der eigentlich König sein sollte. Der Gesandte \Arn aus der \Bergmark geht mit den Boten aus \Lobarn , \Sena und dem jungen \Bangiri zum Hofmarschall \Isodoris.

\section{Empfangen und wieder nicht}
Hofmarschall \Isodoris hört ihre Geschichte an. Er will sie nur zum König lassen, wenn sie sich würdig erweisen. Dazu sollen sie nach Norden und an der Grenze zwischen \Nordmark und \Bergmark einer Räuberbande das Handwerk legen. \Arn ist verzweifelt. Vor der Stadt trifft er auf \Tremor , \Umbra und \Enna . Sie beschließen, nach Norden zu ziehen.

\section{Die Kinder von Kogida}
Die Räuber rauben Kinder aus der Stadt \Kogida und umliegenden Dörfern. Mit geheimnisvoller Hilfe können sie die Räuber vertreiben und bringen den Räuberhauptmann \Denner zurück nach \Rhin.

\section{Die Herren von \Rhingell}
Der Hofmarschall bringt sie zum König. Der junge König weiß nicht, was er machen soll. Er fürchtet das kleine Wesen. Und mit den \Bangiri will er nichts zu tun haben. Er schickt sie einfach fort. Hofmarschall \Isodoris vermittelt \Arn ein Treffen im Ältestenrat. 

\section{Gerüchte in der Stadt}
Der Ältestenrat ist mit dem jungen König unzufrieden. Er hat es versäumt, wie schon sein Vater, die Graue Herrin in ihre Schranken zu weisen und sie daran zu erinnern, für die Leute des \Enland{es} zu sorgen anstelle der \Bangiri. Um die Sicherheit der Wege in die \Bergmark im Norden und im \Rhinland im Westen sei es schlimm bestellt. Berichte über plündernde \Bangiri und \Eislaufer in der \Nordmark bringen ihn nicht zum Handeln. Auf Berichte aus der \Bergmark, dass sich \Eisbestien versammeln, reagiert er nicht.
Der Ältestenrat beschließt ohne das Wissen des jungen Königs, eine Gesandtschaft zur Stadt \Toris am \Dreifluss zu schicken. Man will den König durch seinen Bruder aus dem Süden ersetzen. Die Gesandten sollen weiter zu den \Eisenmeister{n} ziehen, um diese um Beistand zu bitten. Die Kleine \Sena soll daran teilnehmen. Der Herr der \Eisenmeister, eine sehr weise Kreatur, soll sich ein Urteil über sie bilden. Zusätzlich ziehen wieder die 4 \Schattenlaufer mit ihnen, die außerhalb der Stadt gewartet haben.

\section{\Toris}
Die Stadt \Toris liegt wenige Tage von \Rhin entfernt vor der Mündung des \Rhingell in den \Dreifluss. Auf der gegenüberliegenden Seite am \Dreifluss ist ein Handelsposten der \Eisenmeister.

\section{Der Rat von \Toris}
Die Gesandtschaft berät mit dem Ältestenrat der Stadt \Toris. Auch dort ist man beunruhigt. Ein schweres Hochwasser verbietet jedoch eine Fahrt über den Fluss. Es wird beschlossen, nach Süden am \Dreifluss nach \Braucheln am See zu reisen, um dort über das Wasser zu setzen.

\section{Nach Süden}
Nach vier Tagen erreicht die Gruppe die kleine Stadt \Planis vorm \Grunarm. Dort ist man sehr beunruhigt. Wilde Kreaturen streifen bis an die Brücke heran, die nach Süden über den Fluss \Grunarm führt. Seit Tagen wagt sich niemand mehr auf die Felder im Süden. Die Streitmacht der Herren von \Rhingell zog vor 28 Tagen vorbei nach Süden. Seither kamen keine Nachrichten mehr die Straße zurück.
 
% kapitel3\textbf{•}.tex
% \chapter{Das \Grunland}
\section{Das Fort}
Unter großer Vorsicht zieht die Gruppe in der Dämmerung den Weg weiter nach Süden. Als der Tag anbricht, erreichen sie ein verwüstetes und verlassenes Fort am Fluss. Sie verstecken sich dort tagsüber. In der Dämmerung ziehen sie weiter. Ein \Schattenlaufer wird zurück nach \Planis gesandt, um die Nachricht vom zerstörten Fort zu verbreiten.

\section{Gefahr am \Dreifluss}
Hinter dem Fort reicht der Wald über den Weg bis an den \Dreifluss. Bevor der Morgen kommt, erreicht die Gruppe eine zerstörte Brücke. Man teilt sich auf: ein Teil versucht, den Übergang herzurichten, ein zweiter will schauen, ob in der Nähe ein leichterer Übergang zu finden sei. Bei dem Versuch, ein Stück weiter waldeinwärts einen Übergang über die Schlucht zu finden, wird diese Gruppe von Wilden angegriffen. Anschließend wird die gesamte Gesandtschaft zwischen \Dreifluss und \Riesenwald zertreut. Das Wesen wird gestoßen und fällt in ein Loch.

\section{Im \Riesenwald}
Das Mädchen wacht auf. Die Sonne steht im Sie ist ganz allein im Wald. Sie geht weiter und weiter am Rand des Abgrundes, bis die Felsen niedriger werden. Unten zwängt sie sich zwischen Baumstümpfen und Steinen hindurch an einen Teich, wo viele wilde Kreaturen eine riesige, drachenähnliche Gestalt umzingelt haben. Es steht ein furchtbarer Kampf bevor.

Die wilden Kreaturen bemerkten das Mädchen nicht. Sie ging zu dem riesigen Wesen, es senkte seinen Kopf, sie streckte die Hand aus und legte sie auf die Stirn und es verwandelte sich in ein kleines Kind. Die wilden Kreaturen ließen von den beiden ab.

\section{Zurück zum Fluß}
\Nox, einer der \Schattenlaufer, hielt sich die ganze Zeit am Rand versteckt. Nachdem die Kreaturen fort waren, nahm er das \Sturmkind und das Kleine und brachte sie über den Fluss und die Schlucht zurück auf den Weg. Dort trafen sie weitere \Schattenlaufer und den Diener des Gesandten zum Herren der \Eisenmeister. Der Gesandte ist fort. Trotzdem zogen sie in der Dämmerung vorsichtig weiter nach Süden.

\section{Spuren}
An den Fällen des \Dreifluss sehen sie im Morgengrauen in der Ferne die Ruinen von \Darmon, und noch weiter dahinter die Rauchsäulen von \Dariom, der Stadt der Erzgräber am gegenüberliegenden Ufer des Sees. Sie ziehen auf dem Weg weiter nach Süden und durchqueren gegen Mittag ein zerstörtes Dorf. Die Asche ist kalt.

\section{\Braucheln}
Sie eilen weiter. In der Ferne sehen sie die Stadt \Braucheln am See. Rauch liegt über dem Tor zum Ufer, davor wogt ein Kampf. Als die \Schattenlaufer dazu stoßen, geht das \Sturmkind zu den Kreaturen. Sie lassen ab und ziehen in den Wald zurück.

% kapitel4.tex
% \chapter{Die \Eisenmeister}
\section{Ankunft in \Braucheln}
In \Braucheln trafen sie den Bruder des Herren von \Rhingell. Er berichtet über Angriffen von Kreaturen aus den Wäldern, Die Stützpunkte der Holzfäller am Waldrand seien alle verlassen. Es wird seit Wochen kein Holz mehr geschlagen und keine Holzkohle gemacht und über den See geschafft. Es weilt bereits \Safir, ein Gesandter des Herrn der \Eisenmeister, in \Braucheln, der sich über die ausbleibenden Lieferungen erkundigen sollte.

\section{Auf dem Weg nach \Dariom}
Der Gesandte, einige \Schattenlaufer, das \Sturmkind und das Kleine aus dem \Riesenwald folgen dem Abgesandten der \Eisenmeister über das Wasser in die Stadt der \Eisenmeister.

Die Stadt der \Eisenmeister ist riesig und erstreckt sich von den Fällen des \Dreifluss, der mit seinem Wasser riesige Räder für die Erzschmieden antreibt, weit am Ufer und den Felsen nach Süden, wo die Handelswege in die Östlichen und Südlichen Lande verlaufen.

\section{Beim Herren der \Eisenmeister}
Der Herr der \Eisenmeister weist die Hilfegesuche ab. Die Streitereien der Leute aus \Rhingell gehen die \Eisenmeister nichts an. Man hat bereits genug Problem mit den Einfällen der Vielbeinigen über das nördliche Ödland und im Süden. Aber er bietet dem \Sturmkind an, es dem geheimen Herren des Erzes zu präsentieren.

\section{Der Herr des Erzes}
Die hohe Pforte des \Abaton öffnete ihre Flügel. Sie traten in eine Höhle, die so riesig war, dass man weder die Decke, noch links noch rechts ein Ende sehen konnte. Direkt vor den Besuchern begann ein Weg aus funkelnden Edelsteinen, an dem links und rechts silberne Säulen standen. Die Säulen reichten so weit hinauf, dass sie in der Finsternis verschwanden.
Am Ende der Säulenreihen erhob sich ein grauer Thron. Auf beiden Seiten standen riesige Statuten mit glänzenden Hämmern und Hacken anstelle von Händen. Auf dem Thron ruhte eine kleine Gestalt mit silbernen Gewändern und silbernen Haaren. „Kommt näher!“ donnerte es durch die Halle, ohne dass die Gestalt auf dem Thron sich geregt hätte. „Seht den geheimen Herrn des Berges!“ 

„Herr“, begann sie ängstlich. „Ich weiß nicht, wo ich hin soll. Und wo ich herkomme.“
„Kind“, antwortete der geheime Herr und es klang fast, als würde die Stimme lachen, „Was denkst du? Du bist doch keine Gabel, die jemand aus Versehen an den falschen Platz gelegt hat.“

Der geheime Herr des Berges Erzes sitzt tief im Berg bei der Stadt der Erzgräber in einer weiten Höhle. Bewacht von seiner Garde, zweimal zwölf Eisenwesen, sitzt er auf seinem eisernen Thron. Seine Augen sind trüb, aber er ist mit dem Stein verbunden und sieht Dinge, die sind und die sein können, aber nur schwach das Echo der Dinge, die einmal waren. Und dass die Gier nach Holz die Übergriffe im \Riesenwald erzeugt. Und er spürt vor allem, dass ein Kampf zwischen den Brüdern um die Herrschaft über \Rhingell bevorsteht. Darum müssen das \Sturmkind und die Gesandten schnell zurück nach \Toris am \Dreifluss. Wenn wieder Ordnung in \Rhingell einkehrt, wird auch wieder Holz in die Stadt der Erzgräber kommen. Der Gesandte \Safir der Erzgräber und ein Eisenwesen sollen sie begleiten.

\section{Zurück nach \Rhingell}
Auf dem Rückweg auf dem anderen Ufer des Dreiflusses durchqueren sie ein ödes Land. Sie kommen durch einen kleinen Posten der \Eisenmeister. 

% kapitel5.tex
%\chapter{Pi}

% kapitel6.tex
%\chapter{Ex}

% kapitel7.tex
%\chapter{Sen}
\section{Bruderkrieg}
Das Hochwasser ist so weit zurückgegangen, dass der \Dreifluss bei \Toris wieder überquert werden kann. In \Toris erfährt die Gesandtschaft, dass der ältere Bruder bereits vorbeizog und verstärkt mit vielen Kämpfern aus \Toris nach \Rhin rückt. 

In \Rhingell hat sich der junge Herr in der Burg verschanzt. Die Stadt, das ganze Land hat sich gegen ihn gestellt. 

\section{Der Zug des \Pato}
Da rückt von Westen eine riesige Schar \Bangiri gegen die Stadt. Sie fordern die Herausgabe aller \Schattenlaufer in der Stadt.

\section{{\Eno}s Entscheidung}
\Eno geht allein zu den \Bangiri unter der Bedingung, dass \Papato abzieht. \Nox wird der neue Anführer der Schattenläufer.

\section{Die Rettung \Rhingell{s}}
Das \Sturmkind nimmt den älteren und jüngeren Herrn bei der Hand und verwandelt beide. Dann nimmt sie das Kleine aus dem \Riesenwald und erklärt es zum neuen Herrn von \Rhingell und \Grunland. Der junge König legt die Krone nieder um sich dem \Sturmkind anzuschließen. Er hat noch nie die Mauern \Rhin{s} verlassen. Dieses soll seine erste Reise werden.

\section{Der neue Herr der \Eisenmeister}
Das \Sturmkind geht mit \Safir und \Umbra zurück zum geheimen Herren der Erzgräber. Dieser wird eins mit dem Stein und der jetzige Herr der \Eisenmeister nimmt seinen Platz sein. Er verkündet, dass das \Sturmkind nach der Ruine von \Darmon gehen soll um dort zu erfahren, wo sein Platz ist.

\section{\Darmon}
Die Höhlen nach \Darmon kennen kein Licht. In der Finsternis begegnet das \Sturmkind drei \Daimon, die verschwinden, sobald sie diese anruft.
In den Ruinen von \Darmon ist die Geschichte der Welt in funkelnden Steinchen gelegt. Dort wird klar, dass der Wald vor langer Zeit auch auf dieser Seite des Flusses wuchs, ehe die \Eisenmeister ihn vollständig in ihrer Gier verbrannten. Dadurch ragt nun auch das östliche Ödland bis an den Fluss heran, wo früher hohe Bäume standen. Die \Eisenmeister kämpften in diesen Tagen auch gegen Kreaturen aus dem Wald und konnten dem nur Herr werden, in dem sie alles Holz auf dieser Seite des Flusses verbrannten. Das \Sturmkind soll weiter nach Süden und Osten gehen, in den Mittelpunkt der Welt, wo die Säule steht, die den Himmel trägt. Wenn sie dort hinaufsteigt, soll ihren Platz in der Welt sehen.

\end{document}
