\documentclass[12pt,a4paper,onecolumn,twoside,ngerman]{book}

\usepackage[a4paper,left=3.5cm,right=2.5cm,bottom=3cm,top=3cm]{geometry}
\usepackage[ngerman,english]{babel}

% farben
\usepackage{color}

% Korrekte Darstellung der Umlaute
\usepackage[utf8]{inputenc}
\usepackage[T1]{fontenc}

% upper case first letter
\usepackage{lettrine}


% personen
% tern und enland
\newcommand{\Tern}{Tern }
\newcommand{\Beron}{Beron}
\newcommand{\Molitor}{Molitor}
\newcommand{\Ternweg}{{\Tern}weg}
\newcommand{\Sena}{Sena}
\newcommand{\Sturmkind}{Sturmkind}
\newcommand{\Daimon}{Daimon}
\newcommand{\Bangiri}{Bangiiri}
\newcommand{\Pato}{Pato}
\newcommand{\Oggo}{Oggo}
\newcommand{\Papato}{Papato}
\newcommand{\Arwed}{Alfried von \Tern}
\newcommand{\Enland}{Enland}
\newcommand{\Enlaender}{Enländer}
\newcommand{\Schattenlaufer}{Schattenläufer}

% schattenjager
\newcommand{\Eno}{Eno}
\newcommand{\Bomar}{Bomar}
\newcommand{\Dolo}{Dolo}
% Schwiegersohn von Eno
\newcommand{\Nox}{Nox}
% Frau von \Nox
\newcommand{\Mena}{Mena}
% Kind 1 von \Nox und \Mena
\newcommand{\Umbra}{Umbra}
% Kind 2+3 von \Nox und \Mena
\newcommand{\Enna}{Enna}
\newcommand{\Enno}{Enno}
% ?
\newcommand{\Lobo}{Lobo}
% Freund von \Umbra ?
\newcommand{\Tremor}{Tremor}

% lobarn
\newcommand{\Lobarn}{Lobarn}
\newcommand{\Vester}{Vester}
\newcommand{\Naimo}{Naimo}

% nordmark
\newcommand{\Nordmark}{Nordmark}
\newcommand{\Bergmark}{Bergmark}
\newcommand{\Ipes}{Ipes}
\newcommand{\Bron}{Broon}
\newcommand{\Bornhold}{Bornhold}
\newcommand{\Eishold}{Eishold}
\newcommand{\Arn}{Arn}
\newcommand{\Eislaufer}{Eisläufer}
\newcommand{\Eisleute}{Eisleute}
\newcommand{\Eisbestien}{Eisbestien}
\newcommand{\Theodora}{Theodora}

% rhinland
\newcommand{\Rhinland}{Riinland}
\newcommand{\Rhingell}{Riingell}
\newcommand{\Blaufurt}{Blaufurt}
\newcommand{\Mundis}{Mundis}
\newcommand{\Helin}{Heelin}
\newcommand{\Golrin}{Golriin}
\newcommand{\Galadin}{Galaadin}
\newcommand{\Rhinburg}{Riinburg}
\newcommand{\Rhin}{Riin}

% personen
\newcommand{\Habino}{Habino}
\newcommand{\Valem}{Vaalem}
\newcommand{\Palemus}{Paleemus}
\newcommand{\Kalemus}{Kaleemus}
\newcommand{\Isodoriin}{Isodoriin}
\newcommand{\Galeon}{Galeon}
\newcommand{\Demea}{Demea}

% freiberge
\newcommand{\Freiberge}{Freiberge}
\newcommand{\Sudern}{Südern}
\newcommand{\Nachtspringe}{Nachtspringe}
\newcommand{\Schwarzberge}{Schwarzberge}

% bergmark
\newcommand{\Kogida}{Koggida}
\newcommand{\Denner}{Denner}
\newcommand{\Eishohle}{Eishoole}

% dreifluss
\newcommand{\Dreifluss}{Dreifluß}
\newcommand{\Tars}{Taars}
\newcommand{\Toris}{Tooris}
\newcommand{\Planis}{Plaanis}
\newcommand{\Grunarm}{Grünarm}

% grünland
\newcommand{\Grunland}{Grünland}
\newcommand{\Braucheln}{Braucheln}
\newcommand{\Darmis}{Daarmis}
\newcommand{\Darmon}{Daarmon}
\newcommand{\Riesenwald}{Riisenwald}

% eisenmacher
\newcommand{\Eisenland}{Eisenland}
\newcommand{\Eisenmeister}{Eisenmeister}
\newcommand{\Dariom}{Daariom}
\newcommand{\Abaton}{Abbaton}
\newcommand{\Safir}{Saafir}

% oedland
\newcommand{\Staubteufel}{Staubteufel}

\begin{document}

 % Sprache
  \selectlanguage{ngerman}
  
    % arabische Seitezahlen
  \pagenumbering{arabic}
  
  
  % Inhaltsverzeichnis
  \tableofcontents
  
  % Kapitel

% kapitel1.tex

%\begin{huge}
\paragraph{}
\textit{Es war einmal ein Sturm.}

\chapter{}
\section{Nach dem Sturm}
Nach dem Sturm kommt die Stadt \Tern\ nicht zur Ruhe.
Der \Schattenlaufer\ \Eno\ rettete ein Menschenkind aus den Trümmern des Grauen Turms, bevor der \Bangiri\ \Pato\ es erschlagen kann. \Eno\ flieht aus der Stadt, die Wilden verfolgen ihn. Zu einem alten, verlassen Hof, wo ihn \Bomar\ und \Dolo\ erwarten. 

\section{Das geheime Lager}
Weiter nach Südosten am Seeufer entlang. Sie erreichen unweit des \Tern\ Sees ein Dorf mit Namen \Beron\. Viele Häuser sind verlassen. In einem Versteck treffen die \Schattenlaufer\ auf weitere \Schattenlaufer\ und deren Familien. Der Dorfvorsteher, \Nox\ , ist selbst \Schattenlaufer\ . \Eno\ ist eine hochgestellte Person der \Schattenlaufer\ .
Die Zwillinge \Enna  und \Enno , \Nox  Kinder, wollen unbedingt \Schattenlaufer  werden. Sie sind enttäuscht, dass man sie nicht lässt, weil sie zu jung sind. Das Mädchen erholt sich einige Tage. Anfangs spricht sie kein Wort. Sie versteht nicht, was die \Schattenlaufer  sagen.
\Nox  gibt dem Mädchen den Namen \Sena , weil es am 7.Tag der 7.Woche des Jahres gerettet wurde. \Sena    ist die weibliche Form der riinländischen Zahl 7. \Nox  zeigt ihr die Zahlen von 1-7. Es sind ihre ersten Worte im  \Enland .
Im Dorf \Beron  lebt der kleine \Molitor  bei seinem Großvater. Dauernd stellt er Fragen. Sein Großvater meint daher scherzhaft, er sei wohl zu dumm. Er spielt manchmal mit \Enna  und \Enno  und somit auch mit \Sena . 

\section{\Pato{s} Rache}
\Bangiri  durchstreifen das \Enland . Der Anführer, \Pato , hat erfahren, dass ein \Schattenlaufer in den Ruinen von \Tern waren. Sein Großkrieger \Oggo  hat es ihm gesagt. Darum glauben die \Bangiri , dass die \Schattenlaufer  und das Mädchen etwas mit der Zerstörung des Grauen Turms zu tun haben. Die \Schattenlaufer  und die \Bangiri  bekriegen sich. Viele \Enlaender  wurden auf Befehl der Grauen Herrin, die 99 Jahre aus dem Grauen Turm heraus über das \Enland  herrschte, durch Helfer, die \Bangiri , aus ihren alten Wohnungen im \Enland vertrieben.
 
Die wilden \Bangiri  kommen in das Dorf und treiben alle Leute zusammen, auch den Großvater von \Molitor und \Nox . Die übrigen \Schattenlaufer  um \Eno  vertreiben die Wilden, als sie die Einwohner fortführen wollen.

\section{Der Aufbruch}
Die \Bangiri  wollen mit Verstärkung zurückkehren. Die \Schattenlaufer  müssen als den Unterschlupf verlassen. Die Familien aus dem Dorf sollen in die \Nordmark  gehen. \Molitor  soll mit, auch wenn sein Großvater bleibt. Der Großvater sagt, dass er nicht der richtige Großvater ist. Die echten Eltern von \Molitor sind verschollen. Vielleicht gibt es im übrigen \Rhinland  noch Verwandte.
Eine kleine Gruppe mit \Nox , \Eno , \Dolo , \Tremor , \Enna  und \Enno  soll das Kind \Sena  nach \Lobarn  bringen. Dort sollen sie den Vorsteher \Vester  um Rat und Hilfe bitten.

\section{Der Weg nach Norden}
Bei der Kreuzung mit dem alten \Ternweg  trennen sich die Gruppen. Sie bemerken \Bangiri  aus Richtung \Tern , die nach Osten Richtung \Lobarn  ziehen. Die \Schattenlaufer , die das Mädchen nach \Lobarn  
 bringen, werden kurze Zeit später an einer alten Mauer eingekreist. Die \Schattenlaufer  sitzen in der Falle. Da erhebt sich um \Sena  und die übrigen ein Wirbelwind. Als die \Bangiri  versuchen, die Windhose zu durchschreiten, werden sie fortgerissen. Die \Schattenlaufer  verstehen nicht, was geschehen ist. \Sena  kann es nicht erklären. 
\Molitor  erscheint. Er ist von den Anderen fortgelaufen und will mit ins \Rhinland  wandern. Sie setzen ihren Weg fort.

\section{Neuigkeiten}
Die Stadt \Lobarn  liegt hinter einer Brücke über dem Blauen \Rhin  geschützt von dicken Steinmauern auf einem Hügel. Ein hoher Wachturm blickt bis an die Grenze zum \Enland .
\Eno  verkündete die Neuigkeiten aus dem \Enland  und bittet um Beistand. Dadurch entstehen Unruhen. Die Leute in \Lobarn  zweifeln, dass die Zeit der Grauen Herrin wirklich vorbei ist. Auch fürchtet man die \Bangiri  wegen ihrer Zahl. Vor \Sena  fürchten sie sich noch mehr, wenn sie tatsächlich die Grauen Herrin mit ihrer schwarzen Magie besiegt hat. 
\Vester  lässt sie in ein Dorf außerhalb bringen, wo sie bei \Naimo , der Ortsvorsteherin, warten. \Naimo
  erzählt die traurige Geschichte der Grauen Herrin und wie das \Enland verloren wurde.

\section{Die \Bangiri kommen}
Eine Rotte von \Bangiri  zieht nach \Lobarn . Ihr Anführer \Oggo  sucht \Enlaender , speziell \Schattenlaufer  und das Wesen aus den Trümmern des Grauen Turms. 

\section{Das Geschäft mit \Lobarn}
Wirkliche Hilfe gegen die Wilden kann ihnen \Lobarn nicht geben. Das Kind wollen sie nicht haben. \Vester 
  bietet ihnen drei Boten , die das Kind nach \Rhingell  begleiten sollten, um beim Königshof Hilfe zu erbitten. Allerdings versteht sich das Haus von \Rhingell nicht mit den \Schattenlaufer, die in seinen Augen Räuber und Gesetzlose sind. Dafür sollen sich die \Schattenlaufer  um einen kleinen \Bangiri   kümmern, der vor einigen Tagen südlich der Stadt eingefangen wurde. 
Wenn die \Schattenlaufer  zusammen mit dem kleinen \Bangiri  und dem \Sturmkind  \Sena  aus der Stadt verschwinden, dann hoffen sie, keine Probleme mit den verbliebenen \Bangiri , \Rhingell s neuem König oder der womöglich wiederkehrenden Grauen Herrin zu bekommen. 
\Eno  hasst \Bangiri , egal wie alt sie sind. Aber er schlägt ein, weil es \Naimo  und \Nox  anraten. Sie fliehen in der Nacht weiter nach Osten, nach der Stadt \Mundis  am Rand der \Nordmark .

% kapitel2.tex
% \chapter{Kapitel 2: Nach \Rhingell}

\section{Trennung auf Zeit}
Weit vor \Mundis  treffen sie in der Stadt \Blaufurt  auf \Bomar , \Mena  und \Umbra , die die Familien sicher in die \Nordmark  gebracht haben. Sie ziehen zusammen weiter. In \Mundis  treffen sie Gesandte der \Nordmark  mit \Arn  von \Ipes . Die \Nordmark wird vom Herrn von \Bornhold im Auftrag der Herren von \Rhingell regiert. Dort ist man in Sorge, weil sich im Nordwesten bei \Eishold  immer mehr düstere Wesen sammeln. Es sieht so aus, als bereiten die \Eisleute  einen neuen Kriegszug nach \Rhingell  vor.
Der Herr von \Bornhold  lag im Streit mit der Herrin des Grauen Turms, weil sie beim Raubzug der \Eisleute  nicht in den Kampf eintrat, als die \Nordmark , die \Bergmark  und der Westen \Rhingell{s} verwüstet wurden. Erst, als sich die \Eisleute  auch ins \Enland  begaben. Die Leute aus der \Nordmark   
  verstehen sich gut mit den \Schattenlaufer{n} . Viele Leute aus dem \Enland  sind in die \Nordmark   gegangen, um der Grauen Herrin zu entfliehen. 

\Eno  verlässt die Gruppe und geht zurück nach \Lobarn . \Arn  sendet einen Späher mit \Eno  und schickt  Boten zum Herrn von \Bornhold . \Nox  soll diesen ein Stück begleiten und mit \Enno  bei den \Schattenlaufer  Familien am Rand der \Nordmark  nach dem Rechten sehen. Die 3 Boten aus \Lobarn , \Dolo , \Tremor , \Umbra, \Sena , \Molitor  und \Enna  ziehen mit \Arn  und seinem Gefolge weiter nach \Rhingell . 

\section{Der Weg am Fluss}
Die Reise von \Mundis  nach \Rhin  führt nah an der alten Stadt \Golrin  vorbei. Dort hat sich nach dem Krieg der Ritter \Galadin  niedergelassen, der sich nicht mehr dem Willen des Königs beugt. Er stammt aus der Stadt und war verbittert über ihre Zerstörung. Die Leute vom \Galadin  ergreifen sie und bringe nsie nach \Golrin  . \Galadin  versteht sie \Schattenlaufer . Beide Parteien haben Probleme mit den Herren von \Rhingell . \Galadin  und  \Arn  erzählt vom Krieg mit den Eisleuten, in dem die \Nordmark , weite Teile des \Rhinland s und des \Enland s verwüstet wurden. 
Die Gesandtschaft folgt dem Fluss \Rhin . Sie sehen die Stadt \Helin  am anderen Ufer, die auf einem Hügel erbaut von hohen Mauern geschützt wird. Im Krieg hatten die Leute das Glück, dass die \Eislaufer  den Fluss nicht überqueren konnten. So blieben sie vom Schlimmsten verschont. Doch von den Türmen der Stadt konnte man den Rauch sehen, der tagelang aus der Stadt \Golrin   
  aufstieg. 

\section{Die Stadt der zwei Türme}
Der Gesandte und die Boten kommen mit \Sena , \Molitor  und  dem jungen \Bangiri  nach \Rhin , der Hauptstadt \Rhingell{s}. Die \Schattenlaufer  bleiben außerhalb der Stadt. Sie werden von \Habino , dem Vertreter von \Lobarn , empfangen. Dort müssen sie 3 Tage warten, ehe sie zum Hofmarschall gelassen werden. \Theodora , eine Begleiterin von \Arn , erzählt Geschichten vom alten König und dem Verhältnis zwischen dem \Enland , der Grauen Herrin, \Rhinland und \Arwed von \Tern. 
Der weise \Valem , ein Gelehrter aus der Stadt, der mit \Habino  bekannt ist, erzählt vom neuen König \Palemus dem 13. und seinem Bruder \Kalemus , der eigentlich König sein sollte. Der Gesandte \Arn  aus der \Bergmark  geht mit den Boten aus \Lobarn , \Sena  und dem jungen \Bangiri  zum Hofmarschall \Isodoriin.  \Molitor  bleibt derweil bei \Valem .

\section{Die Herren von \Rhingell}
Der junge König \Kalemus  weiß nicht, was er machen soll. Er fürchtet das kleine Wesen. Mit den \Bangiri  
  will er nichts zu tun haben. Er schickt alle wieder fort. Hofmarschall \Isodoriin vermittelt \Arn und \Habino ein Treffen im Ältestenrat. 

\section{Gerüchte in der Stadt}
Der Ältestenrat ist mit dem jungen König unzufrieden. Er hat es versäumt, wie schon sein Vater, die Graue Herrin in ihre Schranken zu weisen und sie daran zu erinnern, für die Leute des \Enland{es} zu sorgen anstelle der \Bangiri . Um die Sicherheit der Wege in die \Bergmark  im Norden und im \Rhinland  im Westen sei es schlimm bestellt. Berichte über Räuber und die Neuigkeiten über \Eisleute  aus der \Nordmark bringen ihn nicht zum Handeln. Auf Berichte aus der \Bergmark , dass sich \Eisbestien   nahe der Grenzen sammeln, reagiert er nicht.

Der Ältestenrat beschließt heimlich einen Boten, den jungen \Galeon  und seine Frau \Demea mit \Sena , dem \Bangiri  und \Molitor  zur Stadt \Toris  am \Dreifluss  zu schicken. Man will den König durch seinen Bruder \Kalemus , der vor einigen Wochen nach Süden zog, um dort zur Ordnung zu sorgen, zu ersetzen. Die Gesandten sollen zu den \Eisenmeister{n} ziehen, um diese um Beistand zu bitten. Die Kleine \Sena  soll daran teilnehmen. Der Herr der \Eisenmeister, eine sehr weise Kreatur, kann sicher helfen, ihre Herkunft zu klären. Die Boten aus \Lobarn  kehren in ihre Stadt zurück. Zusätzlich ziehen wieder die \Schattenlaufer mit ihnen, die außerhalb der Stadt gewartet haben. \Dolo  und \Tremor  führen die Gruppe.

\section{\Toris}
Die Stadt \Toris  liegt wenige Tage von \Rhin  entfernt an der Mündung des \Rhingell  in den \Dreifluss . Auf der gegenüberliegenden Seite am \Dreifluss  ist ein Handelsposten der \Eisenmeister .

\section{Der Rat von \Toris}
Die Gesandtschaft berät mit dem Ältestenrat der Stadt \Toris . Auch dort ist man beunruhigt. Ein schweres Hochwasser verbietet jedoch eine Fahrt über den Fluss. Es wird beschlossen, am \Dreifluss  nach \Braucheln  am Großen See zu reisen, um dort über das Wasser zu den \Eisenmeister n zu fahren.

\section{Nach Süden}
Nach vier Tagen erreicht die Gruppe die Stadt \Planis  vorm \Grunarm . Dort ist man sehr verängstigt. Wilde Kreaturen streifen bis an die Brücke heran, die nach Süden über den Fluss \Grunarm  führt. Seit Tagen wagt sich niemand mehr auf die Felder. Die Streitmacht der Herren von \Rhingell  zog vor 28 Tagen vorbei nach Süden. Seither kamen keine Nachrichten mehr die Straße zurück.
 
% kapitel3\textbf{•}.tex
% \chapter{Das \Grunland}
\section{Das Fort}
Unter großer Vorsicht zieht die Gruppe in der Dämmerung den Weg weiter nach Süden. Als der Tag anbricht, erreichen sie ein verwüstetes und verlassenes Fort am Fluss. Sie verstecken sich dort tagsüber. In der Dämmerung ziehen sie weiter. Ein \Schattenlaufer wird zurück nach \Planis gesandt, um die Nachricht vom zerstörten Fort zu verbreiten.

\section{Gefahr am \Dreifluss}
Hinter dem Fort reicht der Wald über den Weg bis an den \Dreifluss . Bevor der Morgen kommt, erreicht die Gruppe eine zerstörte Brücke. Man teilt sich auf: ein Teil versucht, den Übergang herzurichten, ein zweiter will schauen, ob in der Nähe ein Übergang ist. Bei dem Versuch, ein Stück weiter waldeinwärts einen Übergang über die Schlucht zu finden, werden sie von Wilden angegriffen. Anschließend wird die gesamte Gesandtschaft zwischen \Dreifluss  und \Riesenwald  zerstreut.  \Sena wird gestoßen und fällt in ein Loch.

\section{Im \Riesenwald}
\Sena  wacht auf. Sie ist allein im Wald. Sie geht weiter und weiter am Rand des Abgrundes, bis die Felsen niedriger werden. Unten zwängt sie sich zwischen Baumstümpfen und Steinen hindurch an einen Teich, wo viele wilde Kreaturen eine riesige, drachenähnliche Gestalt umringt haben. 

Die wilden Kreaturen bemerken \Sena  nicht.  Sie geht zu dem Wesen, sie legt die Hand auf die Stirn und im Sturm verwandelte sich die Kreatur in den jungen \Bangiri  . Er gibt sich als \Papato  zu erkennen, als Sohn des Anführers der \Bangiri  im \Enland . Er wollte von seinen Leuten ausreißen, wurde aber auf dem Weg nach den Schwarzbergen von Spähern aus \Lobarn  geschnappt. 

\section{Zurück zum Fluß}
\Tremor  hielt sich die ganze Zeit am Rand versteckt. Nachdem die Kreaturen ablassen, nimmt er \Sena  und \Papato  und bringt sie zurück auf den Weg. Dort treffen sie weitere \Schattenlaufer  und die Boten aus  \Rhingell .

\section{Spuren}
An den Fällen des \Dreifluss  sehen sie im Morgengrauen in der Ferne den Dunst der Nebelinsel und noch weiter dahinter die Rauchsäulen von \Dariom , der Stadt der \Eisenmeister  am gegenüberliegenden Ufer des Sees. Sie ziehen auf dem Weg weiter nach Süden und durchqueren gegen Mittag ein zerstörtes Dorf. Die Asche ist kalt.

\section{\Braucheln}
Sie eilen weiter. In der Ferne sehen sie die Stadt \Braucheln  am See. Vor dem Tor haben sich Kreaturen aus dem Wald versammelt. \Sena  geht in einem Sturmwirbel mit \Papato  zu den Kreaturen. Sie lassen ab und ziehen sich in den Wald zurück.

% \chapter{Die \Eisenmeister}
\section{Ankunft in \Braucheln}
In \Braucheln  treffen sie den Bruder des Herren von \Rhingell , \Kalemus . Er berichtet über Angriffen von Kreaturen aus den Wäldern. Es wird seit Wochen kein Holz mehr geschlagen und keine Holzkohle gemacht und über den See geschafft.  \Safir , ein Gesandter des Herrn der \Eisenmeister , ist in \Braucheln , um sich über die ausbleibenden Lieferungen in die Stadt der \Eisenmeister  auf der anderen Seite des Großen Sees erkundigen sollte.

\section{Auf dem Weg nach \Dariom}
Der Gesandte, einige \Schattenlaufer , das \Sturmkind , \Molitor  und \Papato  folgen dem Abgesandten der \Eisenmeister  über das Wasser in die Stadt der \Eisenmeister .
Die Stadt der \Eisenmeister ist riesig und erstreckt sich von den Fällen des \Dreifluss , der mit seinem Wasser riesige Räder für die Erzschmieden antreibt, weit am Ufer und den Felsen nach Süden, wo die Handelswege in die Östlichen und Südlichen Lande verlaufen.

\section{Beim Herren der \Eisenmeister}
Der Herr der \Eisenmeister  weist die Hilfegesuche ab. Die Streitereien der Leute aus \Rhingell  gehen die \Eisenmeister  nichts an. Man hat genug Problem mit den Einfällen der Vielbeinigen über das nördliche Ödland und den \Staubteufel n in den verlorenen Städten des Hochlandes. 
Er bietet dem \Sturmkind an, es dem geheimen Herren des Erzes zu präsentieren, wenn sie helfen, das Rätsel der \Staubteufel zu lösen.

\section{Das trockene Land}
\Molitor löst das Rätsel der \Staubteufel .

\section{Der Herr des Erzes}
Der geheime Herr des Berges Erzes sitzt tief im Berg \Abaton bei der Stadt der Erzgräber. Bewacht von seiner Garde, zweimal vierzehn Eisenwesen, sitzt er auf seinem eisernen Thron. Seine Augen sind trüb, aber er ist mit dem Stein verbunden und sieht Dinge, die sind und die sein können, aber nur schwach das Echo der Dinge, die einmal waren. 
Da \Sena ihnen mit den Staubteufeln beigestanden hat, schickt er sie mit einem weißen Schlüssel zurück zum König nach \Rhingell . Wenn Ruhe in \Rhingell einkehrt, wird auch wieder Holz in die Stadt der Erzgräber kommen. Der Gesandte \Safir der Erzgräber und ein Eisenwesen sollen sie begleiten. Wenn die Ordnung wieder hergestellt ist, sollen sie wiederkehren. Bis dahin will der Herr des Eisens alles in seiner Macht tun, um Antworten auf ihre Fragen zu finden.

\section{Zurück nach \Rhingell}
Auf dem Rückweg auf dem anderen Ufer des Dreiflusses durchqueren sie ein ödes Land. Sie kommen durch einen kleinen Posten der \Eisenmeister. 
Das Hochwasser ist so weit zurückgegangen, dass der \Dreifluss bei \Toris wieder überquert werden kann.

%\chapter{Pi}

%\chapter{Ex}
\section{Empfangen und wieder nicht}
Hofmarschall \Isodoriin hört ihre Geschichte an. Er will sie nur zum König lassen, wenn sie sich würdig erweisen. Dazu sollen sie nach Norden und an der Grenze zwischen \Nordmark und \Bergmark einer Räuberbande das Handwerk und als Beweis das Herz des Hexenmeisters vorlegen. \Arn ist verzweifelt. Vor der Stadt trifft er auf \Tremor , \Umbra und \Enna . Sie beschließen, nach Norden zu ziehen.

\section{Die Kinder von Kogida}
Die Räuber rauben Kinder aus der Stadt \Kogida und umliegenden Dörfern für den Hexenmeister. Mit geheimnisvoller Hilfe können sie die Räuber vertreiben und bringen den schwarzen Hexenmeister \Denner zurück nach \Rhin .

%\chapter{Sen}
\section{Bruderkrieg}
Sie kehren gerade rechtzeitig zurück. \Kalemus , der ältere Bruder, ist zusammen mit vielen Kämpfern aus \Toris nach \Rhin gezogen. In \Rhingell hat sich der junge König in der Burg verschanzt. Die Stadt, das ganze Land hat sich gegen ihn gestellt. 

Eine große Verschwörung wird von \Nox enthüllt.

\section{Der Zug aus dem Süden}
Aus dem Süden rücken riesige Scharen von Wilden aus den Waldlanden und \Bangiri gegen die Stadt. Sie sind wütend. Sie wollen Rache für das Waldland und die Verluste im \Enland .

\section{{\Eno}s Entscheidung}
\Eno und \Papato gehen zu den \Bangiri . \Nox wird neuer Anführer der \Schattenlaufer .

\section{Die Rettung \Rhingell{s}}
Der junge König \Palemus legt die Krone nieder, um sich dem \Sturmkind auf seiner Suche anzuschließen. Er hat noch nie die Mauern \Rhin{s} verlassen. Dieses soll seine erste Reise werden.

\section{Der neue Herr der \Eisenmeister}
Das \Sturmkind geht mit \Safir und \Umbra zurück nach \Abaton zum geheimen Herren der \Eisenmeister . Er hat nachgedacht. Seine Zeit ist um.  sagt \Sena , sie soll die Nebelinsel von \Darmon besuchen. Wenn überhaupt, kann so dort erfahren, wo sie herkommt. Er wird eins mit dem Stein und der jetzige Herr der \Eisenmeister nimmt seinen Platz sein.

\section{\Darmon}
Vom \Abaton führt ein geheimer Weg zur Nebelinsel \Darmon am Wasserfall. In der Mitte befindet sich ein Kreis aus verkohlten Bäumen. Als die Reisenden den Kreis betreten, fallen sie im Traum in der Zeit, die lange vorbei ist. Sie sehen, dass der Wald vor langer Zeit auch auf dieser Seite des Flusses bis weit auf die Hochebene wuchs, ehe die \Eisenmeister ihn vollständig in ihrer Gier verbrannten. Die \Eisenmeister kämpften in diesen Tagen gegen wilde Kreaturen aus dem Wald und konnten dem nur Herr werden, in dem sie alles Holz auf dieser Seite des Flusses verbrannten. 

Das \Sturmkind soll weiter nach Süden und Osten gehen, zum Mittelpunkt der Welt, wo die Säule steht, die den Himmel trägt. Wenn sie dort hinaufsteigt, soll ihren Platz in der Welt sehen.

%\end{huge}

\end{document}
