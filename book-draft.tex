\documentclass[12pt,a4paper,onecolumn,twoside,ngerman]{book}

\usepackage[a4paper,left=3.5cm,right=2.5cm,bottom=3cm,top=3cm]{geometry}
\usepackage[ngerman,english]{babel}

% farben
\usepackage{color}

% Korrekte Darstellung der Umlaute
\usepackage[utf8]{inputenc}
\usepackage[T1]{fontenc}

% upper case first letter
\usepackage{lettrine}


% personen
% tern und enland
\newcommand{\Tern}{Tern}
\newcommand{\Beron}{Beron}
\newcommand{\Molitor}{Molitor}
\newcommand{\Ternweg}{{\Tern}weg}
\newcommand{\Sena}{Sena}
\newcommand{\Sturmkind}{Sturmkind}
\newcommand{\Daimon}{Daimon}
\newcommand{\Bangiri}{Bangiiri}
\newcommand{\Pato}{Pato}
\newcommand{\Oggo}{Oggo}
\newcommand{\Papato}{Papato}
\newcommand{\Arwed}{Alfried von \Tern}
\newcommand{\Enland}{Enland}
\newcommand{\Enlaender}{Enländer}
\newcommand{\Schattenlaufer}{Schattenläufer}

% schattenjager
\newcommand{\Eno}{Eno}
\newcommand{\Bomar}{Bomar}
\newcommand{\Dolo}{Dolo}
% Schwiegersohn von Eno
\newcommand{\Nox}{Nox}
% Frau von {\Nox}
\newcommand{\Mena}{Mena}
% Kind 1 von {\Nox} und \Mena
\newcommand{\Umbra}{Umbra}
% Kind 2+3 von {\Nox} und \Mena
\newcommand{\Enna}{Enna}
\newcommand{\Enno}{Enno}
% ?
\newcommand{\Lobo}{Lobo}
% Freund von \Umbra ?
\newcommand{\Tremor}{Tremor}

% lobarn
\newcommand{\Lobarn}{Lobarn}
\newcommand{\Vester}{Vester}
\newcommand{\Naimo}{Naimo}

% nordmark
\newcommand{\Nordmark}{Nordmark}
\newcommand{\Bergmark}{Bergmark}
\newcommand{\Ipes}{Ipes}
\newcommand{\Bron}{Broon}
\newcommand{\Bornhold}{Bornhold}
\newcommand{\Eishold}{Eishold}
\newcommand{\Arn}{Arn}
\newcommand{\Eislaufer}{Eisläufer}
\newcommand{\Eisleute}{Eisleute}
\newcommand{\Eisbestien}{Eisbestien}
\newcommand{\Theodora}{Theodora}

% rhinland
\newcommand{\Rhinland}{Riinland}
\newcommand{\Rhingell}{Riingell}
\newcommand{\Blaufurt}{Blaufurt}
\newcommand{\Mundis}{Mundis}
\newcommand{\Helin}{Heelin}
\newcommand{\Golrin}{Golriin}
\newcommand{\Galadin}{Galaadin}
\newcommand{\Rhinburg}{Riinburg}
\newcommand{\Rhin}{Riin}

% personen
\newcommand{\Habino}{Habino}
\newcommand{\Valem}{Vaalem}
\newcommand{\Palemus}{Paleemus}
\newcommand{\Kalemus}{Kaleemus}
\newcommand{\Isodoriin}{Isodoriin}
\newcommand{\Galeon}{Galeon}
\newcommand{\Demea}{Demea}

% freiberge
\newcommand{\Freiberge}{Freiberge}
\newcommand{\Sudern}{Südern}
\newcommand{\Nachtspringe}{Nachtspringe}
\newcommand{\Schwarzberge}{Schwarzberge}

% bergmark
\newcommand{\Kogida}{Koggida}
\newcommand{\Denner}{Denner}
\newcommand{\Eishohle}{Eishoole}

% dreifluss
\newcommand{\Dreifluss}{Dreifluß}
\newcommand{\Tars}{Taars}
\newcommand{\Toris}{Tooris}
\newcommand{\Planis}{Plaanis}
\newcommand{\Grunarm}{Grünarm}

% grünland
\newcommand{\Grunland}{Grünland}
\newcommand{\Braucheln}{Braucheln}
\newcommand{\Darmis}{Daarmis}
\newcommand{\Darmon}{Daarmon}
\newcommand{\Riesenwald}{Riisenwald}

% eisenmacher
\newcommand{\Eisenland}{Eisenland}
\newcommand{\Eisenmeister}{Eisenmeister}
\newcommand{\Dariom}{Daariom}
\newcommand{\Abaton}{Abbaton}
\newcommand{\Safir}{Saafir}

% oedland
\newcommand{\Staubteufel}{Staubteufel}

\begin{document}

 % Sprache
  \selectlanguage{ngerman}
  
    % arabische Seitezahlen
  \pagenumbering{arabic}
  
  
  % Inhaltsverzeichnis
  \tableofcontents
  
  \clearpage{\pagestyle{empty}\cleardoublepage}
  
  \raggedright 
  
  % Kapitel

% kapitel1.tex

%\begin{huge}
\paragraph{}
\textit{Es war einmal ein Sturm.
Niemand im {\Enland} hatte jemals so ein schlimmes Wetter gesehen.
Am Ende schlagen Blitze in den Grauen Turm mitten in der Stadt {\Tern} ein.
So endet die Zeit der Grauen Herrin, die von hier aus 99 Jahre über das {\Enland} und seine Bewohner herrschte.
}

\chapter{}
\section{Nach dem Sturm}
Nach dem Sturm kommt die Stadt {\Tern} nicht zur Ruhe.\linebreak
Der {\Schattenlaufer} {\Eno} rettete ein Menschenkind aus den Trümmern des Grauen Turms, bevor der {\Bangiri} {\Pato} es erschlagen kann. {\Eno} flieht aus der Stadt, die Wilden verfolgen ihn. Zu einem alten, verlassen Hof, wo ihn {\Bomar} und {\Dolo}, zwei weitere {\Schattenlaufer}, erwarten. 

\section{Im Nest}
Weiter nach Südosten am Seeufer entlang. Sie erreichen am {\Tern} See das Dorf {\Beron}. Viele Häuser sind verlassen. In einem Versteck treffen sie auf weitere {\Schattenlaufer} mit ihren Familien. Der Dorfvorsteher, {\Nox}, ist selbst {\Schattenlaufer}. {\Eno} ist eine hochgestellte Person der {\Schattenlaufer}.\linebreak
Die Zwillinge {\Enna} und {\Enno}, {\Nox} Kinder, wollen unbedingt {\Schattenlaufer} werden. Sie sind enttäuscht, dass man sie nicht lässt, weil sie zu jung sind. Das Mädchen erholt sich einige Tage. Anfangs spricht sie kein Wort. Sie versteht nicht, was die {\Schattenlaufer} sagen.\linebreak
{\Nox} gibt dem Mädchen den Namen "{\Sena}", weil es am 7.Tag der 7.Woche des Jahres gerettet wurde. "{\Sena}" ist die weibliche Form der riinländischen Zahl 7. {\Nox} zeigt ihr die Zahlen von 1 bis 7. Es sind ihre ersten Worte im {\Enland}.\linebreak
Im Dorf {\Beron} lebt der kleine {\Molitor} bei seinem Großvater. Dauernd stellt er Fragen. Sein Großvater meint daher scherzhaft, er weißt nichts. {\Molitor} spielt mit {\Enna} und {\Enno} und mit {\Sena}. 

\section{{\Pato}{s} Rache}
{\Bangiri} durchstreifen das {\Enland}. Der Anführer, {\Pato}, hat erfahren, dass {\Schattenlaufer} in den Ruinen von {\Tern} waren. Sein Großkrieger {\Oggo} hat sie gewittert. Darum glauben die {\Bangiri}, dass die {\Schattenlaufer} und das Mädchen etwas mit der Zerstörung des Grauen Turms zu tun haben. {\Schattenlaufer} und  {\Bangiri} bekriegen sich seit Jahren. Viele {\Enlaender} wurden auf Befehl der Grauen Herrin, die vom Grauen Turm über das {\Enland} herrschte, durch ihre Helfer, eben die {\Bangiri}, aus ihren alten Wohnungen im {\Enland} vertrieben.\linebreak 
Die wilden {\Bangiri} kommen in das Dorf und treiben alle Leute zusammen. Auch den Großvater von {\Molitor} und {\Nox}. Die übrigen {\Schattenlaufer} vertreiben die Wilden, als sie die Einwohner fortführen wollen.

\section{Der Aufbruch}
Die {\Bangiri} kehren mit Verstärkung zurück.\linebreak
Die {\Schattenlaufer}  müssen den Unterschlupf verlassen. Die Familien aus dem Dorf sollen in die {\Nordmark}  gehen. {\Molitor} soll mit, auch wenn sein Großvater bleibt. Der Großvater sagt, dass er nicht der richtige Großvater ist. Die Eltern von {\Molitor} sind verschollen. Vielleicht gibt es im übrigen {\Rhinland}  noch Verwandte.\linebreak
Eine kleine Gruppe mit {\Nox}, {\Eno}, {\Dolo}, {\Tremor}, {\Enna} und {\Enno} soll {\Sena} nach {\Lobarn}  bringen. Dort sollen sie den Vorsteher {\Vester} um Rat und Hilfe bitten.

\section{Der Weg nach Norden}
Bei der Kreuzung mit dem alten {\Ternweg} trennen sich die Gruppen. Sie bemerken {\Bangiri} aus Richtung {\Tern}, die nach Osten Richtung {\Lobarn} ziehen.\linebreak
Die {\Schattenlaufer}, die das Mädchen nach {\Lobarn} bringen, werden kurze Zeit später an einer alten Mauer eingekreist. Sie sitzen in der Falle. Da erhebt sich um {\Sena} ein Wirbelwind. Als die {\Bangiri} versuchen, die Windhose zu durchschreiten, werden sie fortgerissen. Die {\Schattenlaufer} verstehen nicht, was geschehen ist. {\Sena} kann es nicht erklären.\linebreak 
{\Molitor} erscheint. Er ist von den Anderen fortgelaufen und will mit ins {\Rhinland}. Sie setzen ihren Weg fort.

\section{Neuigkeiten}
Die Stadt {\Lobarn} liegt hinter einer Brücke über dem Blauen {\Rhin} geschützt von dicken Steinmauern auf einem Hügel. Ein hoher Wachturm blickt bis an die Grenze zum {\Enland}.\linebreak
{\Eno} erzählt, was im {\Enland} vor sich geht und bittet um Beistand. Dadurch entstehen Unruhen. Die Leute in {\Lobarn} zweifeln, dass die Zeit der Grauen Herrin wirklich vorbei ist. Sie fürchten die {\Bangiri}. Vor {\Sena} fürchten sie sich noch mehr. Wenn sie tatsächlich die Grauen Herrin besiegt hat.\linebreak
{\Vester} lässt sie in ein Dorf außerhalb bringen, wo sie bei {\Naimo}, der Vorsteherin, warten.\linebreak
{\Naimo} erzählt die traurige Geschichte der Grauen Herrin und wie das {\Enland} verloren wurde.

\section{Die {\Bangiri}  kommen}
{\Bangiri} ziehen nach {\Lobarn} . Ihr Anführer {\Oggo} sucht {\Enlaender} und {\Schattenlaufer} und das Wesen aus den Trümmern des Grauen Turms. 

\section{Das Geschäft mit \Lobarn}
Wirkliche Hilfe gegen die Wilden kann ihnen {\Lobarn} nicht geben. Das Kind wollen sie nicht haben.\linebreak
{\Vester} bietet ihnen drei Boten, die das Kind nach {\Rhingell} begleiten sollten, um beim Königshof Hilfe zu erbitten. Allerdings versteht sich das Haus von {\Rhingell} nicht mit den {\Schattenlaufer}. Der König hält sie für Räuber und Gesetzlose. Dafür sollen sich die {\Schattenlaufer} um einen kleinen {\Bangiri} kümmern, der vor einigen Tagen südlich der Stadt eingefangen wurde.\linebreak
Wenn die {\Schattenlaufer} zusammen mit dem kleinen {\Bangiri} und dem {\Sturmkind} {\Sena} aus der Stadt verschwinden, dann hoffen sie, keine Probleme mit den verbliebenen {\Bangiri} , {\Rhingell}s neuem König oder der womöglich doch noch wiederkehrenden Grauen Herrin zu bekommen.\linebreak
{\Eno} hasst alle {\Bangiri}, egal wie alt sie sind. Aber er schlägt ein, weil es {\Naimo} und {\Nox} raten. Sie fliehen in der Nacht weiter nach Osten, nach der Stadt {\Mundis} am Rand der {\Nordmark}.

% kapitel2.tex
% \chapter{Kapitel 2: Nach \Rhingell}

\section{Trennung auf Zeit}
Vor {\Mundis  treffen sie in der Stadt {\Blaufurt}  auf {\Bomar}, {\Mena} und {\Umbra}, die die Familien sicher in die {\Nordmark} gebracht haben. Sie ziehen zusammen weiter. In {\Mundis} treffen sie Gesandte der {\Nordmark} mit {\Arn} von {\Ipes}. Die {\Nordmark} wird vom Herrn von {\Bornhold} im Auftrag der Herren von {\Rhingell} regiert. Dort ist man in Sorge, weil sich im Nordwesten bei {\Eishold} düstere Wesen sammeln. Es sieht so aus, als bereiten die {\Eisleute} einen neuen Kriegszug nach {\Rhingell} vor.\linebreak
Der Herr von {\Bornhold} lag im Streit mit der Herrin des Grauen Turms, weil sie beim Raubzug der {\Eisleute}  nicht in den Kampf eintrat, als die {\Nordmark}, die {\Bergmark} und der Westen {\Rhingell}s verwüstet wurden. Erst, als sich die {\Eisleute} ins {\Enland} begaben.\linebreak
Die Leute der {\Nordmark} verstehen sich gut mit den {\Schattenlaufer}n. Viele Leute aus dem {\Enland} sind in die {\Nordmark} gegangen, um der Grauen Herrin zu entfliehen.\linebreak
{\Eno} verlässt die Gruppe und geht zurück nach {\Lobarn}. {\Arn} sendet einen Späher mit {\Eno} und schickt  Boten zum Herrn von {\Bornhold}. {\Nox} soll diesen begleiten und mit {\Enno} bei den {\Schattenlaufer}    Familien am Rand der {\Nordmark} nach dem Rechten sehen. Die 3 Boten aus {\Lobarn}, {\Dolo}, {\Tremor}, {\Umbra}, {\Sena}, {\Molitor} und {\Enna} ziehen mit {\Arn} und seinem Gefolge weiter nach {\Rhingell} . 

\section{Der Weg am Fluss}
Die Reise von {\Mundis}  nach {\Rhin} führt nah an der alten Stadt {\Golrin} vorbei.\linebreak
Dort hat sich nach dem Krieg der Ritter {\Galadin} niedergelassen, der sich nicht mehr dem Willen des Königs beugt. Er stammt aus der Stadt und ist verbittert über ihre Zerstörung.\linebreak
Die Leute von {\Galadin}  ergreifen sie und bringen sie nach {\Golrin}. {\Galadin} versteht die {\Schattenlaufer}. Beide haben Probleme mit den Herren von {\Rhingell}. {\Galadin} und {\Arn} erzählen vom Krieg mit den Eisleuten.\linebreak 
Die Gesandtschaft folgt dem Fluss {\Rhin}. Sie sehen die Stadt {\Helin} am anderen Ufer, die auf einem Hügel erbaut von hohen Mauern geschützt wird. Im Krieg hatten die Leute das Glück, dass die {\Eislaufer} den Fluss nicht überqueren konnten. So blieben sie vom Schlimmsten verschont. Doch von den Türmen der Stadt konnte man den Rauch sehen, der tagelang aus der Stadt {\Golrin} aufstieg. 

\section{Die Stadt der zwei Türme}
Der Gesandte und die Boten kommen mit {\Sena}, {\Molitor} und  dem jungen {\Bangiri} nach {\Rhin}, der Hauptstadt \Rhingell{s}. Die  {\Schattenlaufer} bleiben außerhalb der Stadt.\linebreak
Sie werden von {\Habino}, dem Vertreter {\Lobarn}s, empfangen. Dort müssen sie 3 Tage warten, ehe sie zum Hofmarschall gelassen werden. {\Theodora}, eine Begleiterin von {\Arn} , erzählt Geschichten vom alten König und dem Verhältnis zwischen dem {\Enland}, der Grauen Herrin,{\Rhinland} und {\Arwed} von {\Tern}.\linebreak 
Der weise {\Valem}, ein Gelehrter aus der Stadt, der mit {\Habino} bekannt ist, erzählt vom neuen König {\Palemus} dem 13. und seinem Bruder {\Kalemus}, der eigentlich König sein sollte. Der Gesandte {\Arn} geht mit den Boten aus {\Lobarn}, {\Sena} und dem jungen {\Bangiri} zum Hofmarschall {\Isodoriin}.\linebreak
{\Molitor} bleibt derweil bei {\Valem}.

\section{Die Herren von \Rhingell}
Der junge König {\Kalemus} weiß nicht, was er machen soll. Er fürchtet das kleine Wesen. Mit den {\Bangiri} will er nichts zu tun haben. Er schickt alle wieder fort.\linebreak
Hofmarschall {\Isodoriin} vermittelt {\Arn} und {\Habino} ein Treffen mit dem Ältestenrat der Stadt.

\section{Gerüchte in der Stadt}
Der Ältestenrat ist mit dem jungen König unzufrieden. Er hat es versäumt, wie schon sein Vater, die Graue Herrin in ihre Schranken zu weisen und sie daran zu erinnern, für die Leute des {\Enland}es zu sorgen anstelle der {\Bangiri}. Um die Sicherheit der Wege in die {\Bergmark} im Norden und im {\Rhinland} im Westen ist es schlimm bestellt. Berichte über Räuber und die Neuigkeiten über {\Eisleute} in der {\Nordmark} bringen ihn nicht zum Handeln. Auf Berichte, dass sich {\Eisbestien} nahe der Grenzen sammeln, reagiert er nicht.\linebreak
Der Ältestenrat beschließt heimlich einen Boten, den jungen {\Galeon} und seine Frau {\Demea} mit {\Sena}, dem {\Bangiri} und {\Molitor} zur Stadt {\Toris} am {\Dreifluss} zu schicken. Man will den König durch seinen Bruder {\Kalemus} ersetzen. Dieser zog vor einigen Wochen nach Süden, um dort zur Ordnung zu sorgen. Die Gesandten sollen außerdem zu den {\Eisenmeister}{n} ziehen, um diese um Beistand zu bitten. Die Kleine {\Sena} soll daran teilnehmen. Der Herr der {\Eisenmeister}, eine sehr weise Kreatur, kann sicher helfen, ihre Herkunft zu klären.\linebreak
Die Boten aus {\Lobarn} kehren in ihre Stadt zurück. Zusätzlich ziehen wieder die {\Schattenlaufer} mit ihnen, die außerhalb der Stadt gewartet haben. {\Dolo} und {\Tremor} führen die Gruppe. Es ist das erste Mal, dass {\Tremor} eine Gruppe mitführt.

\section{\Toris}
Die Stadt {\Toris} liegt wenige Tage von {\Rhin} entfernt an der Mündung des {\Rhingell} in den {\Dreifluss}. Auf der gegenüberliegenden Seite am {\Dreifluss} ist ein Handelsposten der {\Eisenmeister} .

\section{Der Rat von \Toris}
Die Gesandtschaft berät mit dem Ältestenrat der Stadt {\Toris}. Auch dort ist man beunruhigt. Ein schweres Hochwasser verbietet jedoch eine Fahrt über den Fluss. Es wird beschlossen, am {\Dreifluss} nach {\Braucheln} am Großen See zu reisen, um dort über das Wasser zu den {\Eisenmeister} n zu fahren.

\section{Nach Süden}
Nach vier Tagen erreicht die Gruppe die Stadt {\Planis} vorm {\Grunarm}. Dort ist man sehr verängstigt. Wilde Kreaturen streifen bis an die Brücke heran, die nach Süden über den Fluss {\Grunarm} führt. Seit Tagen wagt sich niemand mehr auf die Felder. Die Streitmacht der Herren von {\Rhingell} zog vor 28 Tagen vorbei nach Süden. Seither kamen keine Nachrichten mehr die Straße zurück.
 
% kapitel3\textbf{•}.tex
% \chapter{Das \Grunland}
\section{Das Fort}
Unter großer Vorsicht zieht die Gruppe in der Dämmerung den Weg weiter nach Süden.\linebreak
Als der Tag anbricht, erreichen sie ein verwüstetes und verlassenes Fort am Fluss. Sie verstecken sich dort tagsüber. In der Dämmerung ziehen sie weiter. Ein  {\Schattenlaufer} wird zurück nach {\Planis} gesandt, um die Nachricht vom zerstörten Fort zu verbreiten.

\section{Gefahr am \Dreifluss}
Hinter dem Fort reicht der Wald über den Weg bis an den {\Dreifluss}. Bevor der Morgen kommt, erreicht die Gruppe eine zerstörte Brücke.\linebreak
Man teilt sich auf: ein Teil versucht, den Übergang herzurichten, ein zweiter will schauen, ob in der Nähe ein Übergang ist. Bei dem Versuch, ein Stück weiter im Wald einen Übergang über die Schlucht zu finden, werden sie von Wilden angegriffen. Anschließend wird die gesamte Gesandtschaft zwischen {\Dreifluss} und {\Riesenwald} zerstreut.  {\Sena} wird gestoßen und fällt in ein Loch.

\section{Im \Riesenwald}
{\Sena} wacht auf. Sie ist allein im Wald. Sie geht weiter und weiter am Rand des Abgrundes, bis die Felsen niedriger werden. Unten zwängt sie sich zwischen Baumstümpfen und Steinen hindurch an einen Teich, wo viele wilde Kreaturen eine riesige, drachenähnliche Gestalt umringt haben.\linebreak
Die wilden Kreaturen bemerken {\Sena} nicht. Sie geht zu dem Wesen, sie legt die Hand auf die Stirn und im Sturm verwandelte sich die Kreatur in den jungen {\Bangiri}. Er gibt sich als {\Papato} zu erkennen, als Sohn des Anführers der {\Bangiri} im {\Enland}. Er wollte von seinen Leuten ausreißen, wurde aber auf dem Weg nach den Schwarzbergen von Spähern aus {\Lobarn} geschnappt. 

\section{Zurück zum Fluß}
{\Tremor} hat sich die ganze Zeit am Rand versteckt. Nachdem die Kreaturen ablassen, nimmt er {\Sena} und \Papato  und bringt sie zurück auf den Weg. Dort treffen sie weitere {\Schattenlaufer} und die Boten aus {\Rhingell}.

\section{Spuren}
An den Fällen des {\Dreifluss} sehen sie im Morgengrauen in der Ferne den Dunst der Nebelinsel und noch weiter dahinter die Rauchsäulen von {\Dariom}, der Stadt der {\Eisenmeister} am gegenüberliegenden Ufer des Sees.\linebreak
Sie ziehen weiter nach Süden und durchqueren gegen Mittag ein zerstörtes Dorf. Die Asche ist kalt.

\section{\Braucheln}
Sie eilen weiter. In der Ferne sehen sie die Stadt {\Braucheln} am See. Vor dem Tor haben sich Kreaturen aus dem Wald versammelt. {\Sena} geht in einem Sturmwirbel mit {\Papato} zu den Kreaturen. Sie lassen ab und ziehen sich in den Wald zurück.

% \chapter{Die {\Eisenmeister}
\section{Ankunft in {\Braucheln}}
In {\Braucheln} treffen sie den Bruder des Herren von {\Rhingell}, {\Kalemus}. Er berichtet über Angriffen von Kreaturen aus den Wäldern. Es wird seit Wochen kein Holz mehr geschlagen und keine Holzkohle gemacht und über den See geschafft. {\Safir}, Abgesandter des Herrn der {\Eisenmeister}, ist in {\Braucheln}, um sich über die ausbleibenden Lieferungen zu erkundigen.

\section{Auf dem Weg nach \Dariom}
Der Gesandte, einige {\Schattenlaufer}, das {\Sturmkind}, {\Molitor} und \Papato folgen dem Abgesandten der {\Eisenmeister} über das Wasser in die Stadt der {\Eisenmeister}.\linebreak
Die Stadt der {\Eisenmeister} erstreckt sich von den Fällen des {\Dreifluss}es, der mit seinem Wasser riesige Räder für die Erzschmieden antreibt, weit am Ufer und den Felsen nach Süden, wo die Handelswege in die Östlichen und Südlichen Lande verlaufen.

\section{Beim Herren der {\Eisenmeister}}
Der Herr der {\Eisenmeister} weist die Hilfegesuche ab. Die Streitereien der Leute aus {\Rhingell}  gehen die {\Eisenmeister} nichts an. Man hat genug Problem mit Wilden im nördlichen Ödland und den {\Staubteufel}n in den verlorenen Städten des Hochlandes.\linebreak 
Er bietet dem {\Sturmkind} an, es dem geheimen Herren des Erzes zu präsentieren, wenn sie helfen, das Rätsel der \Staubteufel zu lösen.

\section{Das trockene Land}
{\Molitor}löst das Rätsel der {\Staubteufel}.

\section{Der Herr des Erzes}
Der geheime Herr des Berges Erzes sitzt tief im Berg {\Abaton}. Bewacht von seiner Garde, zweimal vierzehn Eisenwesen, auf seinem eisernen Thron. Seine Augen sind trüb, aber er ist mit dem Stein verbunden und sieht Dinge, die sind und die sein können, aber nur schwach das Echo der Dinge, die einmal waren.\linebreak
Da {\Sena} und ihre Begleiter ihnen mit den Staubteufeln beistanden, schickt er sie mit einem weißen Schlüssel zurück zum König nach {\Rhingell}. Wenn Ruhe in {\Rhingell} einkehrt, wird auch wieder Holz in die Stadt der Erzgräber kommen. Der Gesandte \Safir der Erzgräber und ein Eisenwesen sollen sie begleiten.\linebreak
Wenn die Dinge geordnet sind, sollen sie wiederkehren. Bis dahin will der Herr des Eisens alles in seiner Macht tun, um Antworten auf ihre Fragen zu finden.

\section{Zurück nach {\Rhingell}}
Auf dem Rückweg auf dem anderen Ufer des Dreiflusses durchqueren sie ein ödes Land. Sie kommen durch einen kleinen Posten der {\Eisenmeister}.\linebreak 
Das Hochwasser ist so weit zurückgegangen, dass der {\Dreifluss} bei {\Toris} wieder überquert werden kann.

%\chapter{Pi}

%\chapter{Ex}
\section{Empfangen und wieder nicht}
Hofmarschall {\Isodoriin} hört sie an. Er will sie dieses Mal nur zum König lassen, wenn sie sich würdig erweisen. Dazu sollen sie nach Norden, an der Grenze zwischen {\Nordmark} und {\Bergmark},um einer Räuberbande das Handwerk zu legen. Als Beweis sollen sie das Herz des Hexenmeisters vorlegen.\linebreak
{\Arn} ist verzweifelt. Vor der Stadt trifft er auf {\Tremor}, {\Umbra} und {\Enna}. Sie beschließen, nach Norden zu ziehen.

\section{Die Kinder von Kogida}
Die Räuber rauben Kinder aus der Stadt {\Kogida} und umliegenden Dörfern für den Hexenmeister. Mit geheimnisvoller Hilfe können sie die Räuber vertreiben und bringen den schwarzen Hexenmeister {\Denner} in ihre Gewalt. Er bleibt jedoch am Leben. Sie kehren zurück nach {\Rhin}.

%\chapter{Sen}
\section{Bruderkrieg}
Sie kehren gerade rechtzeitig zurück. {\Kalemus}, der ältere Bruder, hat die Geduld verloren. Er ist zusammen mit vielen Kämpfern aus {\Toris} nach {\Rhin} gezogen. In {\Rhingell} hat sich der junge König mit seiner Leibgarde, den 49 weißen Rittern, in der Burg verschanzt. Die Stadt, das ganze Land hat sich gegen ihn gestellt.\linebreak
Eine große Verschwörung wird von {\Nox} enthüllt.

\section{Der Zug aus dem Süden}
Aus dem Süden rücken riesige Scharen von Wilden aus den Waldlanden und {\Bangiri} gegen die Stadt. Sie sind wütend. Sie wollen Rache für das Waldland und die Verluste im {\Enland}.

\section{{\Eno}s Entscheidung}
{\Eno} und {\Papato} gehen zu den {\Bangiri}.\linebreak
{\Nox} wird neuer Anführer der {\Schattenlaufer} und ein neuer Vorsitz wird gewählt.

\section{Die Rettung \Rhingell{s}}
Der junge König {\Palemus} legt die Krone nieder, um sich dem {\Sena} anzuschließen. Er hat noch nie die Mauern \Rhin{s} verlassen. Dieses soll seine erste Reise werden.

\section{Der neue Herr der {\Eisenmeister}}
Das {\Sturmkind} geht mit {\Safir} und {\Umbra} zurück nach {\Abaton} zum geheimen Herren der {\Eisenmeister}.\linebreak
Er hat nachgedacht. Seine Zeit ist um. {\Sena} soll die Nebelinsel von \Darmon besuchen. Wenn überhaupt, kann so dort erfahren, wo sie herkommt. Er wird eins mit dem Stein und der jetzige Herr der {\Eisenmeister} nimmt seinen Platz sein.

\section{\Darmon}
Vom {\Abaton} führt ein geheimer Weg zur Nebelinsel {\Darmon} am Wasserfall. In der Mitte befindet sich ein Kreis aus verkohlten Bäumen.\linebreak
Als die Reisenden den Kreis betreten, fallen sie im Traum in der Zeit, die lange vorbei ist. Sie sehen, dass der Wald vor langer Zeit auch auf dieser Seite des Flusses bis weit auf die Hochebene wuchs, ehe die {\Eisenmeister} ihn vollständig in ihrer Gier verbrannten. Die {\Eisenmeister} kämpften in diesen Tagen gegen wilde Kreaturen aus dem Wald und konnten dem nur Herr werden, in dem sie alles Holz auf dieser Seite des Flusses verbrannten.\linebreak
{\Sena} war schon vor dem Sturm ins {\Enland} gekommen. Die Geister weisen {\Sena} einen Weg nach Südosten, zum Mittelpunkt der Welt. Dort steht die Säule, die den Himmel trägt. Wenn sie dort hinaufsteigt, soll einen Platz sehen, der für sie bestimmt ist.

%\end{huge}

\end{document}
