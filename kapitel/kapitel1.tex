% kapitel1.tex
\chapter{Kapitel 1}
\section{Der Sturm}

Es war einmal ein Sturm.
Niemand im \Enland hatte jemals so ein furchtbares Unwetter erlebt. Der Wind tobte von den Bergen hinab über das kahle Land. Tiefschwarze Wolken bildeten einen riesigen Wirbel um den Grauen Turm in den Mauern der alten Stadt. 

Selbst am Tag war es finster wie in der Nacht. Als der Sturm lange genug gewütet hatte, krochen von seinem Fuß leuchtende Funken hinauf. Ganz langsam, immer höher und höher. Kurz unter der Spitze trafen die Funken auf leuchtende Blitze. Um mit einem Schlag zerplatzte der graue Turm. Dunkle Brocken flogen in alle Himmelsrichtungen. Ein gewaltiger Donner rollte über die Erde. Die düsteren Wolken verschwanden und eine klare Nacht senkte sich über das Land. Die Sterne wanderten an ihre Plätze zurück. Nur ein grünes Feuer schimmerte am Himmel über der Stadt.

Die alte Stadt, die einst die stolze Stadt \Tern am See war, lag still am Sumpf. Aber bald war es mit der Ruhe vorbei. Durch die Ruinen klangen Schreie und das Knistern großer Feuer. War da nicht eben am Rand für einen kurzen Moment eine dunkle Gestalt vorbeigehuscht? Schon war sie wieder zwischen den Schatten verschwunden. Nur ein grünes Auge funkelte noch im Dunkeln und sah, wie große, grunzende Wesen, die in Felle und Fetzen gekleidet waren, durch die Straßen liefen und ab und zu von erschrockenen Gesichtern aus halbverfallen Häusern beobachtet wurden.

\section{Der Schattenjäger}
Lautlos schlich die dunkle Gestalt in den Schatten der Mauern zur Mitte der alten Stadt. Der Trümmer des Grauen Turms bedeckten den Hügel, auf dem er sich einst erhob. Inmitten dunkler Steine und schwarzer Brocken leuchtete ein riesiges, silbernes Ei. 

Die Gestalt trat aus dem Schatten. Sie trug einen dunklen Mantel. Die Kapuze war tief ins Gesicht gezogen und ein Tuch vor die Nase gebunden, so dass nur das leuchtende, grüne Auge zu sehen war. Direkt gegenüber, von der anderen Seite des Hügels, sprang plötzlich ein großes, pelziges Wesen mit langen, kräftigen Armen zu dem Ei. Es brüllte wild und hämmerte gegen das Ding, bis die glänzende Schale zersprang. Dann war es ruhig, so ruhig, dass man nur das Schnaufen hörte. Das Ungeheuer langte mit seinen großen Händen in die Schale und zog etwas mit hellen Haaren und schmutzigen Kleidern heraus. Als es auf dem Boden lag, machte das grimmige Wesen wieder eine Pause. Dann hob es die riesigen Fäuste. 

Die Schattengestalt griff blitzschnell nach einem Stein und warf ihn neben den grimmigen Wesen zwischen die dampfenden Trümmer. Es erstarrte und riss das Maul auf, in dem weiße, scharfe Zähne blitzten. Dann dreht es sich in die Richtung, in die der Stein gefallen war und ging langsam zu der Stelle hinüber. 

Der Mantelträger sprang zu dem Ei und riss das Kind an sich. Das Pelzwesen drehte sich um und sah nur noch den wehenden, dunklen Mantel. Die Überraschung verschlug ihm die Sprache. Als das Pelzwesen begriffen hatte, dass seine Beute fort war, krümmte es seine riesige Gestalt vor Wut und brüllte: „Diebe! Schattenjäger!“. Wenige Augenblicke später schallten von überall her mehr und mehr „Schattenjäger! Schattenjäger!“ - Rufe durch die Dunkelheit.

Der Schattenjäger, den genau das war die Gestalt mit dem dunklen Mantel und dem grünen Auge, rollte unter einen Busch. Dort breitete er seinen Mantel über dem Wesen aus. So etwas hatte er noch nie gesehen. Unter den verschmutzen Kleidern hatte es nur ein ganz dünnes, viel zu kurzes Fell und eine helle Haut. Dafür reichte das Fell vom Kopf in dicken Strähnen bis zur Hüfte hinab. Es lag da wie tot. Helles Blut war an der Nase. Der Jäger tastete es vorsichtig mit seinen Tatzen ab. Es war noch genügend Leben in dem kleinen Körper. Er schlug seinen schwarzen Mantel darüber und drehte sich um.

Im Schatten einer Mauer legte er eine Pause ein. Wenn man genau hinsah, konnte man noch das grüne Auge im Dunkeln erkennen. Zu dem großen, pelzigen Wesen vor den Trümmern des Grauen Turms kamen weitere, die alle wild durcheinander „Schagujagi“ schrien. Die Straßen glichen einem Ameisenhaufen, den man umgestoßen hatte. Lautlos und geduldig entfernte sich der dunkle Jäger mit seiner Beute immer weiter von den Trümmern des Grauen Turms, bis er nach einer Stunde die letzten Mauern und Ruinen hinter sich gelassen hatte.

Ein Stück weit hinter der alten Stadt stieg der Weg einen Hügel hinauf. Hinter dem Hügel sah man weit in der Ferne ringsherum die Berge, die die Grenze des alten \Enlande{s} anzeigten. Der Schattenjäger verließ den Weg nach links und schlich über das niedrige Gras. Etwa nach einer halbe Stunde erreichte er einen schmalen Bach. Die Sterne des Himmels spiegelten sich im Wasser. Er legte seine Hosen ab und watete vorsichtig mit dem Wesen auf den Armen dicht am anderen Ufer den Bach entlang. Nach einigen Minuten stieg er aus dem Wasser. Er steuerte eine alte Scheune an und setzte sich nieder. Er keuchte und lehnte sich zurück. Das Schleichen mit der Last war anstrengend. Der Jäger legte eine Pause ein und deckte sich und das Wesen mit dem dunklen Umhang zu.

\section{Das Kind}
Als die Sonne den Himmel hinaufstieg, wachte er auf. Bei Tag sah man, dass er noch nicht weit gekommen war. Der Rauch vom Grauen Turm bildete dicke, weiße Wolken. Bis zu den Ruinen war nicht weit. Man brauchte von der Raststelle nur ein oder zwei Stunden. Ringsum erhoben sich in der Ferne hohe Bergketten, die das ganze Land umschlossen. Bis zu den Bergen zur Rechten würde er von hier über einen ganzen Tag laufen. Aber da er nicht mehr allein war, würde es zurück länger dauern.

Der Jäger schlug die Kapuze zurück, so dass man sein Fell, das grau und schmutzig wie Asche war, und seine spitzen Ohren sehen konnte. Die Sonne schien warm. Aus seinem Umhang zog er einen kleinen Beutel und trank. Dann schlug er den Umhang soweit zurück, dass er das kleine Wesen genau sehen konnte. Es war ein Mädchen. Es hatte die ganze Zeit auf seinem Schoß gelegen. An Nase und Mund klebte getrocknetes Blut. Er wusch das Blut mit Wasser vorsichtig ab. Der Jäger breitete seinen Umhang aus wie ein Laken. Dann legte er vorsichtig den kleinen Körper darauf und ging zum Bach, um neues Wasser zu holen.

Beim Nachfüllen hörte er ein Geräusch hinter seinem Rücken. Das Wesen hatte sich aufgerichtet und starrte ihn mit großen Augen an. Der Jäger legte seinen Finger auf die Lippen und ging langsam weiter. Er streckte seinen Arm aus und reichte den vollen Beutel weiter. Das Wesen sah aus wie ein Kind. Sein Gesicht war weich, aber nicht rund. Es sagte kein Wort, nahm den Beutel, schaute den Jäger an und trank dann einen großen Schluck. 

„Wie heißt du?“ fragte der Jäger das Mädchen. „Oh, entschuldige“, sagte er, als sie nicht antwortete. „Man nennt mich \Eno.“ Sie versuchte, etwas zu sagen, aber es kam ihr kein Laut über die Lippen. Sie wollte aufstehen, aber fiel dabei um. „Warte!“ rief \Eno und fing sie auf. Sie hatte die Augen wieder geschlossen.

Nach dieser kurzen Pause brachte \Eno das Mädchen weiter nach Norden. Mit der Dämmerung, als der Tag zu Ende ging, hatten sie ein großes Stück des Weges geschafft. Die Wolken stiegen immer noch aus den Trümmern des Grauen Turms in der alten Stadt am Sumpf, als sie am Ende eines weiteren Tages eine alte Scheune am Fuß der Berge im Nordwesten erreicht hatten. Hier traf \Eno zwei weitere Schattenjäger. \Bomar, der größere von beiden, war einen Kopf größer als \Eno und hüllte sein braunes Fell in einen dunkelgrünen Umhang. \Do hatte rötliches Fell und war nur halb so groß wie \Bomar. 
„Was ist das?“ fragte der neugierige \Do. 
„Ich habe sie am Grauen Turm gefunden“, sagte \Eno. „Ein großer Bangiri war bei ihr. Ich glaube, er wollte ihr etwas Schlimmes antun. Darum habe ich sie mitgenommen. Wir können nur kurz rasten und uns stärken, dann müssen wir schnell weiter zur Stadt. Die Kleine ist schwer verletzt.“

\section{Der Ternweg}
Nachdem sie den Bergkamm erreicht hatten, folgten sie einem alten Pfad, der in einem weiten Bogen nach Süden führte. Sie kamen nun schneller vorwärts. Die Sterne funkelten hell durch die finstere Nacht. Mal trug \Eno das Mädchen ein Stück, dann \Bomar. \Do ließ die beiden alleine und ging neben dem Pfad. Schließlich war er so weit entfernt, dass seine gelben Augen nicht mehr im Dunkel leuchteten.

Gegen morgen, als es wieder Tag wurde, sahen die beiden Jäger im Osten eine weite, grüne Ebene. Weit entfernt auf einem Hügel lag eine kleine Stadt. Zur linken Seite führte der Pfad hinunter. Direkt vor ihnen öffneten sich die Berge. Eine breite, alte Straße schlängelte sich zwischen den Hügeln aus dem Enland nach Osten. Sie verließen den Pfad und die Berge und eilten neben der Straße weiter, als \Do plötzlich zu ihnen zurückkehrte. „Aus dem \Enland kommt eine Gruppe. Sie ziehen auf dem alten \Tern{-Weg} nach \Lobarn.“ meldete er. „Wie viele?“ fragte \Eno. „Ungefähr zehn. Sind noch eine Stunde entfernt.“ „Du hast scharfe Augen, \Do. Wir müssen zusehen, dass wir in die Stadt kommen“, antwortete \Eno. „Gehen wir auf der Straße, dort sind wir schneller.“

Ab und zu kamen die drei Jäger an verlassenen Häusern vorbei. Die Dächer waren durchlöchert und die Türen herausgerissen. Hier wohnte schon lange niemand mehr. Der Weg führte nun schnurgerade zu dem Hügel, auf dem die Stadt langsam näher kam. Es dauerte nicht lange, bis hinter ihnen auf dem Weg kleine schwarze Punkte zu sehen waren. „Sie holen auf“, sagte \Do, „Wir sind zu langsam!“ Und tatsächlich rückten die Punkte näher und wurden größer. Bald hatten auch ihrer Verfolger erkannt, dass ihr Ziel direkt vor ihren Augen lag und verdoppelten noch einmal ihr Tempo. „Wir schaffen es nicht!“ rief \Bomar. „Da vorne liegt ein alter Hof, da können wir uns besser verteidigen!“ Die Jäger, die nun die Gejagten waren, hasteten zu einem Steinmauer. Als sie diese erreichten, sprangen am linken und rechten Rand zwei große, zottelige Wesen hervor. In einem weiten Halbkreis versammelten sich mehr und mehr davon. Die drei Jäger standen mit dem Rücken zur Mauer. „Schagujagi gib her!“, brüllte eines der Monster in der Mitte der Angreifer. \Eno wich zurück. Der Bangiri hob seine Tatze, und holte zu einem kräftigen Schlag gegen das Mädchen aus. Aber in dem Augenblick, als er es berührte, war es, als schlug ihn eine unsichtbare Keule so heftig, dass er einen riesigen Satz rückwärts machte. Da näherten sich zwei weitere Bangiri vorsichtig dem Mädchen. Aber als sie weiter herankamen, brüllten sie los und rannten erschrocken davon. Nun machten auch die übrigen Bangiri kehrt. Sie waren wieder allein.

\section{Die Versammlung zu \Lobarn}
Hinter einer kleinen Brücke stieg der Weg an. Die Stadt \Lobarn lag auf einem  Hügel. Es war es nicht mehr weit. Die Schattenjäger sagten kein Wort, bis sie vor dem Tor von \Lobarn hielten. \Eno pochte an.
Ein Lanzenträger schaute von oben herunter. 
„Wer seid ihr? Was wollt ihr?“
„Ich bin \Eno vom Bund der Schajagi. Ich habe Neuigkeiten aus dem \Enland.“ 
Man ließ sie ein und brachte \Eno, \Do, \Bomar und das Mädchen zum Stadthaus, in dem sich viele Leute eingefunden hatten. Sie hatten bemerkt, dass das Wetter umgeschlagen war, kannten aber nicht den Grund. Als sie eintraten, kam ihnen eine alte Gestalt entgegen. 
„Mein Name ist \Vester, ich führe das Wort in dieser Zusammenkunft. Sprecht! Was ist im \Enland geschehen?“
Alle schwiegen und schauten auf \Eno, den Schattenjäger.
„Der Graue Turm ist zerstört. Seine Herrin ist fort.“
Ein Raunen und Flüstern setzte ein.
„Seid ihr Euch sicher, \Eno vom Bund der Schajagi?“
„Ich habe gesehen, wie ein gewaltiger Donnerschlag den Turm in tausend Stücke riss. Aber es gibt auch schlechte Nachrichten. Es sind noch sehr viele Bangiri in und um \Tern. Einige haben mich und meine Begleiter zurück auf dem alten \Tern{weg} verfolgt.“
„Was wollten diese Biester?“
\Eno trat in die Mitte und legte das kleine Mädchen auf den Boden.
„Was ist das? Wo ist dieses Kind her? Was hat es für eine seltsame Gestalt?“
„Ich habe es in den Ruinen des Grauen Turms einem Bangiri genommen“, antwortete \Eno.
„Was wollte der Bangiri von dem Kind?“
„Er wollte ihm Leid zufügen.“
„Gibt es eine Verbindung zwischen dem Fall des Grauen Turms und diesem Geschöpf hier?“, fragte der alte \Vester weiter.
„Ich weiß es nicht“, antwortete \Eno.
„Wenn die Bangiri es wollen, dann werden sie kommen, um es zu holen“, rief jemand. „Und dann sind wir alle in Gefahr!“
„Ruhe!“, rief \Vester. „Still! Ich führe das Wort!“
Aber mit der Ruhe war es vorbei. Die Leute redeten nun wild durcheinander: „Was sollen wir tun, wenn morgen tausend Bangiri vor unseren Toren stehen? Werft die Schattenjäger aus der Stadt! Wir wollen nicht in ihren Streit mit den Bangiri hineingezogen werden! Falls die Graue Herrin gar nicht fort ist, wird ihre Rache fürchterlich sein. Und seht das Kind! Es hat ja nicht mal vernünftiges Fell! Von welcher Art mag es sein? Hat jemand schon mal so etwas gesehen?“
Nein, niemand hatte so ein Kind gesehen. Die Leute drängten sich um die Schattenjäger und das Kind. Die Kleine versteckte sich ängstlich zwischen \Eno und \Bomar. Ein Gemurmel und Geschiebe entstand. Das Geschrei wurde immer lauter. Der alte \Vester wies die Wächter an, den Saal räumen. Die Schattenjäger und das Kind wurden zu einer Seitentür gebracht und verschwanden.

\section{Weiter}
Durch viele schmale Gassen wurden die Schajagi von zwei Wächtern zum Südtor gebracht. Sie verließen die Stadt auf einem alten Weg zwischen Feldern und kleinen Büschen. Nach einer Weile erreichten sie eine kleine Siedlung, die von einem hohen, festen Holzzaum umgeben war. Ein Wächter klopfte mit seinem Schild gegen das breite, hölzerne Tor. Sie wurden eingelassen. Die Wächter führten die Schattenjäger und das Kind zwischen den niedrigen Hütten zu einem großem Holzhaus, das drei Stockwerke hinauf alle anderen Hütten überragte. Einer der Wächter ging hinein und kehrte nach einer ganzen Weile zurück.
„Tretet ein“, sagte er zu den Jägern. „Sie empfängt euch gleich. Ihr bleibt hier, während in \Lobarn über die Sache beraten wird. Hier seid ihr sicher.“ Als er sich zum Gehen wandte, beugte er sich zu \Eno und flüsterte: „Und bleibt hier! \Vester zählt auf euch. Ihr wisst, dass es in \Lobarn schlecht ankommt, wenn ihr heimlich verschwindet.“
\Eno antwortete nicht. Die beiden Wächter verließen ihre Schützlinge.
Als die Schattenjäger das Haus betraten, standen sie in einer hohen Halle aus Holz, ähnlich dem Ratsaal in \Lobarn.
„Kommt herüber! Mein Name ist \Naimo “, sagte eine alte Rotfüchsin in einfachen, braunen Kleidern. „Man hat mir berichtet, was sich der alte \Vester dabei gedacht hat, euch hier zu verstecken. Aber die Kunde aus \Tern möchte ich noch einmal genauer von denen hören, die sie aus dem \Enland hergebracht haben!“
Sie ließen sich in einem kleinen Nebenraum an einem runden, dunklen Tisch nieder und erzählten abwechselnd von \Tern, von dem Donnerschlag und dem Rückweg bis nach \Lobarn. Wie sie ihre Verfolger genau los wurden, verschwiegen sie.
Die Alte hörte die ganze Zeit aufmerksam zu.
„Wie ihr sicher wisst“, sagte sie dann. „ist seit dem Winter ein neuer König in der Ringburg. Als der alte König starb, wurde sein jüngerer Sohn vom Obersten Ratgeber gekrönt. Die Grauen Herrin stand ihm sehr nahe, wie schon seinem Vater. Wer weiß, was der junge König tut, wenn sie fort ist? Oder was die Leute in \Rhin tun werden? Es heißt, dass sie lieber seinem älteren Bruder auf dem Thron wollen. Und auf welche Seite werden sich erst die Ritter der Ringburg stellen? Was ist mit den Bangiri? Alle diese Dinge muss der Rat bedenken.“  

Hatte es die grausame Herrin vertrieben?
Die Leute in \Lobarn fürchteten sich vor der Kleinen. Ein Rat wurde abgehalten und es wurde beschlossen, die Kleine nach der Hauptstadt von \Rhingell zu bringen, sobald sie sich erholt hatte.

\section{Papatos Rache}

Als der Graue Turm in Trümmer ging, verschwand nicht nur die Herrin des Grauen Turms. Schon vor Jahren hatte sie angefangen, aus einem Land weit im Süden, hinter den Stinkenden Sümpfen, neue Diener zu holen. Sie nannten sich Bangiri und waren wilde Gestalten, die sich in Lumpen hüllten und kaum so etwas wie eine Sprache besaßen. 
Sie waren ihrer neuen Herrin treu ergeben. Diese vertrieb die Bewohner aus \Tern und gab die Häuser den Bangiri. Durch seine neue Bewohner verfiel die Stadt \Tern mit den Jahren. Am Ende glich sie mehr einem Ameisenhaufen als der stolzen Stadt früherer Zeiten.

Nicht alle Bangiri waren fort. Die Übriggebliebenen wurden von wildem Zorn gepackt und zogen ebenfalls nach Osten. Nach \Lobarn. Der neue Anführer \Papato hatte erfahren, dass ein Schattenläufer in den Ruinen von \Tern gewesen war. Darum meinten sie, die Schattenläufer, von denen viele aus dem \Enland stammten, hätten den Grauen Turm zerstört und die Bangiri getötet, weil sie einst von dort vertrieben wurden. \Papato schwor allen Schattenläufern Rache.

\section{Die Bangiri kommen}
Eine Rotte von \Bangiri zog nach \Lobarn.
Während die Bangiri verlangten, dass die Leute dort alle Schattenläufer herausgaben und das Wesen, was V am Grauen Turm gefunden hatte, fliehen \Eno und die übrigen Schattenläufern mit der Kleinen aus der Stadt. Sie laufen weiter nach Osten, nach der Stadt \Mundis.

\section{Nach \Mundis}
Vor der Stadt \Mundis treffen sie auf Gesandte der \Nordmark, in denen der Herrn von \Bornhold im Auftrag der Herren von \Rhingell regiert.
Der Herr von \Bornhold lag bis zuletzt im Streit mit der Herrin des Grauen Turms, weil sie beim Raubzug der Eiswölfe erst in den Kampf  eintrat, da die \Nordmark, die \Bergmark und der Westen \Rhingell{s} so verwüstet waren, dass sich die Eiswölfe ins \Enland begaben, wo sie jedoch von der Macht der Grauen Herrin und ihren Dienern, den Bangiri, völlig ausgelöscht wurden. Die Gesandten treiben die Rotte der Bangiri zurück an die Grenze zum \Enland. Der Oberste Gesandte, \Arn, geleitet die Schattenläufer und die Kleine weiter nach Osten.
