% kapitel4.tex
\chapter{Die Eisenmacher}
\section{Ankunft in Braucheln}
In Braucheln trafen sie den Bruder des Herren von Rhingell. Er berichtet über Angriffen von Kreaturen aus den Wäldern, Die Stützpunkte der Holzfäller am Waldrand seien alle verlassen. Es wird seit Wochen kein Holz mehr geschlagen und keine Holzkohle gemacht und über den See geschafft. Es weilt bereits Saphir, ein Gesandter des Herrn der Erzgräber, in Braucheln, der sich über die ausbleibenden Lieferungen erkundigen sollte.

\section{Auf dem Weg nach Darium}
Der Gesandte, einige Schattenläufer, das Sturmkind und das Kleine aus dem Riesenwald folgen dem Abgesandten der Erzgräber über das Wasser in die Stadt der Erzgräber.

Die Stadt der Erzgräber ist riesig und erstreckt sich von den Fällen des Dreiflusses, der mit seinem Wasser riesige Räder für die Erzschmieden antreibt, weit am Ufer und den Felsen nach Süden, wo die Handelswege in die Östlichen und Südlichen Lande verlaufen.

\section{Beim Herren der Eisenmacher}
Der Herr der Erzgräber weist die Hilfegesuche ab. Die Streitereien der Leute aus Rhingell gehen die Erzgräber nichts an. Man hat bereits genug Problem mit den Einfällen der Vielbeinigen über das nördliche Ödland und im Süden. Aber er bietet dem Sturmkind an, es dem geheimen Herren des Erzes zu präsentieren.

\section{Der Herr des Erzes}
Die hohe Pforte öffnete ihre Flügel. Sie traten in eine Höhle, die so riesig war, dass man weder die Decke, noch links noch rechts ein Ende sehen konnte. Direkt vor den Besuchern begann ein Weg aus funkelnden Edelsteinen, an dem links und rechts silberne Säulen standen. Die Säulen reichten so weit hinauf, dass sie in der Finsternis verschwanden.
Am Ende der Säulenreihen erhob sich ein grauer Thron. Auf beiden Seiten standen riesige Statuten mit glänzenden Hämmern und Hacken anstelle von Händen. Auf dem Thron ruhte eine kleine Gestalt mit silbernen Gewändern und silbernen Haaren. „Kommt näher!“ donnerte es durch die Halle, ohne dass die Gestalt auf dem Thron sich geregt hätte. „Seht den geheimen Herrn des Berges!“ 

„Herr“, begann sie ängstlich. „Ich weiß nicht, wo ich hin soll. Und wo ich herkomme.“
„Kind“, antwortete der geheime Herr und es klang fast, als würde die Stimme lachen, „Was denkst du? Du bist doch keine Gabel, die jemand aus Versehen an den falschen Platz gelegt hat.“

Der geheime Herr des Berges Erzes sitzt tief im Berg bei der Stadt der Erzgräber in einer weiten Höhle. Bewacht von seiner Garde, zweimal zwölf Eisenwesen, sitzt er auf seinem eisernen Thron. Seine Augen sind trüb, aber er ist mit dem Stein verbunden und sieht Dinge, die sind und die sein können, aber nur schwach das Echo der Dinge, die einmal waren. Und dass die Gier nach Holz die Übergriffe im Riesenwald erzeugt. Und er spürt vor allem, dass ein Kampf zwischen den Brüdern um die Herrschaft über Rhingell bevorsteht. Darum müssen das Sturmkind und die Gesandten schnell zurück nach Taris am Dreifluss. Wenn wieder Ordnung in Rhingell einkehrt, wird auch wieder Holz in die Stadt der Erzgräber kommen. Der Gesandte Saphir der Erzgräber und ein Eisenwesen sollen sie begleiten.

\section{Zurück nach Rhingell}
Auf dem Rückweg auf dem anderen Ufer des Dreiflusses durchqueren sie ein ödes Land. Sie kommen durch einen kleinen Posten der Erzgräber. Dort wird klar, dass der Wald vor langer Zeit auch auf dieser Seite des Flusses wuchs, ehe die Erzgräber ihn vollständig in ihrer Gier verbrannten. Dadurch ragt nun auch das östliche Ödland bis an den Fluss heran, wo früher hohe Bäume standen. Die Erzgräber kämpften in diesen Tagen auch gegen Kreaturen aus dem Wald und konnten dem nur Herr werden, in dem sie alles Holz auf dieser Seite des Flusses verbrannten.