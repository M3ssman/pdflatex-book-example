% kapitel2.tex
\chapter{Nach \Rhingell}

\section{Nach \Mundis}
Vor der Stadt \Mundis treffen sie auf Gesandte der \Nordmark, in denen der Herrn von \Bornhold im Auftrag der Herren von \Rhingell regiert.
Der Herr von \Bornhold lag bis zuletzt im Streit mit der Herrin des Grauen Turms, weil sie beim Raubzug der Eiswölfe erst in den Kampf  eintrat, da die \Nordmark, die \Bergmark und der Westen \Rhingell{s} so verwüstet waren, dass sich die Eiswölfe ins \Enland begaben, wo sie jedoch von der Macht der Grauen Herrin und ihren Dienern, den Bangiri, völlig ausgelöscht wurden. 

Die Gesandten treiben die Rotte der Bangiri zurück an die Grenze zum \Enland. Der Oberste Gesandte, \Arn, geleitet die Schattenläufer und die Kleine weiter nach Osten.

\section{Die Ungehörigen}

Sie ziehen weiter von \Mundis nach \Helim

\section{Rhin}

Der Gesandte und die Gruppe der Schattenläufer kommt mit dem Mädchen nach Rhin, die Hauptstadt Rhingells. Vor drei Jahren starb der alte Herrscher von Rhingell. Die Schattenläufer sind bei den Herren von Rhingell seit Jahren nicht gut angesehen. Die Graue Herrin gehörte zur Familie der Herrscher von Rhingell und der Streit zwischen den Leuten aus dem Enland und der Grauen Herrin ist bekannt. Der Gesandte Arn aus der Bergmark geht mit dem Wesen und den Neuigkeiten aus dem Enland zum Hofmarschall und zum Ältestenrat.

\section{Der König von Rhingell}
Der junge König sandte seinen älteren Bruder vor 4 Wochen an der Spitze einer kleinen Armee in die Südlichen Waldlande, um dort für Ordnung zu sorgen. Im Waldland war eine große Unruhe nahe der Stadt Braucheln entstanden, weil wieder und immer wieder wilde Kreaturen aus den Wäldern stürmten und alles angingen, was sich ihnen in den Weg stellte. Dadurch wurden immer weniger Bäume aus den Westlichen Riesenwäldern gefällt. Diese Bäume dienen als wichtigsten Handelsgut zwischen den Leuten von Rhingell und dem Land der Erzgräber im Südosten. Holz, Kohle und Nahrung tauschen die Leute von Rhingell gegen Waffen und Werkzeuge.

Der junge König weiß nicht, was er machen soll. Er fürchtet das kleine Wesen. Dazu kommen beunruhigende Gerüchte aus dem Norden über Bestien und Eiswölfe. 

\section{Gerüchte in der Stadt}
Der Ältestenrat ist mit dem jungen König unzufrieden. Er hat es versäumt, wie schon sein Vater, die Graue Herrin in ihre Schranken zu weisen und sie daran zu erinnern, für die Leute des Enlandes zu sorgen anstelle der Bangiri. Seit dem Tod des alten Herrschers trieb es die Graue Herrin besonders schlimm, ohne dass der junge Herr ihr Einhalt gebot. Um die Sicherheit der Wege in die Bergmark im Norden und im Westen sei es schlimm bestellt. Berichte über plündernde Bangiri und Bestien im Norden bringen ihn nicht zum Handeln. Auf Berichte aus der Bergmark, dass sich die Eiswölfe sammeln, reagiert er nicht.

\section{Die geheime Gesandtschaft}
Der Ältestenrat beschließt ohne das Wissen des jungen Königs, eine Gesandtschaft zur Stadt Tors am Dreifluss zu schicken. Man will den König durch seinen Bruder aus dem Süden ersetzen. Die Gesandten sollen weiter zu den Erzgräbern ziehen, um diese um Beistand zu bitten. Die Kleine soll daran teilnehmen. Der Herr der Erzgräber, eine sehr weise Kreatur, soll sich ein Urteil über sie bilden. Zusätzlich ziehen Schattenläufer mit ihnen, die außerhalb der Stadt gewartet haben.

\section{Toris}
Die Stadt Toris liegt wenige Tage von Rhin entfernt vor der Mündung des Rhingells in den Dreifluss. Auf der gegenüberliegenden Seite am Dreifluss ist ein Handelsposten der Erzgräber.

\section{Der Rat von Toris}
Die Gesandtschaft berät mit dem Ältestenrat der Start Toris. Auch dort ist man beunruhigt. Ein schweres Hochwasser verbietet jedoch eine Fahrt über den Fluss. Es wird beschlossen, nach Süden am Dreifluss nach Braucheln am See zu reisen, um dort über das Wasser zu setzen.

\section{Nach Süden}
Nach vier Tagen erreicht die Gruppe die kleine Stadt Planis vorm Grünarm. Dort ist man sehr beunruhigt. Wilde Kreaturen streifen bis an die Brücke heran, die nach Süden über den Fluss Grünarm führt. Seit Tagen wagt sich niemand mehr auf die Felder im Süden. Die Streitmacht der Herren von Rhingell zog vor 28 Tagen vorbei nach Süden. Seither kamen keine Nachrichten mehr die Straße zurück.
