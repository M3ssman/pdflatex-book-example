% include preamble definitions
% what kind of document to use
%\documentclass[12pt,a4paper,onecolumn,twoside]{book}
\documentclass[11pt]{book}

% \usepackage[a4paper,left=3.5cm,right=2.5cm,bottom=3cm,top=3cm]{geometry}
\usepackage[ngerman]{babel}

% Korrekte Darstellung der Umlaute
\usepackage[utf8]{inputenc}

% set default font to sans serif font
\renewcommand{\familydefault}{\sfdefault}

% command for double quotas
\newcommand{\q}[1]{``#1''}

% include names for setting
% tern und enland
\newcommand{\Tern}{Tern}
\newcommand{\GrauerTurm}{Grauer Turm}
\newcommand{\GraueHerrin}{Graue Herrin}
\newcommand{\Ternweg}{{\Tern}weg}
% auch Zahl 7
\newcommand{\Sepa}{Seepa}
\newcommand{\Spea}{Spea}
\newcommand{\Siebenhoffnung}{Siebenhoffnung}
\newcommand{\Sturmkind}{Sturmkind}
\newcommand{\Daimon}{Daimon}
\newcommand{\Bangiri}{Bangiri}
\newcommand{\Pato}{Pato}
\newcommand{\Papato}{Papato}
\newcommand{\Arwed}{Alfried von {\Tern}}
\newcommand{\Enland}{Enland}
\newcommand{\Enlaender}{Enländer}
\newcommand{\Enlander}{Enländer}
\newcommand{\EnlanderGarde}{{\Enlander} Garde}
\newcommand{\Schattenjager}{Schattenläufer}
\newcommand{\Schattenlaufer}{Schattenläufer}
\newcommand{\Schajagi}{Scha-jagie}
\newcommand{\Berna}{Berna}
\newcommand{\AltBerna}{Alt {\Berna}}

% schattenläufer
\newcommand{\Eno}{Eno}
\newcommand{\Bomar}{Bomar}
\newcommand{\Dolo}{Dolo}
% Schwiegersohn von Eno
\newcommand{\Nox}{Nox}
% Frau von \Nox
\newcommand{\Mena}{Mena}
% Kind 1 von \Nox und \Mena
\newcommand{\Umbra}{Umbra}
% Kind 2+3 von \Nox und \Mena
\newcommand{\Ena}{Enna}
\newcommand{\Enno}{Enno}
% ?
\newcommand{\Lobo}{Lobo}
% Freund von \Umbra ?
\newcommand{\Tremor}{Tremor}
% Heilerin
\newcommand{\Salbana}{Salbana}
% auch die Zahl 4
% AltBerna
\newcommand{\Tea}{Teha}
\newcommand{\Marn}{Marn}
\newcommand{\Piedo}{Piedo}
% Kap2
\newcommand{\Balasar}{Balasar}
\newcommand{\BalasarSpaltfuss}{{\Balasar} Spaltfuß}
\newcommand{\Haturan}{Haturan}
\newcommand{\HaturanEisenhorn}{{\Haturan} Eisenhorn}
\newcommand{\Golga}{Golga}
\newcommand{\Gewaro}{Gewaro}

%Zahlen 1-7+0
\newcommand{\Nia}{Nia}
\newcommand{\Dua}{Dua}
\newcommand{\Ria}{Ria}
%\newcommand{\Tea}{Tea}
\newcommand{\Pia}{Pia}
\newcommand{\Exa}{Exa}
%\newcommand{\Sepaa}{Seepa}
\newcommand{\Oka}{Oka}

% lobarn
\newcommand{\Lobarn}{Lobarn}
\newcommand{\Vester}{Vester}
\newcommand{\Naimo}{Naimo}

% nordmark
\newcommand{\Nordmark}{Nordmark}
\newcommand{\Bergmark}{Bergmark}
\newcommand{\Ipes}{Ipes}
\newcommand{\Bron}{Bron}
\newcommand{\Bornhold}{Bornhold}
\newcommand{\Arn}{Arn}
\newcommand{\Eislaufer}{Eisläufer}
\newcommand{\Eisbestien}{Eisbestien}

% rhinland
\newcommand{\Rhinland}{Rienland}
\newcommand{\Rhingell}{Riengell}
\newcommand{\Mundis}{Mundis}
\newcommand{\Helin}{Helin}
\newcommand{\Golrin}{Golrin}
\newcommand{\Rhinburg}{Rienburg}
\newcommand{\Rhin}{Rien}

% freiberge
\newcommand{\Freiberge}{Freiberge}
\newcommand{\Sudern}{Südern}
\newcommand{\Nachtspringe}{Nachtspringe}

% bergmark
\newcommand{\Kogida}{Kogida}

% dreifluss
\newcommand{\Dreifluss}{Dreifluß}
\newcommand{\Tars}{Tars}
\newcommand{\Toris}{Toris}
\newcommand{\Planis}{Planis}
\newcommand{\Grunarm}{Grünarm}

% grünland
\newcommand{\Grunland}{Grünland}
\newcommand{\Braucheln}{Braucheln}
\newcommand{\Darmis}{Darmis}
\newcommand{\Darmon}{Darmon}
\newcommand{\Riesenwald}{Riesenwald}

% eisenmeister
\newcommand{\Eisenland}{Eisenland}
\newcommand{\Eisenmeister}{Eisenmeister}
\newcommand{\Dariom}{Dariom}
\newcommand{\Abaton}{Abaton}
\newcommand{\Safir}{Safir}

% oedland
\newcommand{\Oedland}{Ödland}
\newcommand{\Staubteufel}{Staubteufel}


\begin{document}

% kapitel1.tex
\begin{Large}
\begin{itshape}
Es war einmal ein Sturm.\\
Am Tag blieb die Sonne hinter Wolken. Wolken, dicht und schwarz wie Pech. Der Tag wurde zur Nacht und in der Nacht leuchteten Blitze am Himmel, dass es hell war wie am Tage. Der Wind jagte von den Bergen über das Land. Er stürzte zu dem riesigen See in der Mitte des kleines Reiches, das von seinen Bewohnern {\Enland} genannt wurde.

Am nördlichen Ufer des Sees, das sanft zum Wasser abfiel, lag eine Stadt. Die Stadt hatte feste Mauern aus Stein. Ihre Türme waren hoch wie Riesenbäume. Der Name der Stadt war {\Tern}. {\Tern} war die größte und älteste Stadt im {\Enland}. 

Niemand konnte mehr sagen, seit wann der Sturm tobte. Schwarze Wolken wirbelten am Himmel über der Stadt. Sie kreisten um einen riesigen Turm. Diesen Turm nannte man den "Grauen Turm". Sein Name kam von den hellen Steinen, aus denen er erbaut war. Er überragte alle Häuser und Türme der Stadt um ein Vielfaches. Noch stand der Graue Turm fest wie ein riesiger Fels in dem Wirbelsturm, der um ihn brauste. Ein trotziger König aus Stein, der dem Unwetter die Stirn bot.

Plötzlich erschienen am Fuß des Grauen Turms helle Lichtpunkte. Sie krochen die Mauern hinauf. Schneller und schneller, immer höher und höher. Als sie die höchsten Fenster des Turms erreichten, schoss ein riesiger, weißer Strahl hinauf in die Wolken. Sofort erlosch das Licht und die Stadt versank in Dunkelheit.

Für einen Augenblick war Stille.\\
Selbst das Unwetter hielt den Atem an.\\
Dann bebte die Erde, das Wasser im See brodelte mit einem Mal - ein Knall! Der Graue Turm leuchtete von innen hell auf. Im gleichen Moment zerplatzte er wie ein Ballon und verwandelte sich in eine Wolke aus Schutt, als hätte ihn ein unsichtbares Ungeheuer zerstochen! 

Die Trümmer des Turms ergossen sich über die Stadt. Diese Flut bedeckte Dächer, Tore und Mauern der Häuser, die sich um den Turm drängten. Ein Donnerschlag breitete sich vom Turm aus und lief durch die Stadt, über das Land und das Wasser und trieb Staub und Wellen vor sich her.

Die Wolke aus Schutt wurde zu einem Regen aus Asche, Steinen und Funken. Dieser Regen legte sich über ganz {\Tern} - aber damit nicht genug, er noch weit jenseits der Mauern nieder.

Der Donner zog fort.\\
Der Sturm legte sich zur Ruhe.\\
Und die schwarzen Wolken am Himmel verschwanden.\\
Die Sterne und die Monde kehrten an ihre Plätze zurück. 
Eine klare Nacht senkte sich über das Land. Der See sah mit einem Mal ganz friedlich aus. Seltsame, grüne Lichtsäulen schimmerten über den Bergen, die das Land umgaben. Etwas in dieser Art hatte man im {\Enland} noch nie gesehen. Ihre Lichter tanzten auf dem dunklen Wasser. 

Über der Stadt hing ein Schleier aus Qualm und Dampf. Aus der Mitte stieg Rauch empor, dort, wo einst der Graue Turm stand. Vor wenigen Augenblicken noch der höchste und größte Turm im {\Enland}, wenn nicht in ganz {\Rhingell}, war er zu einem riesigen Krater im Schatten einer riesigen Rauchsäule geworden.
\end{itshape}

\part{Nach dem Sturm}
\chapter{Der \Schattenlaufer}
Der Sturm hatte sich gelegt.\\
Aber die alte Stadt, die einst die stolze Stadt {\Tern} am See war, fand keine Ruhe.
 
Durch weißen Qualm und dunklen Rauch drangen Geschrei und Weinen. Dazu mischte sich das Knistern von vielen Feuern. Am Fuß des Hügels flackerten Flammen aus den Ruinen auf. Der Funkenregen leckte mit tausend brennenden Zungen an den Dächern und Türmen der Stadt {\Tern}. 

Nachdem sie sich von den Schrecken erholt hatten, liefen viele Bewohner zum Ufer. Sie holten Wasser in großen Kübeln und Eimern aus dem See. Neben den {\Enlaender}{n}, die man ihren gesitteten Kleidern und Hemden erkannte, torkelten wilde Kreaturen durch zerstörte Straßen. Es waren große Gestalten mit langen, kräftigen Armen, zottigen Mähnen und gewaltigen Tatzen.

Sie nannten sich selbst "{\Bangiri}", aber für die {\Enlaender} waren sie einfach die "Wilden", weil sie kaum auf zwei Beinen gehen konnten und mehr grunzten als redeten. Sie sahen meist aus wie Bären mit langen Armen und kurzen Beinen, mit strubbeligen Haaren, großen Mäulern und spitzen Zähnen. Die andere Leute in der Stadt nannten die {\Bangiri} einfach die "Fremden", denn sie lebten erst seit einer Generation am See. Es mangelte den {\Bangiri} nicht an Kraft und Schnelligkeit, aber ihnen fehlten die geschickten Finger der \Enland{er}. Und sie waren, nebenbei gesagt, nicht gerade die hellsten Köpfe. 

Die {\Bangiri} taten immer, was die Graue Herrin befahl.\\
Die Graue Herrin bestimmte seit langer Zeit über das {\Enland}. In den Köpfen der {\Enlaender} war sie mit ihrem Herrschersitz verschmolzen, dass man sie nach dem Turm nannte und ihren wirklichen Namen vergaß. Die Graue Herrin hatte die {\Bangiri} als neue, blind ergebene Diener aus Ländern weit im Süden geholt. Sie gab ihnen anstelle von Höhlen und Hütten die Häuser der {\Enlaender} mit Mauern aus Stein und Fenstern aus buntem Glas. 

Wer sein Haus verlassen musste, wurde weder ein Freund der {\Bangiri} noch der Grauen Herrin. Sie hatte keinen offenen Ohren für die Klagen der Vertriebenen. Wer sich gegen die neuen Diener stellte, musste mit schlimmen Strafen rechnen. Weil die {\Bangiri} aber keine begabten Handwerker und Baumeister waren, verfielen jene Teile von {\Tern} mehr und mehr, in denen sie hausten. Die Graue Herrin kümmerte sich derweil um andere Dinge. 

Fast 100 Jahren saß sie oben im Grauen Turm und herrschte über das {\Enland}. Niemals hatte man gehört, dass ein Herrscher im {\Enland} oder sogar in {\Rhingell}, zu dem das {\Enland} gehörte, solange auf dem Thron geblieben war. Nun aber war von dem Turm nur noch eine Rauchsäule übrig und noch konnte keiner sagen, was mit der Grauen Herrin geschehen war. 

Mitten in diesem Durcheinander schlich {\Eno} der {\Schattenlaufer} in die Mitte der Stadt zu den Resten des großen Turms. {\Eno} war ein {\Enlaender} in der Gestalt eines alten Fuchses. Er war schon alt. Sein braunes Fell war von silbernen Strähnen durchsetzt. Er trug einen weiten, dunklen Mantel, der so weit bis auf den Boden reichte, dass man keine Hosen oder gar Füße sehen konnte. Über den Kopf hatte er eine ebenso dunkle, weite Kapuze gezogen. Wenn er stehen blieb, ohne sich zu bewegen, sah er aus wie ein großer, dunkler Stein. Nur sein einziges, grünes Auge funkelte aus dem Schatten. Eine schwarze Augenklappe bedeckte das andere Auge.

Diese Vorsicht war notwendig. {\Schattenlaufer} wurden seit Jahren im ganzen {\Enland} verfolgt und gejagt, weil sie sich offen gegen die Graue Herrin und ihre Helfer stellten. Wurden sie ergriffen, wurden sie hart bestraft. Viele verschwanden auf Nimmerwiedersehen in den Eisengruben, die am Rand der Dunklen Berge weit hinter dem See lagen.\\
Von diesem Schicksal blieb {\Eno}, der von seinen Leuten auch "{\Eno}, der Einäugige" genannte wurde, bislang verschont. 

Die Reste des Grauen Turms bedeckten die Seiten des Hügels, auf dem er einst stand. Viele Brocken waren in den Häusern ringsherum niedergegangen. Türme ohne Dach ragten wie abgebrochene Bäume aus dem Meer der Trümmer. In der Mitte, oben auf dem Hügel, zwischen grauen Steinen und schwarzen Brocken, bemerkte {\Eno} ein helles Leuchten durch den Rauch. 

Er kletterte vorsichtig über einen riesigen Wall aus Schutt und Trümmern. In der Mitte lag eine große Röhre, die weiß wie Silber schimmerte. Sie lag zwischen schwarzen Steinbrocken und war so hoch, das ein gewöhnlicher \Bangiri darin stehen konnte und ein {\Enlaender} schon die Arme ausstrecken musste, um das obere Ende zu berühren. 

Der Weg war nicht weit, aber gefährlich. Es gab für einen {\Schattenlaufer} keine Möglichkeit, hinter einem Busch oder einer Mauer zu verschwinden. Wurde {\Eno} hier gesehen, musste er schnell den Schuttwall erreichen und hinüber steigen. Und was, wenn er dabei seinen Häschern in die Arme lief?

Zusammen mit {\Eno}, dem {\Schattenlaufer}, erreichte ein gewaltiger {\Bangiri} mit Namen {\Pato} die Mitte des Hügels. Die {\Pato} war der Erste der {\Bangiri} in {\Tern}. Er ging zu der Röhre, die ihm nur bis zum Kopf reichte, ohne weiter nach links oder rechts zu sehen. Hätte er sich umgeschaut, wäre ihm der {\Schattenlaufer} gewiss nicht entgangen. Aber er tat es nicht und so bemerkte er nicht, dass er nicht allein war.

{\Pato} war ein Riese. Mit einem Schlag seiner kräftigen Tatzen zerschlug er die glänzende Schale. Dann langte er hinein und zog etwas heraus. Man sah ein Bündel aus schmutzigen Kleidern und einer langen, hellen Mähne. Es war viel kleiner als der {\Bangiri}, gerade so groß wie ein Menschenkind. {\Pato} legte es sanft auf den Boden und betrachtete es eine Weile. 

Als der {\Bangiri} still und nachdenklich vor der Röhre stand, traf ein Kieselstein die Röhre am linken Ende. {\Pato} hob den Kopf und blickte in diese Richtung, doch da war nichts zu sehen. Nur Rauch, Dampf und Trümmern. Er schaute zurück, aber das Bündel lag nicht mehr zu seinen Füßen. Er bemerkte gerade noch im Augenwinkel, dass rechts ein dunkler Mantel wirbelte. {\Eno}, der {\Schattenjager}, hatte es an sich gerissen.\\
Für einen Moment war {\Pato} wie gelähmt, aber dann begriff er, dass seine Beute fort war und stieß ein schauerliches Geheul aus. Da hatte der {\Schattenjager} den Wall bereits erreicht.

{\Eno} hatte gesehen, aus welcher Richtung der {\Bangiri} die Ruinen betreten hatte. Dort rechnete er mit weiteren Verfolgern, darum wählte er eine andere Seite. Kurz nachdem er den Schuttwall überschritten hatte, änderte er ein zweites Mal die Richtung und rannte ein Stückchen nach links. Dann rutschte er geschwind hinunter und blieb still stehen. Er war er eins geworden mit den Schatten in den Trümmern.

Neben dem zottigen Kopf {\Pato}s erschienen oben auf dem Wall weitere Leute. Darunter waren {\Bangiri} und {\Enlander} in blauen Gewändern und mit eisernen Helmen. Sie gehörten zur Garde des Sees. Früher stellten sie die Wächter im {\Enland}, heute fand man sie nur noch selten außerhalb der Stadt {\Tern}. Sie alle machten einen Riesenlärm und purzelten wild durcheinander den Hang hinunter. Da sie nicht mit {\Eno}s Gewitztheit rechneten, der die Richtung direkt nach dem Überschreiten des Walls erneut geändert hatte, kamen sie gut 30 Schritte weit entfernt am Fuße des Abhangs an.

Der {\Schattenlaufer} bewegte sich nicht. Anstatt auszureißen, schloss er sogar noch sein einziges Auge. Nun war gar nicht mehr zu sehen. Und zwischen dem Schutt und im Rauch konnten ihn die {\Bangiri} trotz ihrer feinen Nasen auch nicht wittern. Die Verfolger rasten blind durch die Trümmer davon. {\Eno} machte ein paar Schritte, ehe er im Schatten einer Mauer eine Pause einlegte. 

Er blickte auf das kleine Wesen unter seinem Mantel, dass er die ganze Zeit auf den Armen getragen hatte. Wir hätten es gleich als Menschenkind erkannt, aber {\Eno} nicht. Obwohl er weit gereist und mehr als einmal die Grenzen {\Rhingell}s überschritten hatte, hatte er niemals in seinem Leben einen Menschen gesehen. Was ist das? dachte er. So nackt und kaum Fell. Nur am Kopf war eine lange, helle Mähne. Der {\Schattenlaufer} tastete das Kind vorsichtig mit seinen Tatzen ab. Es war noch Leben in dem kleinen Körper. 

Er schlug wieder den Mantel darüber und drehte sich um. Das Gebrüll der {\Bangiri} entfernte sich und wurde immer leiser. Lautlos und geduldig entfernte sich der dunkle Jäger mit seiner Beute von dem Schuttwall aus den Trümmern des Grauen Turms. Trotz seine Last eilte er geschickt von einer Ecke zur nächsten, ohne das ihn jemand bemerkte. Es waren viele Leute auf den Beinen, viele {\Enlander} und {\Bangiri}, aber der {\Schattenlaufer} konnte allen aus dem Weg gehen.

Nach einer Weile erreichte er eine gerade Straße, die auf ein großes Tor an der Stadtmauer führte. Niemand war zu sehen. Das Tor war sonst sorgfältig verschlossen und bewacht, aber nun stand es halb offen. Einer seiner großen Flügel lag zertrümmert vor der Stadt, der andere lehnt halb herausgerissen an der Wand. 

Kurz vor dem Tor hörte {\Eno} ein Zischen. Hatte man ihn doch gefunden? Nein, aus einem Kellerfenster zu seinen Füßen ertönte zweimal der Ruf einer Eule. {\Eno} hielt an. Die Tür des Hauses öffnete sich, und ein weiterer Mantel wehte auf die Straße. Dieser Mantel war ebenso weit und dunkel wie der des ersten {\Schattenlaufer}s. Als er neben {\Eno} hielt, konnte man sehen, dass der zweite {\Schattenlaufer} fast einen ganzen Kopf größer war. Unter seiner Kapuze funkelten zwei hell-blaue Augen.\\
\q{{\Bomar}!} flüsterte {\Eno}. \q{Warum bist mir gefolgt? Du bist schließlich nicht mehr der Jüngste!}\\
\q{Wenn du denkst, du kannst den ganzen Spaß für dich allein haben, hast du dich getäuscht. Du warst mehrere Stunden fort. Schließlich sagte {\Dolo}, jemand sollte nach dem Rechten sehen! Und dann hat er mich geschickt.}\\
\q{Komm}, {\Eno} winkte seinem Gefährten. \q{Verlassen wir die Stadt!}\\
\q{Wir sollten ihnen helfen}, meinte {\Bomar}, \q{es sind schließlich unsere Leute.}\\
\q{Du hast Recht. Aber denke daran, wie der der Vorsteher entscheiden würde. Es ist zu gefährlich, denn man hat mich leider bemerkt! Die Straßen zum See wimmeln von {\Bangiri}s und Wächtern der {\EnlanderGarde}!}

Sie sahen niemanden auf dem alten, gepflasterten Weg, der aus der Stadt führte. Trotzdem liefen sie nicht auf der Straße entlang, sondern einige Schritte neben ihr. So konnten sie schnell verschwinden, falls doch jemand hier entlang kam. 

Die Straße führte ein ganzes Stück vom See entfernt in Richtung der Berge. Durch das Unglück mit dem Grauen Turm, den Rauch und die Feuer waren die Leute zum See geeilt, um von dort Wasser zu holen. Zur Sicherheit liefen sie jedoch einige Schritte neben dem Weg. So kamen sie langsamer vorwärts, waren aber geschützt vor unangenehmen Überraschungen.

Die Ebene, durch die die Straße führte, war kahl und schwarz, als hätte hier schon einmal ein großer Brand gewütet. Alle Häuser, an denen sie vorbei kamen, waren verbrannt und verlassen. Dieses Feuer musste jedoch schon Jahre zurück liegen, denn einige Mauern waren mit der Zeit bereits von wilden Ranken überzogen. Schweigend gingen die beiden {\Schattenlaufer} weiter. {\Bomar} bemerkte zwar, dass {\Eno} langsamer und gebeugter ging als gewöhnlich, sprach ihn aber nicht darauf an.

An einer Gabelung des Weges folgten sie dem Pfad, der sie wieder in Richtung des Sees führte. Kurz nachdem sie einen Hügel überquerten, legte sie eine Pause ein. In der Ferne sah man die Berge ringsherum, die die Grenze des alten \Enland{es} anzeigten. Das seltsame, grüne Licht am Himmel spiegelte sich in den schneebedeckten Gipfeln der Eisenberge. Es sah aus, als würden diese Berge hinter der Stadt {\Tern} beginnen. Aber in Wahrheit brauchte man mehrere Tage, um sie zu erreichen.

Hinter einem flachen Stein tauchte ein dunkler Umhang auf. Dieser {\Schattenlaufer} trug keine Kapuze, so dass man ihm ins Gesicht schauen konnte. Er war sehr viel kleiner als die anderen beiden und sah aus wie eine Ratte. Das graue Fell war über von dunklen Strähnen durchzogen. An seinem Kinn hing ein dünner, weißer Bart, der ihm bis auf die Brust reichte. Sein Name war {\Dolo}. Er war der älteste der drei {\Schattenlaufer}.\\
\q{{\Eno}! Wie gut, dass du ihn gefunden hast, {\Bomar}!}, krächzte er.\\
\q{Gewiss habe ich das, weiser {\Dolo}}, brummte {\Bomar} und zog die Kapuze herunter. 
{\Bomar} hatte die Gestalt eines Luchses. Rechts und links prangte ein mächtiger, weißer Backenbart, der ihm bis auf die Schultern fiel. Auch {\Eno} setzte die Kapuze ab. Sein Fell schimmerte weiß in der Dunkelheit. Eine dunkle Augenklappe bedeckte sein linkes Auge.

\q{{\Eno}}, sagte {\Dolo} plötzlich, \q{Was verbirgst du unter deinem Gewand?}\\
{\Eno} kniete nieder und schlug seinen Mantel zurück. Nun konnten die anderen sehen, was er aus den Ruinen des Turms mitgebracht hatte.\\
\q{Was ist das?} zischte {\Bomar} unruhig. \q{Ich dachte, die Nacht war für dich zu anstrengend gewesen! Aber so etwas! So etwas habe ich noch nie gesehen!}\\
Und {\Eno} berichtete seinen Gefährten, wo er das Kind gefunden hatte. Die Begegnung mit {\Pato}, dem ersten der {\Bangiri} im {\Enland}, verschwieg er jedoch.\\
{\Dolo} berührte den kleinen Kopf des Kindes und betrachtet ihn. Das Kind hielt die Augen geschlossen, als würde es schlafen. \q{In der Mitte, sagst du?}, wiederholte {\Dolo} langsam. \q{Direkt unter den Hallen des Turmes hielt man die Unglücklichsten der Unglücklichen gefangen. Es heißt, die Graue Herrin hatte mit ihnen etwas ganz Besonderes vor. Nun, ich glaube, es ist verletzt. Aber es wird durchkommen.}\\
{\Bomar} schüttelte sein mächtiges Haupt. \q{Eine Gefangenen direkt aus dem Turm! Das bedeutet Ärger. {\Nox} wird darüber nicht erfreut sein.}

Sie schauten zu der Stadt am See.\\
Der Rauch aus den Trümmern der Stadt {\Tern} zog wie ein heller Schleier über das Wasser.\\
\q{Der Graue Turm ist gefallen. Mag die Graue Herrin, die furchtbare, alte Hexe, zusammen mit seinen Mauern in die andere Welt gegangen sein! Nein, ich weine ihr keine Träne nach. Die, die noch übrig sind, haben andere Sorgen, als entlaufene Gefangenen zu suchen}, sagte {\Nox} schließlich.  
\q{Ich konnte das Kleine nicht dort lassen, zwischen all der Verwüstung, wo ich die Möglichkeit hatte, wenigstem ihm zu helfen.}\\
\q{Hoffen wir, dass es eine Weile dauert, bis sie es vermissen}, entgegnete {\Bomar}.\\
\q{Jetzt bleibt es, wo es ist. Wir werden es nicht zurückbringen. Aber nun lasst uns gehen! Es wird bald hell, und bis zum Dorf ist es noch ein weiter Weg.}

{\Bomar} nahm {\Eno} das Kind ab. Dieser war vom langen Tragen erschöpft. Wie man sehen konnte, war er ein alter Fuchs, und hatte, wie seine Gefährten, schon viele Sommer und Winter im {\Enland} und in {\Rhingell} kommen und gehen sehen. Er war froh, dass er sich mit den anderen beiden die Last teilen konnte. Gemeinsam folgten sie dem Pfad, der sich zwischen den Hügeln oberhalb des Sees entlang schlängelte.

\chapter{Der Amtmann von {\Berna}}
Als die Sonne schon hoch am Himmel stand, kamen sie in das Dorf {\Berna}.\\
Das Dorf bot, wie viele Siedlungen im {\Enland} zu dieser Zeit, einen traurigen Anblick. Es erstreckte sich eine viertel Wegstunde oberhalb des Sees an einem Hang. Fast die Hälfte der Häuser von {\Berna} war mit den Jahren verlassen, aber die große Gemeindehalle am oberen Ende des Dorfes wurde eifrig gepflegt. In einiger Entfernung, direkt unten am Ufer des Sees, standen weitere Häuser, die jedoch alle verlassen aussahen.

{\Bomar} selbst stammte aus diesem Dorf. Er war hier geboren und aufgewachsen.\\
Die drei {\Schattenlaufer} hatten ihre weiten Mäntel verstaut. {\Eno} wickelte das Kind in seinen Mantel, damit es nicht gleich für jeden zu sehen war. Als sie das dritte Haus erreicht hatten, schlug {\Bomar} die schwere Türe zurück und trat vor den anderen ein. Es war sein Haus.

Kurze Zeit später klopfte es.\\
{\Bomar} trat ans Fenster. \q{Ah! Es ist {\Mena}!}\\
{\Mena} war die Tochter von {\Eno}, dem Einäugigen. Obwohl sie wusste, dass er in seinem Alter das Recht zu solchen Unternehmungen hatte, sah sie es doch nicht gerne. Nach den Sitten der {\Schattenlaufer} waren die gefährlichsten Aufgaben den Ältesten vorbehalten. Sie hatten die meiste Erfahrung und hatten, wie man sagt, das Leben bereits gelebt. Wenn sie nicht von ihren Ausflügen zurückkehrten, war der Verlust für die Gemeinschaft der {\Schattenlaufer} nicht so groß, als wäre ein junger Vertreter von ihnen gegangen, der noch viele Jahre zu leben hatte. 

Auf der anderen Seite hatten die Alten bei allen übrigen Angelegenheit im Dorf kein Stimmrecht, auch wenn sie bei Gemeindeversammlungen natürlich ihre Meinung sagen dürften. Der Grund war ebenso einfach: sie hatten mit den Folgen einer wichtigen Entscheidung am kürzesten zu leben, und die jüngeren viel länger. Darum lag es in der Verantwortung der jüngeren Leute, egal ob Mann oder Frau, solche Entscheidungen zu treffen.

{\Mena} hatte die Gestalt einer Tigerkatze. Sie trug eine grobe, braune Hose, eine dunkle Bluse und sah {\Nox} ansonsten überhaupt nicht ähnlich.\\
\q{{\Nox} möchte dich sehen, Vater, alleine}, sagte sie und fügte leise hinzu: \q{Und du sollst vorzeigen, was du aus der Stadt mitgebracht hast.}

{\Eno} nahm das Kind auf seinen Arm. Es lag immer noch leblos da, die Augen geschlossen.
Beide verließen das Haus und gingen die breite Straße den Hügel hinauf, mit dem See im Rücken. 
Immer wieder kamen sie an Häusern vorbei, deren Fensterläden sorgfältig verschlossen waren. Niemand wohnte mehr dort, aber sie sahen trotzdem aus, als würden sie nur darauf warten, dass ihre alten Bewohner eines Tages plötzlich heimkehrten. Vor einem flachen Haus mit offenen Fenstern hielten sie an. Es war das letzte Haus vor der Gemeindehalle, die auf dem höchsten Punkt des Dorfes erbaut war.

\q{Was ist in der Stadt {\Tern} geschehen?}, fragte {\Mena}, als sie die Tür öffnete.\\
\q{Ich kann es dir nicht sagen}, antwortete {\Eno}, \q{und ich weiß auch nicht, ob es tatsächlich so ist, oder ob ich mich täusche. Aber in dem Moment, als der Turm in Trümmer ging, habe ich plötzlich gespürt, dass mit dem Turm die Macht der {\GraueHerrin} zerfallen ist.}\\
\q{Wirklich?}\\
Sie hatten einen hohen Raum betreten, in dem sich ein Dutzend Leute versammelt hatten. Einer von ihnen, mit der Gestalt eines Hundes mit pechschwarzem Fell, erhob sich. Er trug eine ebenso braune Hose wie {\Mena}, aber sein Hemd war blau und mit feinen, silbernen Stickereien verziert. Auf seine linken Brust prangte ein gewundenes Band - das Zeichen eines Amtmannes im {\Enland}. Sein Name war {\Nox} und er führte die Gemeinde von {\Berna} in diesen schwierigen Zeiten. Dann sagte er weiter: \q{Geht jetzt bitte und wartet einen Moment vor der Halle. Ich möchte mit {\Eno} allein reden.}

Nachdem alle Anwesenden, auch {\Mena}, den Raum verlassen hatten, räumte {\Nox} den Tisch in der Mitte. {\Eno} legte das Kind darauf. Sie sahen es beide an, als {\Nox} endlich fragte: \q{{\Eno}, wie nahe warst du am Grauen Turm, als er explodierte?}\\
\q{Ich war bis zum alten Markt vorgedrungen.}\\
\q{Was denkst du, was mit denen geschehen ist, die sich im Turm aufgehalten haben?}\\
{\Eno} antwortete nicht.\\
\q{{\Eno}, sagte {\Nox}}, und seine Stimme zitterte, \q{ich habe gesehen, wie der Turm zu Staub zerfallen ist. Ich kann mir nicht vorstellen, mit welchem Zaubertrick jemand an diesem Ort überleben konnte. Also, sage die Wahrheit: Wo hast du dieses Wesen gefunden? Wo war es versteckt? Wie war es geschützt? Unsere Leute werden Fragen stellen. Wenn die {\GraueHerrin} umgekommen ist, aber dieses kleine Wesen noch lebt, werden sie es ebenso fürchten!}

Da erzählte {\Eno} dem Amtmann alles der Reihe nach, angefangen von der großen silbernen Röhre, über seine Flucht über den Schuttwall bis hin zu dem Moment, da er {\Bomar} in der alten Allee traf. {\Nox} hörte genau zu.
\q{Wenn das wahr ist, dann muss dieses kleine Wesen sehr wichtig für die {\Bangiri} und die {\GraueHerrin} sein. Oh, {\Eno}, was hast du uns da eingebrockt?}\\

Sie legte das Kind auf eine Bank an der Seite und riefen die anderen wieder herein. Es handelte sich nur um angesehen Leute, die in der Versammlung von {\Berna} Rederecht hatten. Alle waren sehr überrascht, als sie sahen, was {\Eno} aus der Stadt {\Tern} mitgebracht hatte.
\q{Wenn das Kleine so wichtig ist, dass es unter den Hallen in einer Röhre aus weißem Metall gefangen war, sollten wir Acht geben}, sagte einer in der Runde. \q{Vielleicht ist es etwas sehr Gefährliches!}\\
\q{So ein Unsinn!} brummte {\Eno}. \q{Seht es doch nur an! Ich glaube wirklich nicht, dass wir es fürchten müssen! Wir müssen dafür sorgen, dass es wieder zu Kräften kommt. Es ist unsere Pflicht jedem, den wir aus den Fängen der {\GraueHerrin} und ihrer Bande retten können, Unterschlupf zu gewähren!}\\
\q{Die Dinge sind nicht immer so, wie sie scheinen}, entgegnete {\Nox}. \q{Und die Feinde unserer Feinde nicht immer unsere Freunde. Ich habe ähnliche Befürchtungen. Auch dürfen wir nicht vergessen, dass es {\Pato}, der Anführer der {\Bangiri}, zuerst entdeckt hat. Was passiert, wenn er danach sucht? Er gibt Nichts her, was er zuerst in seinen Pranken hatte.}\\

\q{Das kann uns von Nutzen sein}, meinte daraufhin eine graue Füchsin in einem blauen Gewand, die an die Bank getreten war um das Kind zu sehen. \q{Wenn wir Ärger mit den {\Bangiri} oder der {\EnlanderGarde} bekommen, und wir etwas haben, das sie möchten, müssen sie tun, was wir verlangen.}

{\Eno} nickte ihr zu.\\
\q{Wir müssen zunächst dafür sorgen, dass es ihm besser geht}, sagt er.\\
{\Salbana} untersuchte das Kind. Von außen waren keine ernsten Verletzungen zu sehen, nur Schrammen und Beulen. Es hielt die Augen geschlossen, wie schon die ganze Zeit über. Die Heilerin bemerkte einen ruhigen und regelmäßigen Atem.

Sie wickelten das Kleine in eine Decke und legten es {\Nox} in die Arme. Gemeinsam mit {\Salbana}, {\Eno} und weiteren Begleitern gingen sie den Weg hinunter zum See.\\
Vom anderen Ufer, gut eine Tagesreise entfernt, wehte immer noch eine riesige Rauchfahne über das Wasser. Die Lage in der Stadt {\Tern} war immer noch sehr bedrohlich.
Immer wieder flammten neue Brände auf und trieben die Verbliebenen zu neuer Eile.

\q{Wir sollten es nicht im Dorf lassen}, sprach {\Nox} weiter. \q{Solange wir nicht wissen, womit wir es genau zu tun haben und wie es mit der Stadt weitergeht. Ich werde es hinunter zum See bringen, nach {\AltBerna}. Dort leben nur die alte {\Tea} und der alte {\Marn} mit seinem Ziehsohn, {\Piedo}. Sie werden uns helfen. Niemand geht sonst nach {\AltBerna}, dort ist das Kind sicher. Du, {\Salbana}}, fuhr er fort und blickte die Graufüchsin in dem blauen Gewand an, \q{hast dort Ruhe, um dich um deinen kleinen Patienten zu kümmern.}

Die Siedlung {\AltBerna} direkt am See bestand selbst zu ihrer Blütezeit aus nicht mehr als zwanzig Hütten. Die Leute, die hier gewohnt hatten, waren meist einfache Fischer und keine Bauern wie in {\Berna}. Ein schlimmes Hochwasser hatte vor elf Jahren fast alle Bewohner vertrieben. Nur wenige waren geblieben. Zu dem Zeitpunkt, als unsere Geschichte spielt, wohnten dort nur noch drei Leute. Die alte {\Tea} war die Amtfrau von {\AltBerna}. Sie hielt ihrem Dorf die Treue, wie ein Kapitän, der als letzter sein Schiff verlässt. Der alte {\Marn} war in jungen Jahren Fährmann auf dem {\Tern}see gewesen und hatte den {\Rhin} bis nach {\Toris} am {\Dreifluss} befahren. Seit dem Hochwasser kümmerte er sich um den kleinen {\Piedo}, den er in einer Wiege auf dem Wasser entdeckt hatte. Der kleine Junge in Gestalt eines Dachses hatte bei dem Unglück seine Eltern verloren. Er stammte nicht aus {\AltBerna}. Damals litten viele Siedlungen am See unter den Fluten. In welchem Dorf er das Licht des {\Enland}es erblickt hatte, würde für immer ein Geheimnis bleiben.

Der alte {\Marn} wohnte direkt am See. Ein Teil seiner Behausung ragte über der Ufer hinaus. Dorthin gingen die Leute mit dem Kind. {\Nox} klopfte an die Tür. Ein alter Mann in der Gestalt eines Otters mit einem breiten Hut auf dem Kopf öffnete.

\q{Wo ist {\Tea}?} fragte {\Nox}. \q{Wir brauchen eure Hilfe. Und wir brauchen eure Verschwiegenheit.}\\
Wie es der Zufall wollte, saß die Amtfrau von {\AltBerna} zusammen mit {\Piedo} in der Stube. Der Junge wurde hinaus geschickt, während die Erwachsenen drinnen redeten.

Seit dem Hochwasser kümmerte sich {\Marn} um {\Piedo}. Er hatte ihn damals in einer Wiege auf dem Wasser entdeckt. Der kleine Junge in Gestalt eines Dachses verlor bei dem Unglück seine Eltern. Er stammte auch nicht aus {\AltBerna}. Damals litten viele Siedlungen am See unter den Fluten. In welchem Dorf er das Licht des {\Enland}es erblickt hatte, würde für immer ein Geheimnis bleiben.

{\Lobo} sprach kurz mit den beiden Anwohnern, dann führten sie {\Nox} mit dem Mädchen in das kleine Gemeindehaus. Es stand auf hohen Pfählen direkt im See. Unterdessen kam {\Piedo} zurück. Er hatte weiter draußen die Netze kontrolliert, in denen sonst Fische gefangen wurden.

Plötzlich wurde es ganz still in der Hütte. Was war geschehen? {\Piedo} schlich sich an das Fenster neben der Türe. Aber er sah nur, dass alle im Kreis um etwas mitten in der Stube standen.

Drinnen hatte sich das kleine Kind aufgerichtet. Ängstlich schaute es ringsherum auf die Leute. \q{{\Salbana} bleibt bei ihm und wir gehen zur Seite. Ich glaube, es hat Angst.}, sagte {\Tea}.\\
Dieser Rat wurde befolgt.

Der alte {\Marn} kratzte sich am Kopf, wobei ihm der Hut in die Stirn rutschte. \q{Ich wollte nur sagen, wie ich das sehe. Es erinnert mich an die Bilder, von denen ich an den Wasserfällen des Dreiflusses bei {\Darmon} gehört habe. Und die Leute dort, auch die {\Eisenmeister}, redeten von Geistern, oder von {\Daimon}s, so nannten sie diese Wesen: {\Daimon}s.}

Die übrigen Leute murmelten zustimmend. Das Wesen war ihnen unheimlich. {\Salbana} reichte dem Kind die Hand hin. Nach einem Moment legte das Kind seine kleine Hand in {\Salbana}s Tatze und blickte die Heilerin mit großen Augen an.

\q{Ich fühle ein gutes, tapferes Herz schlagen}, sagte {\Salbana}, \q{Was soll der Seemannsgarn? Die alten Geschichten bleiben Geschichten. Die Graue Herrin hatte gewiss furchtbare Zauberkräfte, aber sie war auch die letzte Hexe im {\Enland}. Wir leben jetzt und hier.} Dann wandte sie sich an die Kleine: \q{Wie ist dein Name? Du? Dein Name?}. Die Kleine blickte {\Salbana} an und sagte kein Wort.

\q{Ich glaube nicht an Gespenster oder Magie, von der ich nur gehört, sie aber nie gesehen habe}, begann {\Eno} schließlich. \q{Und ich habe auf meinen Reisen bis in die Kalte Steppe oder die Riesenwälder eine Menge ungeheuerlicher und abenteuerlicher Dinge gehört. Ich habe eine gute Seele gespürt. Und da ich geschworen habe, jedem zu helfen, der aus dem Grauen Turm entflieht, habe ich es mit mir genommen.}

\q{Vor Geistern und Geschichte aus fernen Ländern habe ich auch keine Angst}, antwortete {\Nox}, \q{Aber ich fürchte die {\Bangiri} und {\Pato}, denn sie sind uns nahe und wir wissen, wozu sie im Stande sind. Und ich will gar nicht von der {\EnlanderGarde} reden, die ihr Land ebenso im Stich gelassen haben wie die {\GraueHerrin}. Auch wenn sie in der Stadt {\Tern} gerade genug zu tun haben, so kann es doch sein, dass sie nach unserem kleinen ScGast suchen, wenn sie sich daran erinnern. Darum bitte ich euch, es einige Tage bei euch aufzunehmen. Solange, bis wir wissen, wie sich die Dinge in der Stadt entwickeln.}\\

Während sich {\Nox}, {\Tea}, {\Marn} und {\Eno} weiter mit ihren Begleitern unterhielten, hielt {\Salbana} immer noch die Hand des kleinen Mädchens. Das Menschenkind hatte sich in dem Bett, auf das man es gelegt hatte, aufgesetzt und zog eine dunkle Decke über seinen nackten Bauch.\\
\q{Es versteht uns nicht. Wir wissen noch nicht einmal seinen Namen.}, sagte {\Salbana}. \q{Aber irgendwie müssen wir es doch anreden, jetzt, wo es bei uns bleiben soll.}

{\Nox} setzte sich neben dem Mädchen auf eine Holzkiste und schaute es lange an.\\
\q{Alle Dinge brauchen einen Namen}, sagte der Amtmann. \q{Wie soll man sonst später einmal ihre Geschichte erzählen?} Die Umstehenden nickten ihm zu. Da beugte sich {\Nox} ganz nahe zu dem Mädchen herunter und schaute ihm ins Gesicht. Sie blickte ihn ebenfalls an, ohne den Kopf wegzudrehen. \q{{\Oka}, {\Nia}, {\Dua}, {\Ria}, {\Tea}, {\Pia}, {\Exa}, {\Sepa}}, zählte {\Nox} leise. Null, Eins, Zwei, Drei, Vier, Fünf, Sechs, Sieben. Es waren alles alte, enländische Zahlwörter. \q{{\Sepa}}, wiederholte das Kind langsam.\\
\q{{\Sepa}. Nun gut.} {\Nox} nickte. \q{Du hast dir für deinen Besuch einen schlechten Zeitpunkt ausgesucht. Das {\Enland} ist heute und hier ein trauriger Ort. Aber vielleicht wird ja nun alles besser.\\
Ein gewaltiger Sturm hat dich in unser Land geweht. Das war am siebenten und letzten Tag der Woche. Darum gebe ich dir, du Kind des Sturmes, den Namen {\Sepa} {\Spea}. {\Sepa} ist das alten enländische Wort für die Zahl Sieben, und {\Spea} steht für die Hoffnung, dass sich die Dinge zum Guten wenden.}\\
\q{{\Siebenhoffnung}!}, rief {\Piedo} und stürzte in die Stube. Die ganze Zeit hatte er am Fenster gelauscht, aber nun hielt er es nicht mehr aus. \q{Das ist aber ein schöner Name!}

\chapter{Das Fest des Frühlings}
In den folgenden Tagen verzog sich langsam der Rauch über der Stadt {\Tern}. In dem Dorf {\Berna} blieben die Leute unruhig. Nachts zogen die alten Schattenläufer um den See, am zweiten Tag ging in aller Frühe {\Nox} mit zwei Begleitern in die Stadt. Sie kehrten abends mit der Dämmerung zurück, konnten aber nur berichten, dass die Wächter der Garde mit den Kriegern der {\Bangiri} in den Hallen am alten Seehafen ein Lager eingerichtet hatten. Außerdem wurden alle Tore und Durchgänge streng bewacht. Es hieß, dass beide, die Garde und die {\Bangiri}, die Tore so streng kontrollierten, weil sie auf der Suche waren. Wonach gesucht wurde, wusste man nicht. Auch von entflohenen Gefangenen redetet niemand.\\
Von der Grauen Herrin gab es immer noch keine Spur. 

In diesen Tagen kehrte der Frühling zurück ins {\Enland}.\\
{\Salbana} verbrachte jeden Tag im Haus des alten {\Marn}. {\Sepa} ging es von Tag zu Tag besser. Nur mit dem Sprechen hatte sie ihre liebe Not, da ihr weder die Sprachen {\Rhinland}es noch des {\Enland}es vertraut waren. So war es schwierig, sich mit ihr zu verständigen. Zu ihrem Glück war {\Mena}, die Tochter von {\Eno} und Gefährtin von {\Nox}, direkt am zweiten Tag in Begleitung ihrer beiden Kinder, {\Enno} und {\Ena}, zu einem Besuch in {\AltBerna} erschienen. Die beiden Zwillinge, ein Junge und ein Mädchen, waren beide so alt wie {\Sepa}. Sie kamen nach ihrer Mutter und hatten die Gestalt kleiner Katzen. Da sie in ihrem Leben noch nie die gruseligen Seemansgeschichten vom alten {\Marn} gehört hatten, kannten sie keine Furcht vor dem fremden Wesen.

{\Ena} war ein schwarze Katze, die in einem fort plapperte, während {\Enno}, dessen Fell von gelben Streifen durchsetzt war, eher still blieb. Zusammen mit {\Piedo}, der Siebenhoffnung von früh bis spät bestaunte, verbrachten sie den ganzen folgenden Tag in der Hütte in {\AltBerna}. 

Das ganze wiederholte sich auch an den folgenden Tagen. Am Morgen des sechsten Tages klopften die beiden in der Frühe an die Türe des alten {\Marn}. Der Alte Otter {\Marn} tat sehr grummelig, freute sich aber über den Besuch. Seit Jahren gab es nicht mehr so viel Trubel am See, und nun ausgerechnet hier, unter seinem Dach!

\q{Was wollt ihr?}, fragte {\Marn} die beiden. \q{Könnt ihr das Kind nicht einen Tag in Frieden lassen?}\\
\q{Aber Großvater!} entgegnete {\Ena}, die, wie gesagt, etwas vorlaut war. \q{Wir müssen uns doch um {\Siebenhoffnung} kümmern!}

\q{Ja, ja, ist ja schon in Ordnung}, knurrte der Alte. Hinter den Kindern trat {\Salbana} in die Stube. Sie trug einen geflochtenen Korb und stellte diesen auf einen Tisch ab. {\Ena} war unterdessen bereits an das Lager von {\Sepa} geeilt, die ihre neue Freundin mit großen Augen anstarrte.

\q{Weißt du, was am Wochenende ist? Nein, das weißt du nicht - woher auch! Am Wochenende hat unsere große Schwester, also {\Umbra}, sie wird mit drei anderen Mädchen und zwei Jungen in den Kreis der Erwachsenen aufgenommen!}, erzählte sie aufgeregt. \q{Dann darf sie bei den {\Schattenlaufer}n mitmachen! Natürlich nur im Geheimen!}\\

Sie hatte sich auf das Bett gesetzt und wurde nun von {\Salbana} sachte zur Seite geschoben, die auf der Decke ein Brett mit Obst und Brot ablegte. \q{Ach!}, seufzte {\Ena} unbeirrt weiter, \q{Das ist ja so unfair! Wir Kleinen dürfen überhaupt keine Abenteuer erleben! Entweder sind wir zu jung, oder es ist zu gefährlich, oder wir haben nicht die richtigen Sachen! Meist ist es gleich Alles zusammen!}\\

\q{Die Anführer der {\Schattenlaufer} hatten gute Gründe, die Dinge so zu regeln, wie sie es getan haben}, entgegnete der alte {\Marn}. \q{Umbra? {\Schattenlaufer}?}, wiederholte {\Sepa}. \q{Was? Nein, {\Umbra} \begin{itshape}ist\end{itshape} kein {\Schattenlaufer}, aber in wenigen Tagen \begin{itshape}kann\end{itshape} sie einer werden! Meine Schwester! Eine {\Schattenlaufer}in! Das ist ja soo spannend!}

Der Alte schüttelte den Kopf.\\
\q{Ach {\Marn}!}, lachte {\Salbana}. \q{Und {\Tea}, {\Piedo} und du, ihr seid natürlich zum Frühlingsfest eingeladen!}\\
\q{Hm!}, entgegnete {\Marn}. \q{Und wer gibt auf unsere Besucherin Acht? Ihr wollt sie doch bestimmt nicht mitnehmen?} 

Zur gleichen Zeit, war {\Nox}, der Amtmann von {\Berna}, gerade mit dem Frühstück fertig. Da stürzte ein altes Weib in Gestalt einer Ziege in seine Stube. \q{Amtmann!}, rief sie schon an der Türe, \q{{\Bomar} schickt mich! Auf dem Weg von der Stadt kommen {\Bangiri} und Krieger der {\EnlanderGarde}! Ein oder zwei Dutzend! Sie ziehen in unsere Richtung, nach {\Berna}! Und sie haben es eilig! Sie werden gleich da sein!}

Was sollte er tun?\\
{\Nox} räumte seinen Tisch, verabschiedete sich von {\Mena} und eilte auf den Platz vor dem Gemeindehaus. Er hatte ihn gerade erreicht, da bog am unteren Ende des Dorfes ein kleiner Trupp um die Häuser.

\end{Large}
\end{document}

