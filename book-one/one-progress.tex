% include preamble definitions
% what kind of document to use
%\documentclass[12pt,a4paper,onecolumn,twoside]{book}
\documentclass[11pt]{book}

% \usepackage[a4paper,left=3.5cm,right=2.5cm,bottom=3cm,top=3cm]{geometry}
\usepackage[ngerman]{babel}

% Korrekte Darstellung der Umlaute
\usepackage[utf8]{inputenc}

% set default font to sans serif font
\renewcommand{\familydefault}{\sfdefault}

% command for double quotas
\newcommand{\q}[1]{``#1''}

% include names for setting
% tern und enland
\newcommand{\Tern}{Tern}
\newcommand{\GrauerTurm}{Grauer Turm}
\newcommand{\GraueHerrin}{Graue Herrin}
\newcommand{\Ternweg}{{\Tern}weg}
% auch Zahl 7
\newcommand{\Sepa}{Seepa}
\newcommand{\Spea}{Spea}
\newcommand{\Siebenhoffnung}{Siebenhoffnung}
\newcommand{\Sturmkind}{Sturmkind}
\newcommand{\Daimon}{Daimon}
\newcommand{\Bangiri}{Bangiri}
\newcommand{\Pato}{Pato}
\newcommand{\Papato}{Papato}
\newcommand{\Arwed}{Alfried von {\Tern}}
\newcommand{\Enland}{Enland}
\newcommand{\Enlaender}{Enländer}
\newcommand{\Enlander}{Enländer}
\newcommand{\EnlanderGarde}{{\Enlander} Garde}
\newcommand{\Schattenjager}{Schattenläufer}
\newcommand{\Schattenlaufer}{Schattenläufer}
\newcommand{\Schajagi}{Scha-jagie}
\newcommand{\Berna}{Berna}
\newcommand{\AltBerna}{Alt {\Berna}}

% schattenläufer
\newcommand{\Eno}{Eno}
\newcommand{\Bomar}{Bomar}
\newcommand{\Dolo}{Dolo}
% Schwiegersohn von Eno
\newcommand{\Nox}{Nox}
% Frau von \Nox
\newcommand{\Mena}{Mena}
% Kind 1 von \Nox und \Mena
\newcommand{\Umbra}{Umbra}
% Kind 2+3 von \Nox und \Mena
\newcommand{\Ena}{Enna}
\newcommand{\Enno}{Enno}
% ?
\newcommand{\Lobo}{Lobo}
% Freund von \Umbra ?
\newcommand{\Tremor}{Tremor}
% Heilerin
\newcommand{\Salbana}{Salbana}
% auch die Zahl 4
% AltBerna
\newcommand{\Tea}{Teha}
\newcommand{\Marn}{Marn}
\newcommand{\Piedo}{Piedo}
% Kap2
\newcommand{\Balasar}{Balasar}
\newcommand{\BalasarSpaltfuss}{{\Balasar} Spaltfuß}
\newcommand{\Haturan}{Haturan}
\newcommand{\HaturanEisenhorn}{{\Haturan} Eisenhorn}
\newcommand{\Golga}{Golga}
\newcommand{\Gewaro}{Gewaro}

%Zahlen 1-7+0
\newcommand{\Nia}{Nia}
\newcommand{\Dua}{Dua}
\newcommand{\Ria}{Ria}
%\newcommand{\Tea}{Tea}
\newcommand{\Pia}{Pia}
\newcommand{\Exa}{Exa}
%\newcommand{\Sepaa}{Seepa}
\newcommand{\Oka}{Oka}

% lobarn
\newcommand{\Lobarn}{Lobarn}
\newcommand{\Vester}{Vester}
\newcommand{\Naimo}{Naimo}

% nordmark
\newcommand{\Nordmark}{Nordmark}
\newcommand{\Bergmark}{Bergmark}
\newcommand{\Ipes}{Ipes}
\newcommand{\Bron}{Bron}
\newcommand{\Bornhold}{Bornhold}
\newcommand{\Arn}{Arn}
\newcommand{\Eislaufer}{Eisläufer}
\newcommand{\Eisbestien}{Eisbestien}

% rhinland
\newcommand{\Rhinland}{Rienland}
\newcommand{\Rhingell}{Riengell}
\newcommand{\Mundis}{Mundis}
\newcommand{\Helin}{Helin}
\newcommand{\Golrin}{Golrin}
\newcommand{\Rhinburg}{Rienburg}
\newcommand{\Rhin}{Rien}

% freiberge
\newcommand{\Freiberge}{Freiberge}
\newcommand{\Sudern}{Südern}
\newcommand{\Nachtspringe}{Nachtspringe}

% bergmark
\newcommand{\Kogida}{Kogida}

% dreifluss
\newcommand{\Dreifluss}{Dreifluß}
\newcommand{\Tars}{Tars}
\newcommand{\Toris}{Toris}
\newcommand{\Planis}{Planis}
\newcommand{\Grunarm}{Grünarm}

% grünland
\newcommand{\Grunland}{Grünland}
\newcommand{\Braucheln}{Braucheln}
\newcommand{\Darmis}{Darmis}
\newcommand{\Darmon}{Darmon}
\newcommand{\Riesenwald}{Riesenwald}

% eisenmeister
\newcommand{\Eisenland}{Eisenland}
\newcommand{\Eisenmeister}{Eisenmeister}
\newcommand{\Dariom}{Dariom}
\newcommand{\Abaton}{Abaton}
\newcommand{\Safir}{Safir}

% oedland
\newcommand{\Oedland}{Ödland}
\newcommand{\Staubteufel}{Staubteufel}


\begin{document}

% kapitel1.tex
\begin{huge}
\begin{itshape}
Es war einmal ein Sturm.\\
Am Tag blieb die Sonne hinter Wolken. Wolken, dicht und schwarz wie Pech. Der Tag wurde zur Nacht und in der Nacht leuchteten Blitze am Himmel, dass es hell war wie am Tage. Der Wind jagte von den Bergen über das Land. Er stürzte zu dem riesigen See in der Mitte des kleines Reiches, das von seinen Bewohnern {\Enland} genannt wurde. 

Am nördlichen Ufer des Sees, das sanft zum Wasser abfiel, lag eine Stadt. Die Stadt hatte feste Mauern aus Stein. Ihre Türme waren hoch wie Riesenbäume. Der Name der Stadt war {\Tern}. {\Tern} war die größte und älteste Stadt im {\Enland}. 

Niemand konnte mehr sagen, seit wann der Sturm tobte. Schwarze Wolken wirbelten am Himmel über der Stadt. Sie kreisten um einen riesigen Turm. Diesen Turm nannte man den "Grauen Turm". Sein Name kam von den hellen Steinen, aus denen er erbaut war. Er überragte alle Häuser und Türme der Stadt um ein Vielfaches. Noch stand der Graue Turm fest wie ein riesiger Fels in dem Wirbelsturm, der um ihn brauste. Ein trotziger König aus Stein, der dem Unwetter die Stirn bot.

Plötzlich erschienen am Fuß des Grauen Turms helle Lichtpunkte. Sie krochen die Mauern hinauf. Schneller und schneller, immer höher und höher. Als sie die höchsten Fenster des Turms erreichten, schoss ein riesiger, weißer Strahl hinauf in die Wolken. Sofort erlosch das Licht und die Stadt versank in Dunkelheit.

Für einen Augenblick war Stille.\\
Selbst das Unwetter hielt den Atem an.\\
Dann bebte die Erde, das Wasser im See brodelte mit einem Mal - ein Knall! Der Graue Turm leuchtete von innen hell auf. Im gleichen Moment zerplatzte er wie ein Ballon und verwandelte sich in eine Wolke aus Schutt, als hätte ihn ein unsichtbares Ungeheuer zerstochen! 

Die Trümmer des Turms ergossen sich über die Stadt. Diese Flut bedeckte Dächer, Tore und Mauern der Häuser, die sich um den Turm drängten. Ein Donnerschlag breitete sich vom Turm aus und lief durch die Stadt, über das Land und das Wasser und trieb Staub und Wellen vor sich her.

Die Wolke aus Schutt wurde zu einem Regen aus Asche, Steinen und Funken. Dieser Regen legte sich über ganz {\Tern} - aber damit nicht genug, er noch weit jenseits der Mauern nieder.

Der Donner zog fort.\\
Der Sturm legte sich zur Ruhe.\\
Und die schwarzen Wolken am Himmel verschwanden.\\
Die Sterne und die Monde kehrten an ihre Plätze zurück. 
Eine klare Nacht senkte sich über das Land. Der See sah mit einem Mal ganz friedlich aus. Seltsame, grüne Lichtsäulen schimmerten über den Bergen, die das Land umgaben. Etwas in dieser Art hatte man im {\Enland} noch nie gesehen. Ihre Lichter tanzten auf dem dunklen Wasser. 

Über der Stadt hing ein Schleier aus Qualm und Dampf. Aus der Mitte stieg Rauch empor, dort, wo einst der Graue Turm stand. Vor wenigen Augenblicken noch der höchste und größte Turm im {\Enland}, wenn nicht in ganz {\Rhingell}, war er zu einem riesigen Krater im Schatten einer riesigen Rauchsäule geworden.
\end{itshape}

\part{Nach dem Sturm}
\chapter{Der \Schattenlaufer}
Der Sturm hatte sich gelegt.\\
Aber die alte Stadt, die einst die stolze Stadt {\Tern} am See war, fand keine Ruhe.
 
Durch weißen Qualm und dunklen Rauch drangen Geschrei und Weinen. Dazu mischte sich das Knistern von vielen Feuern. Am Fuß des Hügels flackerten Flammen aus den Ruinen auf. Der Funkenregen leckte mit tausend brennenden Zungen an den Dächern und Türmen der Stadt {\Tern}. 

Nachdem sie sich von den Schrecken erholt hatten, liefen viele Bewohner zum Ufer. Sie holten Wasser in großen Kübeln und Eimern aus dem See. Neben den {\Enlaender}{n}, die man ihren gesitteten Kleidern und Hemden erkannte, torkelten wilde Kreaturen durch zerstörte Straßen. Es waren große Gestalten mit langen, kräftigen Armen, zottigen Mähnen und gewaltigen Tatzen.

Sie nannten sich selbst "{\Bangiri}", aber für die {\Enlaender} waren sie einfach die "Wilden", weil sie kaum auf zwei Beinen gehen konnten und mehr grunzten als redeten. Sie sahen meist aus wie Bären mit langen Armen und kurzen Beinen, mit strubbeligen Haaren, großen Mäulern und spitzen Zähnen. Die andere Leute in der Stadt nannten die {\Bangiri} einfach die "Fremden", denn sie lebten erst seit einer Generation am See. Es mangelte den {\Bangiri} nicht an Kraft und Schnelligkeit, aber ihnen fehlten die geschickten Finger der \Enland{er}. Und sie waren, nebenbei gesagt, nicht gerade die hellsten Köpfe. 

Die {\Bangiri} taten immer, was die Graue Herrin befahl.\\
Die Graue Herrin bestimmte seit langer Zeit über das {\Enland}. In den Köpfen der {\Enlaender} war sie mit ihrem Herrschersitz verschmolzen, dass man sie nach dem Turm nannte und ihren wirklichen Namen vergaß. Die Graue Herrin hatte die {\Bangiri} als neue, blind ergebene Diener aus Ländern weit im Süden geholt. Sie gab ihnen anstelle von Höhlen und Hütten die Häuser der {\Enlaender} mit Mauern aus Stein und Fenstern aus buntem Glas. 

Wer sein Haus verlassen musste, wurde weder ein Freund der {\Bangiri} noch der Grauen Herrin. Sie hatte keinen offenen Ohren für die Klagen der Vertriebenen. Wer sich gegen die neuen Diener stellte, musste mit schlimmen Strafen rechnen. Weil die {\Bangiri} aber keine begabten Handwerker und Baumeister waren, verfielen jene Teile von {\Tern} mehr und mehr, in denen sie hausten. Die Graue Herrin kümmerte sich derweil um andere Dinge. 

Fast 100 Jahren saß sie oben im Grauen Turm und herrschte über das {\Enland}. Niemals hatte man gehört, dass ein Herrscher im {\Enland} oder sogar in {\Rhingell}, zu dem das {\Enland} gehörte, solange auf dem Thron geblieben war. Nun aber war von dem Turm nur noch eine Rauchsäule übrig und noch konnte keiner sagen, was mit der Grauen Herrin geschehen war. 

Mitten in diesem Durcheinander schlich {\Eno} der {\Schattenlaufer} in die Mitte der Stadt zu den Resten des großen Turms. {\Eno} war ein {\Enlaender} in der Gestalt eines alten Fuchses. Er war schon alt. Sein braunes Fell war von silbernen Strähnen durchsetzt. Er trug einen weiten, dunklen Mantel, der so weit bis auf den Boden reichte, dass man keine Hosen oder gar Füße sehen konnte. Über den Kopf hatte er eine ebenso dunkle, weite Kapuze gezogen. Wenn er stehen blieb, ohne sich zu bewegen, sah er aus wie ein großer, dunkler Stein. Nur sein einziges, grünes Auge funkelte aus dem Schatten. Eine schwarze Augenklappe bedeckte das andere Auge.

Diese Vorsicht war notwendig. {\Schattenlaufer} wurden seit Jahren im ganzen {\Enland} verfolgt und gejagt, weil sie sich offen gegen die Graue Herrin und ihre Helfer stellten. Wurden sie ergriffen, wurden sie hart bestraft. Viele verschwanden auf Nimmerwiedersehen in den Eisengruben, die am Rand der Dunklen Berge weit hinter dem See lagen.\\
Von diesem Schicksal blieb {\Eno}, der von seinen Leuten auch "{\Eno}, der Einäugige" genannte wurde, bislang verschont. 

Die Reste des Grauen Turms bedeckten die Seiten des Hügels, auf dem er einst stand. Viele Brocken waren in den Häusern ringsherum niedergegangen. Türme ohne Dach ragten wie abgebrochene Bäume aus dem Meer der Trümmer. In der Mitte, oben auf dem Hügel, zwischen grauen Steinen und schwarzen Brocken, bemerkte {\Eno} ein helles Leuchten durch den Rauch. 

Er kletterte vorsichtig über einen riesigen Wall aus Schutt und Trümmern. In der Mitte lag eine große Röhre, die weiß wie Silber schimmerte. Sie lag zwischen schwarzen Steinbrocken und war so hoch, das ein gewöhnlicher \Bangiri darin stehen konnte und ein {\Enlaender} schon die Arme ausstrecken musste, um das obere Ende zu berühren. 

Der Weg war nicht weit, aber gefährlich. Es gab für einen {\Schattenlaufer} keine Möglichkeit, hinter einem Busch oder einer Mauer zu verschwinden. Wurde {\Eno} hier gesehen, musste er schnell den Schuttwall erreichen und hinüber steigen. Und was, wenn er dabei seinen Häschern in die Arme lief?

Zusammen mit {\Eno}, dem {\Schattenlaufer}, erreichte ein gewaltiger {\Bangiri} mit Namen {\Pato} die Mitte des Hügels. Die {\Pato} war der Erste der {\Bangiri} in {\Tern}. Er ging zu der Röhre, die ihm nur bis zum Kopf reichte, ohne weiter nach links oder rechts zu sehen. Hätte er sich umgeschaut, wäre ihm der {\Schattenlaufer} gewiss nicht entgangen. Aber er tat es nicht und so bemerkte er nicht, dass er nicht allein war.

{\Pato} war ein Riese. Mit einem Schlag seiner kräftigen Tatzen zerschlug er die glänzende Schale. Dann langte er hinein und zog etwas heraus. Man sah ein Bündel aus schmutzigen Kleidern und einer langen, hellen Mähne. Es war viel kleiner als der {\Bangiri}, gerade so groß wie ein Menschenkind. {\Pato} legte es sanft auf den Boden und betrachtete es eine Weile. 

Als der {\Bangiri} still und nachdenklich vor der Röhre stand, traf ein Kieselstein die Röhre am linken Ende. {\Pato} hob den Kopf und blickte in diese Richtung, doch da war nichts zu sehen. Nur Rauch, Dampf und Trümmern. Er schaute zurück, aber das Bündel lag nicht mehr zu seinen Füßen. Er bemerkte gerade noch im Augenwinkel, dass rechts ein dunkler Mantel wirbelte. {\Eno}, der {\Schattenjager}, hatte es an sich gerissen.\\
Für einen Moment war {\Pato} wie gelähmt, aber dann begriff er, dass seine Beute fort war und stieß ein schauerliches Geheul aus. Da hatte der {\Schattenjager} den Wall bereits erreicht.

{\Eno} hatte gesehen, aus welcher Richtung der {\Bangiri} die Ruinen betreten hatte. Dort rechnete er mit weiteren Verfolgern, darum wählte er eine andere Seite. Kurz nachdem er den Schuttwall überschritten hatte, änderte er ein zweites Mal die Richtung und rannte ein Stückchen nach links. Dann rutschte er geschwind hinunter und blieb still stehen. Er war er eins geworden mit den Schatten in den Trümmern.

\section{Die Nacht nach dem Sturm}
Neben dem zottigen Kopf {\Pato}s erschienen oben auf dem Wall weitere Leute. Darunter waren {\Bangiri} und {\Enlander} in blauen Gewändern und mit eisernen Helmen. Sie gehörten zur Garde des Sees. Früher stellten sie die Wächter im {\Enland}, heute fand man sie nur noch selten außerhalb der Stadt {\Tern}. Sie alle machten einen Riesenlärm und purzelten wild durcheinander den Hang hinunter. Da sie nicht mit {\Eno}s Gewitztheit rechneten, der die Richtung direkt nach dem Überschreiten des Walls erneut geändert hatte, kamen sie gut 30 Schritte weit entfernt am Fuße des Abhangs an.

Der {\Schattenlaufer} bewegte sich nicht. Anstatt auszureißen, schloss er sogar noch sein einziges Auge. Nun war gar nicht mehr zu sehen. Und zwischen dem Schutt und im Rauch konnten ihn die {\Bangiri} trotz ihrer feinen Nasen auch nicht wittern. Die Verfolger rasten blind durch die Trümmer davon. {\Eno} machte ein paar Schritte, ehe er im Schatten einer Mauer eine Pause einlegte. 

Er blickte auf das kleine Wesen unter seinem Mantel, dass er die ganze Zeit auf den Armen getragen hatte. Wir hätten es gleich als Menschenkind erkannt, aber {\Eno} nicht. Obwohl er weit gereist und mehr als einmal die Grenzen {\Rhingell}s überschritten hatte, hatte er niemals in seinem Leben einen Menschen gesehen. Was ist das? dachte er. So nackt und kaum Fell. Nur am Kopf war eine lange, helle Mähne. Der {\Schattenlaufer} tastete das Kind vorsichtig mit seinen Tatzen ab. Es war noch Leben in dem kleinen Körper. 

Er schlug wieder den Mantel darüber und drehte sich um. Das Gebrüll der {\Bangiri} entfernte sich und wurde immer leiser. Lautlos und geduldig entfernte sich der dunkle Jäger mit seiner Beute von dem Schuttwall aus den Trümmern des Grauen Turms. Trotz seine Last eilte er geschickt von einer Ecke zur nächsten, ohne das ihn jemand bemerkte. Es waren viele Leute auf den Beinen, viele {\Enlander} und {\Bangiri}, aber der {\Schattenlaufer} konnte allen aus dem Weg gehen.

Nach einer Weile erreichte er eine gerade Straße, die auf ein großes Tor an der Stadtmauer führte. Niemand war zu sehen. Das Tor war sonst sorgfältig verschlossen und bewacht, aber nun stand es halb offen. Einer seiner großen Flügel lag zertrümmert vor der Stadt, der andere lehnt halb herausgerissen an der Wand. 

Kurz vor dem Tor hörte {\Eno} ein Zischen. Hatte man ihn doch gefunden? Nein, aus einem Kellerfenster zu seinen Füßen ertönte zweimal der Ruf einer Eule. {\Eno} hielt an. Die Tür des Hauses öffnete sich, und ein weiterer Mantel wehte auf die Straße. Dieser Mantel war ebenso weit und dunkel wie der des ersten {\Schattenlaufer}s. Als er neben {\Eno} hielt, konnte man sehen, dass er einen Kopf größer war. Unter der  Kapuze funkelten zwei blaue Augen.\\
"{\Bomar}!" flüsterte {\Eno}. "Warum bist mir gefolgt?"\\
"Du warst so lange fort. Stundenlang! Wir haben uns Sorgen gemacht."\\
"Komm", {\Eno} winkte seinem Gefährten. "Verlassen wir diese Ruinen!"\\
"Sollen wir den Leute nicht helfen?" fragte {\Bomar} besorgt.\\
"Jetzt ist nicht die richtige Zeit. Es ist hier viel zu gefährlich für uns! Die Straßen wimmeln von {\Bangiri}s und Kriegern der {\EnlanderGarde}!"

Sie trafen keine Seele auf dem Weg, der aus der Stadt führte. Die Straße führte ein ganzes Stück vom See entfernt in Richtung der Berge. Durch das Unglück mit dem Grauen Turm, den Rauch und die Feuer waren die Leute zum See geeilt, um von dort Wasser zu holen. Zur Sicherheit liefen sie jedoch einige Schritte neben dem Weg. So kamen sie langsamer vorwärts, waren aber geschützt vor unangenehmen Überraschungen.

Die Ebene, durch die die Straße führte, war kahl und schwarz, als hätte hier schon einmal ein großer Brand gewütet. Alle Häuser, an denen sie vorbei kamen, waren verbrannt und verlassen. Diese Ereignis musste jedoch schon Jahre zurück liegen, denn einige Mauern waren von wilden Ranken überzogen. Schweigend gingen die beiden {\Schattenlaufer} weiter.

An einer Gabelung des Weges folgten sie dem Pfad, der sie wieder in Richtung des Sees führte. Kurz nachdem sie einen Hügel überquerten, legte sie eine Pause ein. In der Ferne sah man die Berge ringsherum, die die Grenze des alten \Enland{es} anzeigten. Das seltsame, grüne Licht am Himmel spiegelte sich in den schneebedeckten Gipfeln der Eisenberge. Es sah aus, als würden diese Berge hinter der Stadt {\Tern} beginnen. Aber in Wahrheit brauchte man mehrere Tage, um sie zu erreichen.

Hinter einem flachen Stein tauchte ein dunkler Umhang auf. Dieser {\Schattenlaufer} trug jedoch keine Kapuze, so dass man ihm ins Gesicht schauen konnte. Er war sehr viel kleiner als die anderen beiden und sah aus wie eine Ratte. Das dunkle Fell war über und über mit lila Flecken übersät.\\
"{\Eno}! Du hast ihn gefunden, {\Bomar}!"
"Gewiss habe ich das, {\Dolo}", brummte {\Bomar} und zog die Kapuze herunter. 
{\Bomar} hatte die Gestalt eines Luchses. Rechts und links prangte ein mächtiger, weißer Backenbart, der ihm hinunter bis auf die Schultern reichte. Auch {\Eno} setzte die Kapuze ab. Sein Fell schimmerte weiß in der Dunkelheit. Eine dunkle Augenklappe bedeckte sein linkes Auge.

{\Eno} kniete nieder und schlug seinen Mantel zurück. Nun konnten die anderen sehen, was er aus den Ruinen des Turms mitgebracht hatte.\\
"Was ist das?" zischte {\Bomar} unruhig. "Ich habe noch nie so etwas gesehen!"\\
Und {\Eno} berichtete seinen Gefährten, wo er das Kind gefunden hatte. Die Begegnung mit {\Pato}, dem ersten der {\Bangiri} im {\Enland}, verschwieg er jedoch.\\
{\Dolo} hob den kleinen Kopf an und betrachtet ihn einen Moment. Das Kind hielt die Augen geschlossen, als würde es schlafen. "Aus der Mitte? Direkt unter den Hallen hielt man die Unglücklichen gefangen, mit denen die Graue Herrin etwas besonderes vor hatte. Das bedeutet Ärger. {\Nox} wird darüber nicht erfreut sein."

Sie schauten zu der Stadt am See.\\
Der Rauch aus den Trümmern von {\Tern} zog langsam wie ein heller Schleier über das Wasser.\\
"Der Graue Turm ist gefallen. Mag die Graue Herrin, die furchtbare, alte Hexe, zusammen mit seinen Steinen in eine andere Welt gegangen sein. Ich weine ihr keine Träne nach. Die, die noch übrig sind, haben gerade andere Sorgen, als entlaufene Gefangenen zu suchen", sagte {\Nox} schließlich.  
"{Ich} konnte doch das kleine Kind nicht einfach zurück lassen, in all der Verwüstung."\\
"Dann wollen wir hoffen, dass es noch eine Weile dauert, bis sie es vermissen", antwortete {\Bomar}.
"Lasst uns gehen. Es wird bald hell, und bis zum Dorf ist es noch ein gutes Stück."

{\Bomar} nahm {\Eno} das Kind ab. Das lange Tragen hatte {\Eno} erschöpft. Wie man jetzt sehen konnte, war er bereits ein alter Fuchs, und hatte, so wie seine Gefährten, schon viele Sommer und Winter im {\Enland} und in {\Rhingell} kommen und gehen sehen. Er war froh, dass er sich mit den anderen beiden die Last teilen konnte. Gemeinsam folgten sie dem breiten Pfad.

\section{Das geheime Lager}
Als die Sonne schon hoch am Himmel stand, kamen sie in ein Dorf namens {\Berna}.

PROGRESS

Weiter nach Norden, Richtung \Nordmark. Als der Tag zu Ende ging, hatten sie ein großes Stück des Weges geschafft. Die Wolken stiegen immer noch aus den Trümmern des Grauen Turms in der alten Stadt am Sumpf, als sie am Ende eines weiteren Tages ein kleines Dorf erreichen. Viele Häuser sind verlassen. In einem großen Versteck treffen die \Schattenjager auf eine ganze Gruppe von anderen \Schattenjager{n}. \Nox hat hier das Sagen und befiehlt, dass das Kind hier nicht bleiben kann und dass das geheime Lager verschlossen werden soll. Er rechnet mit einer großen Suche und umherziehenden \Bangiri im \Enland{.}

\section{Der Ternweg}
Nachdem sie den Bergkamm erreicht hatten, folgten sie einem alten Pfad, der in einem weiten Bogen nach Süden führte. Sie kamen nun schneller vorwärts. Die Sterne funkelten hell durch die finstere Nacht. Mal trug \Eno das Mädchen ein Stück, dann \Bomar  \Dolo ließ die beiden alleine und ging neben dem Pfad. Schließlich war er so weit entfernt, dass seine gelben Augen nicht mehr im Dunkel leuchteten.

Gegen morgen, als es wieder Tag wurde, sahen die beiden Jäger im Osten eine weite, grüne Ebene. Weit entfernt auf einem Hügel lag eine kleine Stadt. Zur linken Seite führte der Pfad hinunter. Direkt vor ihnen öffneten sich die Berge. Eine breite, alte Straße schlängelte sich zwischen den Hügeln aus dem \Enland nach Osten. Sie verließen den Pfad und die Berge und eilten neben der Straße weiter, als \Dolo plötzlich zu ihnen zurückkehrte. „Aus dem \Enland kommt eine Gruppe. Sie ziehen auf dem alten \Tern{-Weg} nach \Lobarn.“ meldete er. „Wie viele?“ fragte \Eno. „Ungefähr zehn. Sind noch eine Stunde entfernt.“ „Du hast scharfe Augen. Wir müssen zusehen, dass wir in die Stadt kommen“, antwortete \Eno. „Gehen wir auf der Straße, dort sind wir schneller.“

Ab und zu kamen die drei Jäger an verlassenen Häusern vorbei. Die Dächer waren durchlöchert und die Türen herausgerissen. Hier wohnte schon lange niemand mehr. Der Weg führte nun schnurgerade zu dem Hügel, auf dem die Stadt langsam näher kam. Es dauerte nicht lange, bis hinter ihnen auf dem Weg kleine schwarze Punkte zu sehen waren. „Sie holen auf“, sagte \Dolo{,} „Wir sind zu langsam!“ Und tatsächlich rückten die Punkte näher und wurden größer. Bald hatten auch ihrer Verfolger erkannt, dass ihr Ziel direkt vor ihren Augen lag und verdoppelten noch einmal ihr Tempo. „Wir schaffen es nicht!“ rief \Bomar. „Da vorne liegt ein alter Hof, da können wir uns besser verteidigen!“ Die Jäger, die nun die Gejagten waren, hasteten zu einem Steinmauer. Als sie diese erreichten, sprangen am linken und rechten Rand zwei große, zottelige Wesen hervor. In einem weiten Halbkreis versammelten sich mehr und mehr davon. Die drei Jäger standen mit dem Rücken zur Mauer. „Schagujagi gib her!“, brüllte eines der Monster in der Mitte der Angreifer. \Eno wich zurück. Der \Bangiri hob seine Tatze, und holte zu einem kräftigen Schlag gegen das Mädchen aus. Aber in dem Augenblick, als er es berührte, war es, als schlug ihn eine unsichtbare Keule so heftig, dass er einen riesigen Satz rückwärts machte. Da näherten sich zwei weitere \Bangiri vorsichtig dem Mädchen. Aber als sie weiter herankamen, brüllten sie los und rannten erschrocken davon. Nun machten auch die übrigen \Bangiri kehrt. Sie waren wieder allein.

\section{Die Versammlung zu \Lobarn}
Hinter einer kleinen Brücke stieg der Weg an. Die Stadt \Lobarn lag auf einem  Hügel. Es war es nicht mehr weit. Die\Schattenjager sagten kein Wort, bis sie vor dem Tor von \Lobarn hielten. \Eno pochte an.
Ein Lanzenträger schaute von oben herunter. 
„Wer seid ihr? Was wollt ihr?“
„Ich bin \Eno vom Bund der Schajagi. Ich habe Neuigkeiten aus dem \Enland.“ 
Man ließ sie ein und brachte \Eno, \Dolo, \Bomar und das Mädchen zum Stadthaus, in dem sich viele Leute eingefunden hatten. Sie hatten bemerkt, dass das Wetter umgeschlagen war, kannten aber nicht den Grund. Als sie eintraten, kam ihnen eine alte Gestalt entgegen. 
„Mein Name ist \Vester, ich führe das Wort in dieser Zusammenkunft. Sprecht! Was ist im \Enland geschehen?“
Alle schwiegen und schauten auf \Eno, den\Schattenjager.
„Der Graue Turm ist zerstört. Seine Herrin ist fort.“
Ein Raunen und Flüstern setzte ein.
„Seid ihr Euch sicher, \Eno vom Bund der Schajagi?“
„Ich habe gesehen, wie ein gewaltiger Donnerschlag den Turm in tausend Stücke riss. Aber es gibt auch schlechte Nachrichten. Es sind noch sehr viele \Bangiri in und um \Tern. Einige haben mich und meine Begleiter zurück auf dem alten \Tern{weg} verfolgt.“
„Was wollten diese Biester?“
\Eno trat in die Mitte und legte das kleine Mädchen auf den Boden.
„Was ist das? Wo ist dieses Kind her? Was hat es für eine seltsame Gestalt?“
„Ich habe es in den Ruinen des Grauen Turms einem \Bangiri genommen“, antwortete \Eno.
„Was wollte der \Bangiri von dem Kind?“
„Er wollte ihm Leid zufügen.“
„Gibt es eine Verbindung zwischen dem Fall des Grauen Turms und diesem Geschöpf hier?“, fragte der alte \Vester weiter.
„Ich weiß es nicht“, antwortete \Eno.
„Wenn die \Bangiri es wollen, dann werden sie kommen, um es zu holen“, rief jemand. „Und dann sind wir alle in Gefahr!“
„Ruhe!“, rief \Vester. „Still! Ich führe das Wort!“
Aber mit der Ruhe war es vorbei. Die Leute redeten wild durcheinander: „Was sollen wir tun, wenn morgen tausend \Bangiri vor unseren Toren stehen? Werft die \Schattenjager aus der Stadt! Wir wollen nicht in ihren Streit mit den \Bangiri hineingezogen werden! Falls die Graue Herrin gar nicht fort ist, wird ihre Rache fürchterlich sein. Und seht das Kind! Es hat ja nicht mal vernünftiges Fell! Von welcher Art mag es sein? Hat jemand schon mal so etwas gesehen?“
Nein, niemand hatte so ein Kind gesehen. Die Leute drängten sich um die \Schattenjager und das Kind. Die Kleine versteckte sich ängstlich zwischen \Eno und \Bomar. Ein Gemurmel und Geschiebe entstand. Das Geschrei wurde immer lauter. Der alte \Vester wies die Wächter an, den Saal räumen. Die \Schattenjager und das Kind wurden zu einer Seitentür gebracht und verschwanden.

\section{Das Geschäft mit \Lobarn}
Durch viele schmale Gassen wurden die Schajagi von zwei Wächtern zum Südtor gebracht. Sie verließen die Stadt auf einem alten Weg zwischen Feldern und kleinen Büschen. Nach einer Weile erreichten sie eine kleine Siedlung, die von einem hohen, festen Holzzaum umgeben war. Ein Wächter klopfte mit seinem Schild gegen das breite, hölzerne Tor. Sie wurden eingelassen. Die Wächter führten die \Schattenjager und das Kind zwischen den niedrigen Hütten zu einem großem Holzhaus, das drei Stockwerke hinauf alle anderen Hütten überragte. Einer der Wächter ging hinein und kehrte nach einer ganzen Weile zurück.
„Tretet ein“, sagte er zu den Jägern. „Sie empfängt euch gleich. Ihr bleibt hier, während in \Lobarn über die Sache beraten wird. Hier seid ihr sicher.“ Als er sich zum Gehen wandte, beugte er sich zu \Eno und flüsterte: „Und bleibt hier! \Vester zählt auf euch. Ihr wisst, dass es in \Lobarn schlecht ankommt, wenn ihr heimlich verschwindet.“
\Eno antwortete nicht. Die beiden Wächter verließen ihre Schützlinge.
Als die \Schattenjager das Haus betraten, standen sie in einer hohen Halle aus Holz, ähnlich dem Ratsaal in \Lobarn.
„Kommt herüber! Mein Name ist \Naimo “, sagte eine alte Rotfüchsin in einfachen, braunen Kleidern. „Man hat mir berichtet, was sich der alte \Vester dabei gedacht hat, euch hier zu verstecken. Aber die Kunde aus \Tern möchte ich noch einmal genauer von denen hören, die sie aus dem \Enland hergebracht haben!“
Sie ließen sich in einem kleinen Nebenraum an einem runden, dunklen Tisch nieder und erzählten abwechselnd von \Tern, von dem Donnerschlag und dem Rückweg bis nach \Lobarn. Wie sie ihre Verfolger genau los wurden, verschwiegen sie.

Die Alte hörte die ganze Zeit aufmerksam zu.
„Wie ihr sicher wisst“, sagte sie dann. „ist seit dem Winter ein neuer König in der Ringburg. Als der alte König starb, wurde sein jüngerer Sohn vom Obersten Ratgeber gekrönt. Die Grauen Herrin stand ihm sehr nahe, wie schon seinem Vater. Wer weiß, was der junge König tut, wenn sie fort ist? Oder was die Leute in \Rhin tun werden? Es heißt, dass sie lieber seinem älteren Bruder auf dem Thron wollen. Und auf welche Seite werden sich erst die Ritter der Ringburg stellen? Was ist mit den \Bangiri? Alle diese Dinge muss der Rat bedenken.“  

\section{\Pato{s} Rache}
Nicht alle \Bangiri waren fort. Die Übriggebliebenen zogen den alten Ternweg nach Osten. Nach \Lobarn. Der Anführer \Pato hatte erfahren, dass ein \Schattenjager in den Ruinen von \Tern gewesen war. Darum meinten die \Bangiri{,} dass die \Schattenjager den Grauen Turm zerstörten. Die \Schattenjager haben mit den Jahren viele \Bangiri getötet, weil sie einst von dort vertrieben wurden. \Pato schwor allen Einwohnern des \Enland{s} Rache.

\section{Die \Bangiri kommen}
Eine Rotte von \Bangiri zog nach \Lobarn.
Während die \Bangiri verlangten, dass die Leute dort alle \Schattenjager herausgeben, die sich in der Gegend versteckt halten. Und das Wesen, was in der silbernen Röhre am Grauen Turm lag.
Der alte \Vester erlaubt jedoch den \Schattenjager{n}, zu gehen. Wenn sie ein wildes \Bangiri{-}Junges mitnehmen. Die Leute aus \Lobarn haben es vor einigen Wochen im Süden aufgegriffen. Sie wissen nicht, wohin damit. Wenn die \Schattenjager zusammen mit dem kleinen \Bangiri und dem \Sturmkind \Sena aus der Stadt verschwinden, dann hoffen sie, keine Probleme mit den verbliebenen \Bangiri oder \Rhingell{s} neuem König zu bekommen. \Eno ist davon nicht begeistert. Er ist ein geschworener Todfeind der \Bangiri{,} egal wie alt sie sind. Aber er schlägt ein, wie es die alte \Naimo geraten hat. Sie fliehen in der Nacht weiter nach Osten, nach der Stadt \Mundis am Rand der \Nordmark.

% kapitel2.tex
\chapter{Kapitel 2: Nach \Rhingell}

\section{Nach \Mundis}
Vor der Stadt \Mundis treffen sie auf Gesandte der \Nordmark mit dem obersten Gesandten \Arn von \Ipes{.} Die \Nordmark wird vom Herrn von \Bornhold im Auftrag der Herren von \Rhingell regiert. Dort ist man in Sorge, weil sich im Nordwesten immer mehr Eiswesen sammeln.

Der Herr von \Bornhold lag bis zuletzt im Streit mit der Herrin des Grauen Turms, weil sie beim Raubzug der Eiswölfe erst in den Kampf  eintrat, da die \Nordmark, die \Bergmark und der Westen \Rhingell{s} so verwüstet waren, dass sich die Eiswölfe ins \Enland begaben, wo sie jedoch von der Macht der Grauen Herrin und ihren Dienern, den \Bangiri, völlig ausgelöscht wurden. 

In \Mundis wird die kleine Gesandtschaft von einer Gruppe \Schattenjager zusätzlich begleitet. Ihr gehören \Nox, der Neffe von \Eno, \Umbra und ihr kleiner Bruder \Tremor an. Die Leute aus der \Nordmark verstehen sich gut mit den \Schattenjager{n}. Viele Leute, die das \Enland verlassen haben, sind in die \Nordmark gegangen. Die \Schattenjager verteidigten die neue Heimat auch beim Einfall der Eiswölfe. 

\section{Die Ungehörigen}

Diese Gruppe soll für die Sicherheit des Gesandten sorgen, da die Reise von \Mundis nach \Rhin an \Golrin vorbei führt. Dort hat sich nach dem Krieg ein Ritter niedergelassen, der sich nicht mehr dem Willen des Königs beugt. Er stammt aus der Stadt und war verbittert über ihre Zerstörung.

Die Gesandtschaft folgt dem Fluss \Rhin . Sie sehen die Stadt \Helin am anderen Ufer, die auf einem schrägen Hügel erbaut von hohen Mauern geschützt wird. Im Krieg hatten die Leute dort das Glück, dass die \Eislaufer den Fluss nicht überqueren konnten. So blieben sie vom Schlimmsten verschont. Doch von den höchsten Türmen der Stadt konnte man den Rauch sehen, der tagelang aus der Stadt \Golrin aufstieg.

\section{\Rhin}

Der Gesandte und die Gruppe der Schattenläufer kommt mit dem Mädchen nach \Rhin, die Hauptstadt \Rhingell{s}. Vor drei Jahren starb der alte Herrscher von \Rhingell. Die \Schattenjager sind bei den Herren von \Rhingell seit Jahren nicht gut angesehen. Die Graue Herrin gehörte zur Familie der Herrscher von \Rhingell und der Streit zwischen den Leuten aus dem \Enland und der Grauen Herrin ist bekannt. Der Gesandte \Arn aus der \Bergmark geht mit dem Wesen und den Neuigkeiten aus dem \Enland zum Hofmarschall und zum Ältestenrat.

\section{Der König von \Rhingell}
Der junge König sandte seinen älteren Bruder vor 4 Wochen an der Spitze einer kleinen Armee in die \Grunland{e}, um dort für Ordnung zu sorgen. Im \Grunland war eine große Unruhe nahe der Stadt \Braucheln entstanden, weil wieder und immer wieder wilde Kreaturen aus den Wäldern stürmten und alles angingen, was sich ihnen in den Weg stellte. Dadurch wurden immer weniger Bäume aus den Westlichen Riesenwäldern gefällt. Diese Bäume dienen als wichtigsten Handelsgut zwischen den Leuten von \Rhingell und dem Land der \Eisenmeister im Südosten. Holz, Kohle und Nahrung tauschen die Leute von \Rhingell gegen Waffen und Werkzeuge.

Der junge König weiß nicht, was er machen soll. Er fürchtet das kleine Wesen. Dazu kommen beunruhigende Gerüchte aus dem Norden über \Eisbestien und \Eislaufer. 

\section{Gerüchte in der Stadt}
Der Ältestenrat ist mit dem jungen König unzufrieden. Er hat es versäumt, wie schon sein Vater, die Graue Herrin in ihre Schranken zu weisen und sie daran zu erinnern, für die Leute des \Enland{es} zu sorgen anstelle der \Bangiri. Seit dem Tod des alten Herrschers trieb es die Graue Herrin besonders schlimm, ohne dass der junge Herr ihr Einhalt gebot. Um die Sicherheit der Wege in die \Bergmark im Norden und im \Rhinland im Westen sei es schlimm bestellt. Berichte über plündernde \Bangiri und \Eislaufer in der \Nordmark bringen ihn nicht zum Handeln. Auf Berichte aus der \Bergmark, dass sich \Eisbestien versammeln, reagiert er nicht.

\section{Die geheime Gesandtschaft}
Der Ältestenrat beschließt ohne das Wissen des jungen Königs, eine Gesandtschaft zur Stadt \Toris am \Dreifluss zu schicken. Man will den König durch seinen Bruder aus dem Süden ersetzen. Die Gesandten sollen weiter zu den \Eisenmeister{n} ziehen, um diese um Beistand zu bitten. Die Kleine soll daran teilnehmen. Der Herr der \Eisenmeister, eine sehr weise Kreatur, soll sich ein Urteil über sie bilden. Zusätzlich ziehen \Schattenjager mit ihnen, die außerhalb der Stadt gewartet haben.

\section{\Toris}
Die Stadt \Toris liegt wenige Tage von \Rhin entfernt vor der Mündung des \Rhingell in den \Dreifluss. Auf der gegenüberliegenden Seite am \Dreifluss ist ein Handelsposten der \Eisenmeister.

\section{Der Rat von \Toris}
Die Gesandtschaft berät mit dem Ältestenrat der Stadt \Toris. Auch dort ist man beunruhigt. Ein schweres Hochwasser verbietet jedoch eine Fahrt über den Fluss. Es wird beschlossen, nach Süden am \Dreifluss nach \Braucheln am See zu reisen, um dort über das Wasser zu setzen.

\section{Nach Süden}
Nach vier Tagen erreicht die Gruppe die kleine Stadt \Planis vorm \Grunarm. Dort ist man sehr beunruhigt. Wilde Kreaturen streifen bis an die Brücke heran, die nach Süden über den Fluss \Grunarm führt. Seit Tagen wagt sich niemand mehr auf die Felder im Süden. Die Streitmacht der Herren von \Rhingell zog vor 28 Tagen vorbei nach Süden. Seither kamen keine Nachrichten mehr die Straße zurück.
 
% kapitel3\textbf{•}.tex
\chapter{Das \Grunland}
\section{Das Fort}
Unter großer Vorsicht zieht die Gruppe in der Dämmerung den Weg weiter nach Süden. Als der Tag anbricht, erreichen sie ein verwüstetes und verlassenes Fort am Fluss. Sie verstecken sich dort tagsüber. In der Dämmerung ziehen sie weiter. Ein \Schattenjager wird zurück nach \Planis gesandt, um die Nachricht vom zerstörten Fort zu verbreiten.

\section{Gefahr am \Dreifluss}
Hinter dem Fort reicht der Wald über den Weg bis an den \Dreifluss. Bevor der Morgen kommt, erreicht die Gruppe eine zerstörte Brücke. Man teilt sich auf: ein Teil versucht, den Übergang herzurichten, ein zweiter will schauen, ob in der Nähe ein leichterer Übergang zu finden sei. Bei dem Versuch, ein Stück weiter waldeinwärts einen Übergang über die Schlucht zu finden, wird diese Gruppe von Wilden angegriffen. Anschließend wird die gesamte Gesandtschaft zwischen \Dreifluss und \Riesenwald zertreut. Das Wesen wird gestoßen und fällt in ein Loch.

\section{Im \Riesenwald}
Das Mädchen wacht auf. Die Sonne steht im Sie ist ganz allein im Wald. Sie geht weiter und weiter am Rand des Abgrundes, bis die Felsen niedriger werden. Unten zwängt sie sich zwischen Baumstümpfen und Steinen hindurch an einen Teich, wo viele wilde Kreaturen eine riesige, drachenähnliche Gestalt umzingelt haben. Es steht ein furchtbarer Kampf bevor.

Die wilden Kreaturen bemerkten das Mädchen nicht. Sie ging zu dem riesigen Wesen, es senkte seinen Kopf, sie streckte die Hand aus und legte sie auf die Stirn und es verwandelte sich in ein kleines Kind. Die wilden Kreaturen ließen von den beiden ab.

\section{Zurück zum Fluß}
\Nox, einer der \Schattenjager, hielt sich die ganze Zeit am Rand versteckt. Nachdem die Kreaturen fort waren, nahm er das \Sturmkind und das Kleine und brachte sie über den Fluss und die Schlucht zurück auf den Weg. Dort trafen sie weitere \Schattenjager und den Diener des Gesandten zum Herren der \Eisenmeister. Der Gesandte ist fort. Trotzdem zogen sie in der Dämmerung vorsichtig weiter nach Süden.

\section{Spuren}
An den Fällen des \Dreifluss sehen sie im Morgengrauen in der Ferne die Ruinen von \Darmon, und noch weiter dahinter die Rauchsäulen von \Dariom, der Stadt der Erzgräber am gegenüberliegenden Ufer des Sees. Sie ziehen auf dem Weg weiter nach Süden und durchqueren gegen Mittag ein zerstörtes Dorf. Die Asche ist kalt.

\section{\Braucheln}
Sie eilen weiter. In der Ferne sehen sie die Stadt \Braucheln am See. Rauch liegt über dem Tor zum Ufer, davor wogt ein Kampf. Als die \Schattenjager dazu stoßen, geht das \Sturmkind zu den Kreaturen. Sie lassen ab und ziehen in den Wald zurück.

% kapitel4.tex
\chapter{Die \Eisenmeister}
\section{Ankunft in \Braucheln}
In \Braucheln trafen sie den Bruder des Herren von \Rhingell. Er berichtet über Angriffen von Kreaturen aus den Wäldern, Die Stützpunkte der Holzfäller am Waldrand seien alle verlassen. Es wird seit Wochen kein Holz mehr geschlagen und keine Holzkohle gemacht und über den See geschafft. Es weilt bereits \Safir, ein Gesandter des Herrn der \Eisenmeister, in \Braucheln, der sich über die ausbleibenden Lieferungen erkundigen sollte.

\section{Auf dem Weg nach \Dariom}
Der Gesandte, einige \Schattenjager, das \Sturmkind und das Kleine aus dem \Riesenwald folgen dem Abgesandten der \Eisenmeister über das Wasser in die Stadt der \Eisenmeister.

Die Stadt der \Eisenmeister ist riesig und erstreckt sich von den Fällen des \Dreifluss, der mit seinem Wasser riesige Räder für die Erzschmieden antreibt, weit am Ufer und den Felsen nach Süden, wo die Handelswege in die Östlichen und Südlichen Lande verlaufen.

\section{Beim Herren der \Eisenmeister}
Der Herr der \Eisenmeister weist die Hilfegesuche ab. Die Streitereien der Leute aus \Rhingell gehen die \Eisenmeister nichts an. Man hat bereits genug Problem mit den Einfällen der Vielbeinigen über das nördliche Ödland und im Süden. Aber er bietet dem \Sturmkind an, es dem geheimen Herren des Erzes zu präsentieren.

\section{Der Herr des Erzes}
Die hohe Pforte des \Abaton öffnete ihre Flügel. Sie traten in eine Höhle, die so riesig war, dass man weder die Decke, noch links noch rechts ein Ende sehen konnte. Direkt vor den Besuchern begann ein Weg aus funkelnden Edelsteinen, an dem links und rechts silberne Säulen standen. Die Säulen reichten so weit hinauf, dass sie in der Finsternis verschwanden.
Am Ende der Säulenreihen erhob sich ein grauer Thron. Auf beiden Seiten standen riesige Statuten mit glänzenden Hämmern und Hacken anstelle von Händen. Auf dem Thron ruhte eine kleine Gestalt mit silbernen Gewändern und silbernen Haaren. „Kommt näher!“ donnerte es durch die Halle, ohne dass die Gestalt auf dem Thron sich geregt hätte. „Seht den geheimen Herrn des Berges!“ 

„Herr“, begann sie ängstlich. „Ich weiß nicht, wo ich hin soll. Und wo ich herkomme.“
„Kind“, antwortete der geheime Herr und es klang fast, als würde die Stimme lachen, „Was denkst du? Du bist doch keine Gabel, die jemand aus Versehen an den falschen Platz gelegt hat.“

Der geheime Herr des Berges Erzes sitzt tief im Berg bei der Stadt der Erzgräber in einer weiten Höhle. Bewacht von seiner Garde, zweimal zwölf Eisenwesen, sitzt er auf seinem eisernen Thron. Seine Augen sind trüb, aber er ist mit dem Stein verbunden und sieht Dinge, die sind und die sein können, aber nur schwach das Echo der Dinge, die einmal waren. Und dass die Gier nach Holz die Übergriffe im \Riesenwald erzeugt. Und er spürt vor allem, dass ein Kampf zwischen den Brüdern um die Herrschaft über \Rhingell bevorsteht. Darum müssen das \Sturmkind und die Gesandten schnell zurück nach \Toris am \Dreifluss. Wenn wieder Ordnung in \Rhingell einkehrt, wird auch wieder Holz in die Stadt der Erzgräber kommen. Der Gesandte \Safir der Erzgräber und ein Eisenwesen sollen sie begleiten.

\section{Zurück nach \Rhingell}
Auf dem Rückweg auf dem anderen Ufer des Dreiflusses durchqueren sie ein ödes Land. Sie kommen durch einen kleinen Posten der \Eisenmeister. Dort wird klar, dass der Wald vor langer Zeit auch auf dieser Seite des Flusses wuchs, ehe die \Eisenmeister ihn vollständig in ihrer Gier verbrannten. Dadurch ragt nun auch das östliche Ödland bis an den Fluss heran, wo früher hohe Bäume standen. Die \Eisenmeister kämpften in diesen Tagen auch gegen Kreaturen aus dem Wald und konnten dem nur Herr werden, in dem sie alles Holz auf dieser Seite des Flusses verbrannten.

% kapitel5.tex
\chapter{Pi}

% kapitel6.tex
\chapter{Ex}

% kapitel7.tex
\chapter{Sen}
\section{Bruderkrieg}
Das Hochwasser ist so weit zurückgegangen, dass der \Dreifluss bei \Toris wieder überquert werden kann. In \Toris erfährt die Gesandtschaft, dass der ältere Bruder bereits vorbeizog und verstärkt mit vielen Kämpfern aus \Toris nach \Rhin rückt. 

In \Rhingell hat sich der junge Herr in der Burg verschanzt. Die Stadt, das ganze Land hat sich gegen ihn gestellt. 

\section{Der Zug des \Papato}
Da rückt von Westen eine riesige Schar \Bangiri gegen die Stadt. Sie fordern die Herausgabe aller \Schattenjager in der Stadt.

\section{{\Eno}s Entscheidung}
\Eno geht allein zu den \Bangiri unter der Bedingung, dass \Papato abzieht. \Nox wird der neue Anführer der Schattenläufer.

\section{Die Rettung \Rhingell{s}}
Das \Sturmkind nimmt den älteren und jüngeren Herrn bei der Hand und verwandelt beide. Dann nimmt sie das Kleine aus dem \Riesenwald und erklärt es zum neuen Herrn von \Rhingell und \Grunland. Der junge König legt die Krone nieder um sich dem \Sturmkind anzuschließen. Er hat noch nie die Mauern \Rhin{s} verlassen. Dieses soll seine erste Reise werden.

\section{Der neue Herr der \Eisenmeister}
Das \Sturmkind geht mit \Safir und \Umbra zurück zum geheimen Herren der Erzgräber. Dieser wird eins mit dem Stein und der jetzige Herr der \Eisenmeister nimmt seinen Platz sein. Er verkündet, dass das \Sturmkind nach der Ruine von \Darmon gehen soll um dort zu erfahren, wo sein Platz ist.

\section{\Darmon}
Die Höhlen nach \Darmon kennen kein Licht. In der Finsternis begegnet das \Sturmkind drei \Daimon, die verschwinden, sobald sie diese anruft.
In den Ruinen von \Darmon ist die Geschichte der Welt in funkelnden Steinchen gelegt. Das \Sturmkind soll weiter nach Süden und Osten gehen, in den Mittelpunkt der Welt, wo die Säule steht, die den Himmel trägt. Wenn sie dort hinaufsteigt, soll ihren Platz in der Welt sehen.

\end{huge}
\end{document}
