\documentclass[12pt,a4paper,onecolumn,oneside,ngerman]{book}

% tern und enland
\newcommand{\Tern}{Tern}
\newcommand{\GrauerTurm}{Grauer Turm}
\newcommand{\GraueHerrin}{Graue Herrin}
\newcommand{\Ternweg}{{\Tern}weg}
\newcommand{\Sepa}{Seepa}
\newcommand{\Sturmkind}{Sturmkind}
\newcommand{\Daimon}{Daimon}
\newcommand{\Bangiri}{Bangiri}
\newcommand{\Pato}{Pato}
\newcommand{\Papato}{Papato}
\newcommand{\Arwed}{Alfried von {\Tern}}
\newcommand{\Enland}{Enland}
\newcommand{\Enlaender}{Enländer}
\newcommand{\Enlander}{Enländer}
\newcommand{\EnlanderGarde}{{\Enlander} Garde}
\newcommand{\Schattenjager}{Schattenläufer}
\newcommand{\Schattenlaufer}{Schattenläufer}
\newcommand{\Schajagi}{Scha-jagie}
\newcommand{\Berna}{Berna}
\newcommand{\AltBerna}{Alt {\Berna}}

% schattenläufer
\newcommand{\Eno}{Eno}
\newcommand{\Bomar}{Bomar}
\newcommand{\Dolo}{Dolo}
% Schwiegersohn von Eno
\newcommand{\Nox}{Nox}
% Frau von \Nox
\newcommand{\Mena}{Mena}
% Kind 1 von \Nox und \Mena
\newcommand{\Umbra}{Umbra}
% Kind 2+3 von \Nox und \Mena
\newcommand{\Ena}{Enna}
\newcommand{\Enno}{Enno}
% ?
\newcommand{\Lobo}{Lobo}
% Freund von \Umbra ?
\newcommand{\Tremor}{Tremor}
% Heilerin
\newcommand{\Salbana}{Salbana}
\newcommand{\Tea}{Teha}
\newcommand{\Marn}{Marn}
\newcommand{\Piedo}{Piedo}

% lobarn
\newcommand{\Lobarn}{Lobarn}
\newcommand{\Vester}{Vester}
\newcommand{\Naimo}{Naimo}

% nordmark
\newcommand{\Nordmark}{Nordmark}
\newcommand{\Bergmark}{Bergmark}
\newcommand{\Ipes}{Ipes}
\newcommand{\Bron}{Bron}
\newcommand{\Bornhold}{Bornhold}
\newcommand{\Arn}{Arn}
\newcommand{\Eislaufer}{Eisläufer}
\newcommand{\Eisbestien}{Eisbestien}

% rhinland
\newcommand{\Rhinland}{Rienland}
\newcommand{\Rhingell}{Riengell}
\newcommand{\Mundis}{Mundis}
\newcommand{\Helin}{Helin}
\newcommand{\Golrin}{Golrin}
\newcommand{\Rhinburg}{Rienburg}
\newcommand{\Rhin}{Rien}

% freiberge
\newcommand{\Freiberge}{Freiberge}
\newcommand{\Sudern}{Südern}
\newcommand{\Nachtspringe}{Nachtspringe}

% bergmark
\newcommand{\Kogida}{Kogida}

% dreifluss
\newcommand{\Dreifluss}{Dreifluß}
\newcommand{\Tars}{Tars}
\newcommand{\Toris}{Toris}
\newcommand{\Planis}{Planis}
\newcommand{\Grunarm}{Grünarm}

% grünland
\newcommand{\Grunland}{Grünland}
\newcommand{\Braucheln}{Braucheln}
\newcommand{\Darmis}{Darmis}
\newcommand{\Darmon}{Darmon}
\newcommand{\Riesenwald}{Riesenwald}

% eisenmeister
\newcommand{\Eisenland}{Eisenland}
\newcommand{\Eisenmeister}{Eisenmeister}
\newcommand{\Dariom}{Dariom}
\newcommand{\Abaton}{Abaton}
\newcommand{\Safir}{Safir}

% oedland
\newcommand{\Oedland}{Ödland}
\newcommand{\Staubteufel}{Staubteufel}


\begin{document}

% Sprache
\selectlanguage{ngerman}
  
% arabische Seitezahlen
\pagenumbering{arabic}
  
  
% Inhaltsverzeichnis
\tableofcontents
  
\clearpage{\pagestyle{empty}\cleardoublepage}
  
\raggedright 
  
%\begin{huge}
\paragraph{}
\textit{Es war einmal ein Sturm.
Niemand im {\Enland} hatte jemals so ein schlimmes Wetter gesehen.
Am Ende schlagen Blitze in den Grauen Turm mitten in der Stadt {\Tern} ein.
So endet die Zeit der Grauen Herrin, die von hier aus 99 Jahre über das {\Enland} und seine Bewohner herrschte.
}

\part{Rettung und Flucht nach dem Sturm}
\chapter[Der {\Schattenlaufer}]{}
\begin{enumerate}
  \item {\Eno} der {\Schattenlaufer} rettete ein Menschenkind aus den Trümmern des Grauen Turms, bevor ein riesiger {\Bangiri} es erschlagen kann
  \item {\Eno} flieht zu einem alten, verlassen Hof
  \item dort erwarten ihn {\Bomar} und {\Dolo}, zwei weitere {\Schattenlaufer}
\end{enumerate}

\chapter[Im Nest]{}
\begin{enumerate}
  \item südöstlich am {\Tern}See entlang erreichen sie das Dorf {\Beron}
  \item das Dorf ist heruntergekommen, viele Häuser verlassen
  \item der Orstvorsteher heißt {\Nox}
  \item in einem Versteck treffen sie auf weitere {\Schattenlaufer} mit Familien
  \item {\Nox} ist insgeheim auch {\Schattenlaufer}
  \item Das Menschenkind Mädchen erholt sich einige(?) Tage, spricht aber kein Wort und versteht selbst nichts
  \item {\Nox} zeigt ihr die Zahlen von 1 bis 7; es sind ihre ersten Worte im {\Enland}
  \item {\Nox} nennt das Mädchen "{\Sena}"; "{\Sena}"{ }ist die weibliche Form der rienländischen Zahl 7; sie wurde am 7.Tag der 7.Woche des Jahres gerettet wurde
  \item Im Dorf {\Beron} lebt der kleine {\Molitor} bei seinem Großvater
  \item {\Molitor} spielt mit {\Enna}, {\Enno} und {\Sena}
  \item Konflikte
  \begin{enumerate}
    \item {\Nox} und {\Eno} haben unterschiedliche Vorstellungen, wie die {\Schattenlaufer} handeln sollen
    \item {\Nox} ist der Lebensgefährte von {\Mena}, {\Nox}' Ziehtochter
    \item Die Zwillinge {\Enna} und {\Enno}, {\Nox}' Kinder, wollen {\Schattenlaufer} werden, aber ie dürfen nicht, weil sie zu jung sind
    \item {\Molitor} fragt seinem Großvater Löcher in den Bauch; woraufhin dieser schwerhaft meint, {\Molitor} wissen nichts
  \end{enumerate}
\end{enumerate}

\chapter[{\Pato}{s} Rache]{}
\begin{enumerate}
  \item {\Pato}, Anführer der {\Bangiri}, erfährt von seinem Großkrieger {\Oggo}, dass {\Schattenlaufer} in den Ruinen von {\Tern} waren
  \item {\Bangiri} und Garde durchstreifen das {\Enland} und suchen nach dem, was der Großkrieger beim Grauen Turm verloren hat
  \item Garde und {\Bangiri} kommen in das Dorf {\Beron} und treiben alle Leute zusammen
  \item Die {\Schattenlaufer} vertreiben die {\Bangiri}, als sie die Einwohner fortführen wollen
  \item Konflikte
  \begin{enumerate}
    \item {\Schattenlaufer} als Verteidiger des {\Enland}es gegen die alte Garde von Tern: {\Enland} gegen {\Enland} 
    \item {\Schattenlaufer} und die alteingesessenen Bewohner gegen {\Bangiri}, die von der Grauen Herrin ins {\Enland} geholt wurden
    \item Die Garde glaubt, dass die {\Schattenlaufer} oder das Wesen aus den Trümmern etwas mit der Zerstörung des Grauen Turms zu tun haben
  \end{enumerate}
\end{enumerate}

\chapter[Der Aufbruch]{}
\begin{enumerate}
  \item Die Angreifer wollen mit Verstärkung zurückkehren
  \item Die {\Schattenlaufer} beschließen, dass Dorf {\Beron} aufzugeben und nach Norden in die {\Nordmark} zu gehen
  \item Wer möchte, kann sich anschließen
  \item {\Nox}, {\Eno}, {\Dolo}, {\Tremor}, {\Enna} und {\Enno} sollen {\Sena} in die Stadt {\Lobarn} an der Grenze zum {\Rhinland} bringen
  \begin{enumerate}
    \item Die Gruppe soll in {\Lobarn} Vorsteher {\Vester} um Rat und Hilfe bitten und berichten, was im {\Enland} vor sich geht
    \item Der Großvater verrät {\Molitor}, dass er nicht der richtige Großvater ist; die Eltern von {\Molitor} sind verschollen, aber vielleicht gibt es im weiten {\Rhinland} noch irgendwo Verwandte
  \end{enumerate}
\end{enumerate}

\chapter[Der Weg nach Norden]{}
\begin{enumerate}
  \item An der Kreuzung mit dem alten {\Ternweg} trennen sich die Gruppen
  \item {\Lobarn}-Gruppe bemerkt {\Bangiri} aus Richtung {\Tern}, die nach Osten Richtung {\Lobarn} ziehen
  \item Sie werden an einer alten Mauer eingekreist und sitzen in der Falle
  \item Um {\Sena} erhebt sich ein Wirbelwind; als die Angreifer versuchen, die Windhose zu durchschreiten, werden sie fortgerissen. 
  \item Die {\Schattenlaufer} geben {\Sena} den Beinamen "{\Sturmkind}"
  \item {\Molitor} erscheint kurz darauf; er ist von der anderen Gruppe fortgelaufen
  \item Konflikte
  \begin{enumerate}
    \item {\Sena} erschrickt vor sich selbst
    \item {\Molitor} will mit ins {\Rhinland} ziehen, aber {\Nox} ist dagegen
  \end{enumerate}
\end{enumerate}

\chapter[Neuigkeiten]{}
\begin{enumerate}
  \item {\Lobarn} liegt hinter einer Brücke über dem Blauen {\Rhin} geschützt von dicken Steinmauern auf einem Hügel mit einem hohen Wachturm, der bis zu den Hügeln des {\Enland} blickt
  \item {\Eno} auf der Ratsversammlung von {\Lobarn} erzählt, was im {\Enland} vor sich geht und bittet um Beistand
  \item Es entsteht Unruhe und {\Vester}, der Ortsvorsteher, lässt sie in ein Dorf außerhalb bringen, wo sie bei {\Naimo}, der Vorsteherin, warten sollen
  \item Konflikte
  \begin{enumerate}
    \item Die Leute in {\Lobarn} zweifeln, dass die Zeit der Grauen Herrin wirklich vorbei ist
    \item Die Leute in {\Lobarn} fürchten {\Bangiri} und Garde
  \end{enumerate}
\end{enumerate}

\chapter[{\Naimo} erzählt die traurige Geschichte der Grauen Herrin und wie das {\Enland} verloren wurde]{}
\begin{enumerate}
  \item Viele {\Enlaender} wurden auf Befehl der Grauen Herrin, die vom Grauen Turm über das {\Enland} herrschte, durch ihre Helfer aus ihren Wohnungen im {\Enland} vertrieben
\end{enumerate}

\chapter[Die {\Bangiri} kommen]{}
\begin{enumerate}
  \item {\Bangiri} kommen nach {\Lobarn}
  \item {\Oggo}, Anführer der {\Bangiri}-Gruppe sucht {\Enlaender} und {\Schattenlaufer} aus {\Beron} und redet mit {\Vester}
  \item Außerdem sucht er das Wesen aus den Trümmern des Grauen Turms
  \item {\Vester} geht mit drei Boten heimlich zu den wartenden {\Schattenlaufer}n
  \item Konflikte
  \begin{enumerate}
    \item Die Stadt {\Lobarn} möchte nicht in die Streitereien zwischen {\Schattenlaufer}n, {\Bangiri}, {\Garde}, König und {\Enland} gezogen werden
    \item Seit letztem Winter gibt es einen neuen, jungen König namens {\Palemus}, dessen Position in dieser Lage unbekannt ist
  \end{enumerate}
\end{enumerate}

\chapter[Das Geschäft mit \Lobarn]{}
\begin{enumerate}
  \item {\Vester} will die Wahrheit über das Kind erfahren
  \item {\Vester} bietet drei Boten, die das Kind nach {\Rhingell} begleiten sollten, um es zum König zu bringen, wenn sich die {\Schattenlaufer} um einen kleinen {\Bangiri} kümmern, der vor einigen Tagen südlich der Stadt eingefangen wurde
  \item {\Vester} glaubt, wenn die {\Schattenlaufer} zusammen mit dem kleinen {\Bangiri} und dem {\Sturmkind} {\Sena} aus der Stadt verschwinden, dann hoffen sie, keine Probleme mit den verbliebenen {\Bangiri} , {\Rhingell}s neuem König oder der womöglich doch noch wiederkehrenden Grauen Herrin zu bekommen
.\linebreak
. Aber er schlägt ein, weil es {\Naimo} und {\Nox} raten. Sie fliehen in der Nacht weiter nach Osten, nach der Stadt {\Mundis} am Rand der {\Nordmark}.
  \item Konflikte
  \begin{enumerate}
    \item Das Königshaus von {\Rhingell} war mit der Grauen Herrin verwandt und hält {\Schattenlaufer} für Räuber und Diebe
    \item {\Eno} hasst alle {\Bangiri}, egal wie alt sie sind, und möchte den kleinen {\Bangiri} nicht mitnehmen
  \end{enumerate}  
\end{enumerate}

\part{Nach \Rhingell}
\chapter[Trennung auf Zeit]{}
\begin{enumerate}
  \item Vor {\Mundis} in der Stadt {\Blaufurt} Zusammentreffen mit {\Bomar}, {\Mena} und {\Umbra}
  \item Zusammen weiter
  \item In {\Mundis} Zusammentreffen mit Gesandten der {\Nordmark} unter {\Arn} von {\Ipes}
  \item {\Eno} verlässt die Gruppe und geht zurück nach {\Lobarn}
  \item {\Arn} sendet einen Späher mit {\Eno} und schickt einen Boten zum Herrn von {\Bornhold}; {\Nox} soll diesen begleiten und mit {\Enno} bei den {\Schattenlaufer} Familien am Rand der {\Nordmark} nach dem Rechten sehen
  \item Die 3 Boten aus {\Lobarn}, {\Dolo}, {\Tremor}, {\Umbra}, {\Sena}, {\Molitor} und {\Enna} ziehen mit {\Arn} und seinem Gefolge weiter nach {\Rhingell}
  \item Konflikte
  \begin{enumerate}
    \item In der {\Nordmark}, die vom Herrn von {\Bornhold} im Auftrag der Herren von {\Rhingell} regiert wird, sorgt man sich, weil sich im Nordwesten bei {\Eishold} düstere Wesen sammeln. Es sieht so aus, als bereiten die {\Eisleute} einen Überfall nach {\Rhingell} vor
  \end{enumerate}
  \item Hintergrund
  \begin{enumerate}
    \item Die Leute der {\Nordmark} verstehen sich gut mit den {\Schattenlaufer}n, denn viele Leute aus dem {\Enland} sind in die {\Nordmark} gegangen, um der Grauen Herrin zu entfliehen
  \end{enumerate}
\end{enumerate}

\chapter[Der Weg am Fluss]{}
\begin{enumerate}
  \item Weg von {\Mundis} nach {\Rhin} führt nah der Stadt {\Golrin} vorbei
  \item Dort hat sich im Krieg Ritter {\Galadin} niedergelassen
  \item Seine Leute ergreifen die Gesandtschaft und bringen sie nach {\Golrin}
  \item {\Galadin} empfängt die Gesandtschaft
  \item Konflikte
  \begin{enumerate}
    \item {\Galadin} folgt nicht mehr dem Königshaus; er stammt aus der Stadt und ist verbittert über ihre Zerstörung im Krieg
    \item {\Galadin} respektiert den Widerstand der {\Schattenlaufer} und möchte sie auf seine Seite ziehen; sie möchten aber nicht
  \end{enumerate}
\end{enumerate}

\chapter[Galadin und Arn erzählen vom Krieg mit den Eisleuten]{Der Krieg mit den Eisleuten}
\begin{enumerate}
  \item {\Galadin} und {\Arn} erzählen vom Krieg mit den Eisleuten
  \item Der  Raubzug der {\Eisleute} verwüstete die {\Nordmark}, die {\Bergmark} und den Westen {\Rhingell}s
  \item Die Herrin des Grauen Turms trat erst in den Kampf ein, als die {\Eisleute} ins {\Enland} zogen
  \item Auf den Feldern vor {\Tern} wurden die {\Eisleute} durch Zauberei besiegt
  \item {\Rhingell}s Westen blieb am Boden zerstört
  \item Hintergrund
  \begin{enumerate}
    \item Der Herr von {\Bornhold} lag im Streit mit der Herrin des Grauen Turms wegen ihres Zögerns 
  \end{enumerate}
\end{enumerate}
 
\chapter[Die Stadt der zwei Türme]{Die Stadt der zwei Türme}
\begin{enumerate}
  \item Nach Aufbruch folgt die Gesandtschaft wieder dem Fluss {\Rhin}
  \item Am anderen Ufer liegt die Stadt {\Helin} auf einem Hügel, geschützt von hohen Mauern. Im Krieg hatten die Leute das Glück, dass die {\Eislaufer} den Fluss nicht überqueren konnten. So blieben sie vom Schlimmsten verschont. Doch von den Türmen der Stadt konnte man den Rauch sehen, der tagelang aus der Stadt {\Golrin} aufstieg
  \item Der Gesandte und die Boten kommen mit {\Sena}, {\Molitor} und  dem jungen {\Bangiri} nach {\Rhin}, der Hauptstadt \Rhingell{s}
  \item Die {\Schattenlaufer} bleiben außerhalb der Stadt
  \item Die übrige Gesandtschaft wird von {\Habino}, dem Vertreter {\Lobarn}s, empfangen
  \item 3 Tage warten, ehe Empfang beim Hofmarschall
\end{enumerate}

\chapter[Die Geschichte des Königshauses]{Die letzten Könige von {\Rhingell}}
\begin{enumerate}
  \item {\Theodora}, eine Begleiterin von {\Arn} erzählt Geschichten vom alten König
  \item erklärt das Verhältnis zwischen dem {\Enland}, der Grauen Herrin und dem {\Rhinland}
  \item Erzählt vom Aufstand des {\Arwed} von {\Tern}, dem letzten Heerführer der Garde, der aus dem {\Enland} stammte
  \item Nach der Abwehr der {\Eisleute} hatte die Graue Herrin freie Hand im {\Enland} und wurde persönliche Beschützerin des alten Königs von {\Rhingell} wie auch seines Nachfolgers
  \item Der weise {\Valem}, ein Gelehrter aus der Stadt, der mit {\Habino} bekannt ist, erzählt vom neuen König {\Palemus} dem 13. und seinem Bruder {\Kalemus}, der eigentlich König sein sollte
\end{enumerate}

\chapter{Die Herren von \Rhingell}
\begin{enumerate}
  \item {\Arn} geht mit den Boten aus {\Lobarn}, {\Sena} und dem jungen {\Bangiri} zum Hofmarschall {\Isodoriin} und schließlich zum König
  \item {\Molitor} bleibt derweil bei {\Valem}
  \item Der König schickt alle fort
  \item der Hofmarschall berichtet von den {\Eisenmeister}n, die vielleicht besser helfen können
  \item Hofmarschall {\Isodoriin} vermittelt {\Arn} und {\Habino} ein Treffen mit dem Ältestenrat der Stadt
  \item Konflikte
  \begin{enumerate}
    \item Der junge König {\Kalemus} fürchtet das kleine Mädchen und hasst es, weil sein Erscheinen mit dem Verschwinden der Graue Herrin zusammenhängt, die seine Beschützerin war
    \item Mit dem {\Bangiri} will er nichts zu tun haben
  \end{enumerate}
\end{enumerate}

\chapter{Gerüchte in der Stadt}
\begin{enumerate}
  \item Auf dem Treffen des Ältestenrats wird beschlossen, heimlich einen Boten, den jungen {\Galeon} und seine Frau {\Demea} mit {\Sena}, dem {\Bangiri} und {\Molitor} zur Stadt {\Toris} am {\Dreifluss} zu schicken
  \item Sie sollen dann {\Kalemus} aufsuchen und ihm das Angebot des Ältestenrats erklären
  \item Die Gesandten sollen weiter zu den {\Eisenmeister}{n} ziehen, um diese um Beistand zu bitten
  \item {\Sena} soll zum Geheimen Herren der {\Eisenmeister} gehen, einer sehr weisen Kreatur, die ihr helfen kann, dass Geheimnis ihrer Herkunft zu klären
  \item Konflikte
  \begin{enumerate}
    \item Der Ältestenrat ist mit dem jungen König unzufrieden, weil er, wie sein Vorgänger, die Graue Herrin hat gewähren lassen und sich unter ihren Schutz gestellt hat und sich nicht um das Land kümmert (Sicherheit der Wege in die {\Bergmark} im Norden und im Westen ; Berichte über Räuber und die Neuigkeiten über {\Eisleute} in der {\Nordmark}, die die Gesandtschaft von {\Arn} überbracht hat)
    \item Man will den König durch seinen Bruder {\Kalemus} ersetzen 
  \end{enumerate}
\end{enumerate}

\chapter{\Toris}
\begin{enumerate}
  \item Die Boten aus {\Lobarn} kehren in ihre Stadt zurück
  \item Die {\Schattenlaufer}, die außerhalb der Stadt gewartet haben, ziehen mit ihnen ; {\Dolo} und {\Tremor} führen die Gruppe
  \item Die Stadt {\Toris} liegt wenige Tage von {\Rhin} entfernt an der Mündung des {\Rhingell} in den {\Dreifluss}. Auf der gegenüberliegenden Seite am {\Dreifluss} ist ein Handelsposten der {\Eisenmeister}
  \item Der Dreifluss kann wegen Hochwasser nicht überquert werden
  \item Konflikte
  \begin{enumerate}
    \item Es ist das erste Mal, dass {\Tremor} eine Gruppe mitführt ; {\Dolo} hält ihn für ungeeignet
  \end{enumerate}
\end{enumerate}

\chapter{Der Rat von \Toris}
\begin{enumerate}
  \item Die Gesandtschaft berät mit dem Ältestenrat der Stadt {\Toris}, der ebenfalls beunruhigt ist
  \item In {\Toris} leben viele {\Enland}er, die in den vergangenen Jahren ihre Heimat verließen
  \item man beschließt, am {\Dreifluss} nach {\Braucheln} am Großen See zu reisen, um dort über das Wasser zu den {\Eisenmeister}n zu fahren
\end{enumerate}

\chapter{Nach Süden}
\begin{enumerate}
  \item Wenige Tage später erreicht die Gruppe die Stadt {\Planis} vorm {\Grunarm}(Fluss)
  \item Die Streitmacht der Herren von {\Rhingell} zog vor 28 Tagen vorbei nach Süden, aber seither kamen keine Nachrichten mehr die Straße zurück
  \item Konflikte
  \begin{enumerate}
    \item In {\Planis} ist man sehr verängstigt. Wilde Kreaturen streifen bis an die Brücke heran, die nach Süden über den Fluss {\Grunarm} führt. Seit Tagen wagt sich niemand mehr auf die Felder
    \item Die Gesandten bekommen Angst
  \end{enumerate}
\end{enumerate}
 
\part{Das \Grunland}
\chapter{Das Fort}
Unter großer Vorsicht zieht die Gruppe in der Dämmerung den Weg weiter nach Süden.\linebreak
Als der Tag anbricht, erreichen sie ein verwüstetes und verlassenes Fort am Fluss. Sie verstecken sich dort tagsüber. In der Dämmerung ziehen sie weiter. Ein  {\Schattenlaufer} wird zurück nach {\Planis} gesandt, um die Nachricht vom zerstörten Fort zu verbreiten.

\chapter{Gefahr am \Dreifluss}
Hinter dem Fort reicht der Wald über den Weg bis an den {\Dreifluss}. Bevor der Morgen kommt, erreicht die Gruppe eine zerstörte Brücke.\linebreak
Man teilt sich auf: ein Teil versucht, den Übergang herzurichten, ein zweiter will schauen, ob in der Nähe ein Übergang ist. Bei dem Versuch, ein Stück weiter im Wald einen Übergang über die Schlucht zu finden, werden sie von Wilden angegriffen. Anschließend wird die gesamte Gesandtschaft zwischen {\Dreifluss} und {\Riesenwald} zerstreut.  {\Sena} wird gestoßen und fällt in ein Loch.

\chapter{Im \Riesenwald}
{\Sena} wacht auf. Sie ist allein im Wald. Sie geht weiter und weiter am Rand des Abgrundes, bis die Felsen niedriger werden. Unten zwängt sie sich zwischen Baumstümpfen und Steinen hindurch an einen Teich, wo viele wilde Kreaturen eine riesige, drachenähnliche Gestalt umringt haben.\linebreak
Die wilden Kreaturen bemerken {\Sena} nicht. Sie geht zu dem Wesen, sie legt die Hand auf die Stirn und im Sturm verwandelte sich die Kreatur in den jungen {\Bangiri}. Er gibt sich als {\Papato} zu erkennen, als Sohn des Anführers der {\Bangiri} im {\Enland}. Er wollte von seinen Leuten ausreißen, wurde aber auf dem Weg nach den Schwarzbergen von Spähern aus {\Lobarn} geschnappt. 

\chapter{Zurück zum Fluß}
{\Tremor} hat sich die ganze Zeit am Rand versteckt. Nachdem die Kreaturen ablassen, nimmt er {\Sena} und \Papato  und bringt sie zurück auf den Weg. Dort treffen sie weitere {\Schattenlaufer} und die Boten aus {\Rhingell}.

\chapter{Spuren}
An den Fällen des {\Dreifluss} sehen sie im Morgengrauen in der Ferne den Dunst der Nebelinsel und noch weiter dahinter die Rauchsäulen von {\Dariom}, der Stadt der {\Eisenmeister} am gegenüberliegenden Ufer des Sees.\linebreak
Sie ziehen weiter nach Süden und durchqueren gegen Mittag ein zerstörtes Dorf. Die Asche ist kalt.

\chapter{\Braucheln}
Sie eilen weiter. In der Ferne sehen sie die Stadt {\Braucheln} am See. Vor dem Tor haben sich Kreaturen aus dem Wald versammelt. {\Sena} geht in einem Sturmwirbel mit {\Papato} zu den Kreaturen. Sie lassen ab und ziehen sich in den Wald zurück.

\part{Die {\Eisenmeister}}
\chapter{Ankunft in {\Braucheln}}
In {\Braucheln} treffen sie den Bruder des Herren von {\Rhingell}, {\Kalemus}. Er berichtet über Angriffen von Kreaturen aus den Wäldern. Es wird seit Wochen kein Holz mehr geschlagen und keine Holzkohle gemacht und über den See geschafft. {\Safir}, Abgesandter des Herrn der {\Eisenmeister}, ist in {\Braucheln}, um sich über die ausbleibenden Lieferungen zu erkundigen.

\chapter{Auf dem Weg nach \Dariom}
Der Gesandte, einige {\Schattenlaufer}, das {\Sturmkind}, {\Molitor} und \Papato folgen dem Abgesandten der {\Eisenmeister} über das Wasser in die Stadt der {\Eisenmeister}.\linebreak
Die Stadt der {\Eisenmeister} erstreckt sich von den Fällen des {\Dreifluss}es, der mit seinem Wasser riesige Räder für die Erzschmieden antreibt, weit am Ufer und den Felsen nach Süden, wo die Handelswege in die Östlichen und Südlichen Lande verlaufen.

\chapter{Beim Herren der {\Eisenmeister}}
Der Herr der {\Eisenmeister} weist die Hilfegesuche ab. Die Streitereien der Leute aus {\Rhingell}  gehen die {\Eisenmeister} nichts an. Man hat genug Problem mit Wilden im nördlichen Ödland und den {\Staubteufel}n in den verlorenen Städten des Hochlandes.\linebreak 
Er bietet dem {\Sturmkind} an, es dem geheimen Herren des Erzes zu präsentieren, wenn sie helfen, das Rätsel der \Staubteufel zu lösen.

\chapter{Das trockene Land}
{\Molitor}löst das Rätsel der {\Staubteufel}.

\chapter{Der Herr des Erzes}
Der geheime Herr des Berges Erzes sitzt tief im Berg {\Abaton}. Bewacht von seiner Garde, zweimal vierzehn Eisenwesen, auf seinem eisernen Thron. Seine Augen sind trüb, aber er ist mit dem Stein verbunden und sieht Dinge, die sind und die sein können, aber nur schwach das Echo der Dinge, die einmal waren.\linebreak
Da {\Sena} und ihre Begleiter ihnen mit den Staubteufeln beistanden, schickt er sie mit einem weißen Schlüssel zurück zum König nach {\Rhingell}. Wenn Ruhe in {\Rhingell} einkehrt, wird auch wieder Holz in die Stadt der Erzgräber kommen. Der Gesandte \Safir der Erzgräber und ein Eisenwesen sollen sie begleiten.\linebreak
Wenn die Dinge geordnet sind, sollen sie wiederkehren. Bis dahin will der Herr des Eisens alles in seiner Macht tun, um Antworten auf ihre Fragen zu finden.

\chapter{Zurück nach {\Rhingell}}
Auf dem Rückweg auf dem anderen Ufer des Dreiflusses durchqueren sie ein ödes Land. Sie kommen durch einen kleinen Posten der {\Eisenmeister}.\linebreak 
Das Hochwasser ist so weit zurückgegangen, dass der {\Dreifluss} bei {\Toris} wieder überquert werden kann.

\part{Der Hexenmeister}
\chapter{Empfangen und wieder nicht}
Hofmarschall {\Isodoriin} hört sie an. Er will sie dieses Mal nur zum König lassen, wenn sie sich würdig erweisen. Dazu sollen sie nach Norden, an der Grenze zwischen {\Nordmark} und {\Bergmark},um einer Räuberbande das Handwerk zu legen. Als Beweis sollen sie das Herz des Hexenmeisters vorlegen.\linebreak
{\Arn} ist verzweifelt. Vor der Stadt trifft er auf {\Tremor}, {\Umbra} und {\Enna}. Sie beschließen, nach Norden zu ziehen.

\chapter{Die Kinder von Kogida}
Die Räuber rauben Kinder aus der Stadt {\Kogida} und umliegenden Dörfern für den Hexenmeister. Mit geheimnisvoller Hilfe können sie die Räuber vertreiben und bringen den schwarzen Hexenmeister {\Denner} in ihre Gewalt. Er bleibt jedoch am Leben. Sie kehren zurück nach {\Rhin}.

\part{Der Bruderkrieg}
\chapter{Bruderkrieg}
Sie kehren gerade rechtzeitig zurück. {\Kalemus}, der ältere Bruder, hat die Geduld verloren. Er ist zusammen mit vielen Kämpfern aus {\Toris} nach {\Rhin} gezogen. In {\Rhingell} hat sich der junge König mit seiner Leibgarde, den 49 weißen Rittern, in der Burg verschanzt. Die Stadt, das ganze Land hat sich gegen ihn gestellt.\linebreak
Eine große Verschwörung wird von {\Nox} enthüllt.

\chapter{Der Zug aus dem Süden}
Aus dem Süden rücken riesige Scharen von Wilden aus den Waldlanden und {\Bangiri} gegen die Stadt. Sie sind wütend. Sie wollen Rache für das Waldland und die Verluste im {\Enland}.

\chapter{{\Eno}s Entscheidung}
{\Eno} und {\Papato} gehen zu den {\Bangiri}.\linebreak
{\Nox} wird neuer Anführer der {\Schattenlaufer} und ein neuer Vorsitz wird gewählt.

\chapter{Die Rettung \Rhingell{s}}
Der junge König {\Palemus} legt die Krone nieder, um sich dem {\Sena} anzuschließen. Er hat noch nie die Mauern \Rhin{s} verlassen. Dieses soll seine erste Reise werden.

\chapter{Der neue Herr der {\Eisenmeister}}
Das {\Sturmkind} geht mit {\Safir} und {\Umbra} zurück nach {\Abaton} zum geheimen Herren der {\Eisenmeister}.\linebreak
Er hat nachgedacht. Seine Zeit ist um. {\Sena} soll die Nebelinsel von \Darmon besuchen. Wenn überhaupt, kann so dort erfahren, wo sie herkommt. Er wird eins mit dem Stein und der jetzige Herr der {\Eisenmeister} nimmt seinen Platz sein.

\chapter{\Darmon}
Vom {\Abaton} führt ein geheimer Weg zur Nebelinsel {\Darmon} am Wasserfall. In der Mitte befindet sich ein Kreis aus verkohlten Bäumen.\linebreak
Als die Reisenden den Kreis betreten, fallen sie im Traum in der Zeit, die lange vorbei ist. Sie sehen, dass der Wald vor langer Zeit auch auf dieser Seite des Flusses bis weit auf die Hochebene wuchs, ehe die {\Eisenmeister} ihn vollständig in ihrer Gier verbrannten. Die {\Eisenmeister} kämpften in diesen Tagen gegen wilde Kreaturen aus dem Wald und konnten dem nur Herr werden, in dem sie alles Holz auf dieser Seite des Flusses verbrannten.\linebreak
{\Sena} war schon vor dem Sturm ins {\Enland} gekommen. Die Geister weisen {\Sena} einen Weg nach Südosten, zum Mittelpunkt der Welt. Dort steht die Säule, die den Himmel trägt. Wenn sie dort hinaufsteigt, soll einen Platz sehen, der für sie bestimmt ist.

%\end{huge}

\end{document}
